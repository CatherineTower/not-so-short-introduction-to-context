%%% vim:set filetype=context tw=72 : %%%
%%% File:     TitlePage.mkiv
%%% Component:Cover, credits and table of contents
%%% Author:   Joaquín Ataz-López
%%% Begun:    April 2020
%%% Concluded:August 2020
%%%
%%% Edited with: Emacs + AuTeX - And at times with vim + context-plugin
%%%

\environment ../introCTX_env.mkiv

\startcomponent a01_TitlePage.mkiv

%%%%%%%%%%%%%%%%%%%%%%%%%%%%%%%%%%%%%%%%%%%%%%%%%%%%%%%%%%%%%%%%%%%%
% Cover
% I have copied the document on luatex and context (included in the
% documentation on context. I have adapted it slightly since my title
% has three lines while the original has only 2, and I have changed the
% background colour).
%%%%%%%%%%%%%%%%%%%%%%%%%%%%%%%%%%%%%%%%%%%%%%%%%%%%%%%%%%%%%%%%%%%%

  \setvariables
    [document]
    [titleA=A not so short,
    titleB=introduction,
    titleC=to \ConTeXt\ Mark~IV,
    author=Joaquín Ataz-López]

  \startpagemakeup

    \startMPcode

      StartPage ;

        fill Page enlarged 5mm withcolor \MPcolor{purple} ;
        draw anchored.lrt(image(draw textext("\getvariable{document}{titleA}")  xsized(.750PaperWidth)      withcolor white),(lrcorner Page) shifted (-PaperWidth/20, PaperWidth/2.2)) ;
        draw anchored.lrt(image(draw textext("\getvariable{document}{titleB}")  xsized(.750PaperWidth)      withcolor white),(lrcorner Page) shifted (-PaperWidth/20, PaperWidth/3.1)) ;
        draw anchored.lrt(image(draw textext("\getvariable{document}{titleC}")  xsized(.750PaperWidth)      withcolor white),(lrcorner Page) shifted (-PaperWidth/20, PaperWidth/5)) ;
        draw anchored.urt(image(draw textext("\getvariable{document}{author}")   xsized(.375PaperWidth) rotated 90 withcolor white),(urcorner Page) shifted (-PaperWidth/20,-PaperWidth/20)) ;

        setbounds currentpicture to Page ;

      StopPage ;

    \stopMPcode

  \stoppagemakeup

%%%%%%%%%%%%%%%%%%%%%%%%%%%%%%%%%%%%%%%%%%%%%%%%%%%%%%%%%%%%%%%%%%%%
% Credit titles
%%%%%%%%%%%%%%%%%%%%%%%%%%%%%%%%%%%%%%%%%%%%%%%%%%%%%%%%%%%%%%%%%%%%

  {
    \page[yes]
    \page[blank]\parindent0pt
    \ \vfill

    \switchtobodyfont[small]
    {\bf A not so short introduction to \ConTeXt\ Mark~IV}\\
   Version 1.6 [\RevisionDate]\\ 
    \blank

    % I wanted to use the copyleft symbol, but I have not found out how
    % to generate it. With \mirror{\copyright} the symbol is generated,
    % but a rather annoying linebreak is inserted after it. Copy and
    % paste from the internet doesn't work and I don't know why. (It is
    % not the first time this happens).

    {\copyright} 2020-2021, Joaquín Ataz-López\\
    \blank
    Original title: Una introducción (no demasiado breve) a \ConTeXt\ Mark~IV\\
    English Translation: A good friend who wishes to remain anonymous.

    The author of this text (and its English translator) authorises its
    free distribution and use, including the right to copy and
    redistribute this document in digital format on condition that its
    authorship is acknowledged, and that it is not included in any
    software package or suite, or in documentation whose conditions of
    use or distribution do not include the free right of recipients to
    copy and distribute.  Authorisation is likewise given for
    translation of the document, provided that the authorship of the
    original text is indicated, and that the translated text is
    distributed under the FDL licence of the {\em Free Software
    Foundation}, the {\em Creative Commons} licence that authorises
    copying and redistribution, or a similar licence.

    The above notwithstanding, publication or marketing or translation
    of this document in paper form will require the author's express
    authorisation in writing.

    % It is not a totally free licence since it does not allow it to be
    % marketed in paper form. The point is to pevent some "smartass"
    % from publishing this document as a book. However, apart from this
    % restriction, I believe that the licence continues to reflect quite
    % well the spirit of the FDL licence, in considerably less space.

    \setupnotation[numberconversion=set 0]
    Version history:

    \startitemize[packed]\smallbodyfont

      \item \date[y=2020, m=8, d=18]: Version 1.0 (Spanish only):
      Original document.

      \item \date[y=2020, m=8, d=23]: Version 1.1 (Spanish only):
        Correction of minor errors, typos and misunderstandings by the
        author.

      \item \date[y=2020, m=9, d=3]: Version 1.15 (Spanish only): More
        errors, typos and misunderstandings.

      \item \date[y=2020, m=9, d=5]: Version 1.16 (Spanish only): More
        errors, typos and misunderstandings as well as some very minor
        changes to make the text clearer (I hope).

      \item \date[y=2020, m=9, d=6]: Version 1.17 (Spanish only): The
        number of minor errors I am finding is incredible. I would just
        need to stop re-reading the document to find no more!

      \item \date[y=2020, m=10, d=21]: Version 1.5 (Spanish only):
        Introduction of suggestions and correction of errors reported by
        NTG-context users.

      \item \RevisionDate: Version 1.6: Corrections suggested after a
      new and careful reading of the document, on the occasion of its
      translation to English. This is the first version in English.

    \stopitemize

  }

  %%%%%%%%%%%%%%%%%%%%%%%%%%%%%%%%%%%%%%%%%%%%%%%%%%%%%%%%%%%%%%%%%%%%
  % Table of contents
  %%%%%%%%%%%%%%%%%%%%%%%%%%%%%%%%%%%%%%%%%%%%%%%%%%%%%%%%%%%%%%%%%%%%

  \starttitle
    [title={Table of Contents}]

    % Header
    \setupheadertexts
      [\tfx Table of Contents]
      [\tfx\pagenumber]

    % Format of entries in the ToC
    \setuplist
      [part]
      [
        before={\blank[2*big]},
        after={\blank[big]},
        style=\bfa,
        aligntitle=yes,
      ]

    \setuplist
      [chapter]
      [
        before=\blank,
        style=\bf,
        margin=.5cm,
        aligntitle=yes
      ]

    \setuplist
      [section]
      [
        aligntitle=yes,
        style=\tfx,
        margin=1.5cm
      ]

    % Let us define our list (called a Toc)
    \definecombinedlist
      [Toc]
      [part, chapter, section]
      [level=subsection, alternative=c]

    % and insert it in the context of smaller interline space
    \start
      \setupwhitespace[none]
      \switchtobodyfont[11pt]
      \setupinterlinespace[small]
      \placeToc
    \stop

  \stoptitle

\stopcomponent

%%% Local Variables:
%%% mode: ConTeXt
%%% mode: electric-indent
%%% coding: utf-8-unix
%%% TeX-master: "../introCTX.mkiv"
%%% End:

