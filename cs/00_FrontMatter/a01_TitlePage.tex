%%% vim:set filetype=context tw=72 : %%%
%%% File:     TitlePage.mkiv
%%% Component:Cover, credits and table of contents
%%% Author:   Joaquín Ataz-López
%%% Begun:    April 2020
%%% Concluded:August 2020
%%%
%%% Edited with: Emacs + AuTeX - And at times with vim + context-plugin
%%%

\environment ../introCTX_env.mkiv

\startcomponent a01_TitlePage.mkiv

%%%%%%%%%%%%%%%%%%%%%%%%%%%%%%%%%%%%%%%%%%%%%%%%%%%%%%%%%%%%%%%%%%%%
% Cover
% I have copied the document on luatex and context (included in the
% documentation on context. I have adapted it slightly since my title
% has three lines while the original has only 2, and I have changed the
% background colour).
%%%%%%%%%%%%%%%%%%%%%%%%%%%%%%%%%%%%%%%%%%%%%%%%%%%%%%%%%%%%%%%%%%%%

  \setvariables
    [document]
    [titleA=\bf Nepříliš krátký,
     titleB=\bf úvod,
     titleC=\bf do \ConTeXt{}u Mark~IV,
     author=\bf Joaquín Ataz-López]

  \startpagemakeup

    \startMPcode

      StartPage ;

        fill Page enlarged 5mm withcolor \MPcolor{purple} ;
        draw anchored.lrt(image(draw textext("\getvariable{document}{titleA}")  xsized(.750PaperWidth)      withcolor white),(lrcorner Page) shifted (-PaperWidth/10, PaperWidth/2)) ;
        draw anchored.lrt(image(draw textext("\getvariable{document}{titleB}")  xsized(.750PaperWidth)      withcolor yellow),(lrcorner Page) shifted (-PaperWidth/10, PaperWidth/4)) ;
        draw anchored.lrt(image(draw textext("\getvariable{document}{titleC}")  xsized(.750PaperWidth)      withcolor yellow),(lrcorner Page) shifted (-PaperWidth/10, PaperWidth/8)) ;
        draw anchored.urt(image(draw textext("\getvariable{document}{author}")  xsized(.450PaperWidth) rotated 90 withcolor white),(urcorner Page) shifted (-PaperWidth/10,-PaperWidth/20)) ;

        setbounds currentpicture to Page ;

      StopPage ;

    \stopMPcode

  \stoppagemakeup

%%%%%%%%%%%%%%%%%%%%%%%%%%%%%%%%%%%%%%%%%%%%%%%%%%%%%%%%%%%%%%%%%%%%
% Credit titles
%%%%%%%%%%%%%%%%%%%%%%%%%%%%%%%%%%%%%%%%%%%%%%%%%%%%%%%%%%%%%%%%%%%%

  {
    \page[yes]
    \page[blank]\parindent0pt
    \ \vfill

    \switchtobodyfont[small]
    {\bf Nepříliš krátký úvod do \ConTeXt{}u Mark~IV}\\
   Verze 1.6 [\RevisionDate]\\ 
   Český překlad: leden 2022\\ 
    \blank

    % I wanted to use the copyleft symbol, but I have not found out how
    % to generate it. With \mirror{\copyright} the symbol is generated,
    % but a rather annoying linebreak is inserted after it. Copy and
    % paste from the internet doesn't work and I don't know why. (It is
    % not the first time this happens).

    {\copyright} 2020--2022, Joaquín Ataz-López\\
    \blank
    Původní název: Una introducción (no demasiado breve) a~\ConTeXt\ Mark~IV\\
    Anglický překlad: Joaquínův dobrý kamarád, který si přeje zůstat v~anonymitě.\\
%    {\bf Nepříliš krátký úvod do \ConTeXt{}u Mark~IV}\\
    Český překlad: 
	Tomáš Hála, 
	Aleš Ďurčanský, 
	Jan Janča, 
	Natália Várošová, 
	Robert Blaha,
	Tamara Kocurová, 
	Zdeněk Svoboda

Autor tohoto textu i~překladatelé české verze povolují jeho volné šíření
a~používání, včetně práva kopírovat a~šířit tento dokument v~digitální podobě,
pod podmínkou, že bude zachování uvedení autorství původního textu
i~překladu, a~že nebude součástí žádného softwarového balíku nebo sady nebo
dokumentace, jejíž podmínky používání nebo šíření nezahrnují právo příjemců
na volné kopírování a~šíření.

Stejně tak je povolen překlad dokumentu za předpokladu, že je uvedeno
autorství původního textu (případně překladu) a~že přeložený text je šířen
pod licencí FDL nadace {\em Free Software Foundation}, licencí {\em Creative
Commons}, která povoluje kopírování a~další šíření, nebo podobnou licencí.

%    The author of this text (and its English translator) authorises its
%    free distribution and use, including the right to copy and
%    redistribute this document in digital format on condition that its
%    authorship is acknowledged, and that it is not included in any
%    software package or suite, or in documentation whose conditions of
%    use or distribution do not include the free right of recipients to
%    copy and distribute.  Authorisation is likewise given for
%    translation of the document, provided that the authorship of the
%    original text is indicated, and that the translated text is
%    distributed under the FDL licence of the {\em Free Software
%    Foundation}, the {\em Creative Commons} licence that authorises
%    copying and redistribution, or a similar licence.

Bez ohledu na výše uvedené, zveřejnění, marketing nebo překlad
tohoto dokumentu v~listinné podobě bude vyžadovat výslovný písemný 
souhlas původního autora, případně i~autorů překladu.

%    The above notwithstanding, publication or marketing or translation
%    of this document in paper form will require the author's express
%    authorisation in writing.

    % It is not a totally free licence since it does not allow it to be
    % marketed in paper form. The point is to pevent some "smartass"
    % from publishing this document as a book. However, apart from this
    % restriction, I believe that the licence continues to reflect quite
    % well the spirit of the FDL licence, in considerably less space.

    \setupnotation[numberconversion=set 0]
    Historie verzí (výběr):

    \startitemize[packed]\smallbodyfont

      \item \date[y=2020, m=8, d=18]: Verze 1.0 (pouze ve španělštině):
      Původní verze.

% Nevýznamé, nezařazujeme (TH)
%       \item \date[y=2020, m=8, d=23]: Version 1.1 (Spanish only):
%        Correction of minor errors, typos and misunderstandings by the
%        author.
%
%      \item \date[y=2020, m=9, d=3]: Version 1.15 (Spanish only): More
%        errors, typos and misunderstandings.
%
%      \item \date[y=2020, m=9, d=5]: Version 1.16 (Spanish only): More
%        errors, typos and misunderstandings as well as some very minor
%        changes to make the text clearer (I hope).
%
%      \item \date[y=2020, m=9, d=6]: Version 1.17 (Spanish only): The
%        number of minor errors I am finding is incredible. I would just
%        need to stop re-reading the document to find no more!

      \item \date[y=2020, m=10, d=21]: Verze 1.5 (pouze ve španělštině):
        Verze po zapracování připomínek a~oprav uživatelů \ConTeXt{}u
	 	diskusního listu {\em NTG-context}.

      \item \RevisionDate: Verze 1.6: 
	Opravy navržené nové a~pečlivé revizi dokumentu v~souvislosti s~jeho
	překladem do angličtiny První verze anglického překladu.

    \stopitemize

  }

  %%%%%%%%%%%%%%%%%%%%%%%%%%%%%%%%%%%%%%%%%%%%%%%%%%%%%%%%%%%%%%%%%%%%
  % Table of contents
  %%%%%%%%%%%%%%%%%%%%%%%%%%%%%%%%%%%%%%%%%%%%%%%%%%%%%%%%%%%%%%%%%%%%

  \starttitle
    [title={Obsah}]

    % Header
    \setupheadertexts
      [\tfx Obsah]
      [\tfx\pagenumber]

    % Format of entries in the ToC
    \setuplist
      [part]
      [
        before={\blank[2*big]},
        after={\blank[big]},
        style=\bfa,
        aligntitle=yes,
      ]

    \setuplist
      [chapter]
      [
        before=\blank,
        style=\bf,
        margin=.5cm,
        aligntitle=yes
      ]

    \setuplist
      [section]
      [
        aligntitle=yes,
        style=\tfx,
        margin=1.5cm
      ]

    % Let us define our list (called a Toc)
    \definecombinedlist
      [Toc]
      [part, chapter, section]
      [level=subsection, alternative=c]

    % and insert it in the context of smaller interline space
    \start
      \setupwhitespace[none]
      \switchtobodyfont[11pt]
      \setupinterlinespace[small]
      \placeToc
    \stop

  \stoptitle

\stopcomponent

%%% Local Variables:
%%% mode: ConTeXt
%%% mode: electric-indent
%%% coding: utf-8-unix
%%% TeX-master: "../introCTX.mkiv"
%%% End:

