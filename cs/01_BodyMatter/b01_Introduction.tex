%%% File:        b01_Introduction.mkiv
%%% Author:      Joaquín Ataz-López
%%% Begun:       March 2020
%%% Concluded:   March 2020
%%% Contents:    First chapter of the introduction to ConTeXt: a general
%%%              overview of the system. The contents were partly
%%%              by the presentation of LaTeX
%%%              by Kopka and Daly in Chapter 1 of their Guide to
%%%              LaTeX
%%%
%%% Edited with: Emacs + AuTeX - And at times with vim + context-plugin
%%%

\environment ../introCTX_env.mkiv

\startcomponent b01_Introduction.mkiv

\startchapter 
  [
    title=\ConTeXt: a general overview,
    reference=cap:panorama
  ]

\TocChap

\startsection 
  [title=What is \ConTeXt\ then?]

\ConTeXt\ is a {\em typesetting system}, or in other words: an extensive
set of tools aimed at giving the user absolute and complete control over
the appearance and presentation of a specific electronic document
intended for print on paper or to be shown on screen. This chapter
explains what this means.  But first, let us highlight some of the
characteristics of \ConTeXt.

\startitemize

  \item There are two {\em flavours} of \ConTeXt\ known as Mark~II and
  Mark~IV respectively. \ConTeXt\ Mark~II is frozen, i.e. it is
  considered to be an already fully-developed language that is not
  intended to have further changes or new things added. A new version
  would appear only in the case where some error needs to be corrected.
  \ConTeXt\ Mark~IV, on the other hand, continues to be developed so
  that new versions appear from time to time that introduce some
  improvement or additional utility. But, although still in development,
  it is a very mature language in which changes introduced by the new
  versions are quite subtle and exclusively affect the low level
  operation of the system. For the average user these changes are
  totally transparent; it is as if they did not exist. Although both
  {\em flavours} have much in common, there are also some incompatible
  features between them. Hence this introduction focuses only on
  \ConTeXt\ Mark~IV.

  \item \ConTeXt\ is software {\em libre} (or free software, but not
  just in the sense of {\em gratis}). The program properly speaking
  (that is, the complex of computer tools that make up \ConTeXt), is
  distributed under the {\em GNU General Public Licence}. The
  documentation is offered under the \quotation{{\em Creative Commons}}
  licence that allows it to be freely copied and distributed.

  \item \ConTeXt\ is neither a word processor program nor a text editing
  program, but a collection of tools aimed at {\em transforming} a text
  we have previously written in our favourite text editor. Therefore,
  when we work with \ConTeXt:

  \startitemize

    \item We begin by writing one or more text files with any kind of
    text editor.

    \item In these files, along with the text that makes up the contents
    of the document, there is a range of instructions that tell
    \ConTeXt\ about the appearance that the final document generated
    from the original text files must have. The complete set of
    \ConTeXt\ instructions, in fact, is a {\em language}; and since this
    language allows one to {\em program} the typographical
    transformation of a text, we can say that \ConTeXt\ is a {\em
    typographical programming language}.

    \item Once we have written the source files, these will be processed
    by a program (also called \MyKey{context}\footnote{\ConTeXt\ is both
    a language and a program at the same time (besides being other
    things). This fact, in a text like the current one, poses the
    problem that at times we have to distinguish between these two
    aspects. This is why I have adopted the typographical convention of
    writing \quotation{ConTeXt} with its logo (\ConTeXt) when I want to
    refer exclusively to the language, or to both the language and the
    program. However, when I want to refer exclusively to the program, I
    will write \MyKey{context} all in lower case and in the monospaced
    type that is typical of computer terminals and typewriters. I will
    also use this type for examples and mentions of commands belonging
    to this language.}), which will generate a PDF file from them ready
    to be sent to a print shop or to be shown on screen.

  \stopitemize

  \item In \ConTeXt, therefore, we must differentiate between the
  document we are writing, and the document that \ConTeXt\ generates. To
  avoid any doubts, in this introduction I will call the text document
  that contains formatting instructions the {\em source file}, and the
  PDF document generated by \ConTeXt\ from the source file I will call
  the {\em final document}.

\stopitemize

The above basic points will be further developed below.

\stopsection

\startsection [title=Typesetting texts]

Writing a document (book, article, chapter, leaflet, print out, paper
...) and putting it together typographically are two very different
activities. Writing the document is much the same as drafting it; this
is done by the author who decides on its content and structure. The
document created directly by the author, just as he or she wrote it, is
called the {\em manuscript}. By its very nature, only the author or
those permitted to read it have access to the manuscript. Its
dissemination beyond this intimate group requires the manuscript to be
{\em published}. Today, publishing something -- in the etymological
sense of making it \quotation{accessible to the public} -- is as simple
as putting it on the internet, available to anyone who finds it and
wants to read it. But until relatively recently, publication was a
cost-intensive process dependent on certain professionals specialised in
it, only accessed by those manuscripts which, because of their content,
or because of their author, were considered to be particularly
interesting. And even today we tend to reserve the word {\em
publication} for this kind of {\em professional publication} where the
manuscript undergoes a series of transformations in its appearance aimed
at improving the {\em legibility} of the document. This series of
transformations is what we call {\em typesetting}.

The aim of typesetting is -- generally speaking, and leaving aside
advertisement-type texts that seek to attract the reader's attention --
to produce documents with the greatest {\em legibility}, meaning the
quality of the printed text that invites or facilitates its reading and
ensures that the reader feels comfortable with it. Many things
contribute to this; some, of course, have to do with the document's {\em
contents}: (quality, clarity, organisation...), but others depend on
things like the type and dimensions of the font used, the use of white
space in the document, visual separation between paragraphs, etc. In
addition, there are other kinds of resources, not so much of the graphic
or visual kind, such as the presence or otherwise in the document of
specific aids to the reader like page headers and footers, indexes,
glossaries, use of bold type, margin headings, etc. The knowledge and
correct handling of all the resources available to a typesetter could be
called the \quotation{art of typesetting} or the \quotation{art of
printing}.

Historically, and until the advent of the computer, the tasks and roles
of writer and typesetter were kept quite distinct. The author wrote by
hand or on a 19th century machine called a typewriter, the typographical
resources of which were even more limited than those who wrote by hand;
and then the writer gave the originals to the publisher or printer who
transformed them to obtain the printed document.

Today, computer science has made it easier for the author to decide on
the composition down to the last detail. However, this does not do away
with the fact that the qualities that a good author needs are not the
same as those needed by a good typesetter. Depending on the kind of
document being dealt with, the author needs an understanding of the
subject matter being written about, clarity of exposition,
well-structured thinking that allows for the creation of a
well-organised text, creativity, a sense of rhythm, etc. But the
typesetter has to combine a good knowledge of the conceptual and
graphical resources at his or her disposal, and sufficient good taste to
be able to use them harmoniously.

With a good word processing program\footnote{According to a rather old
convention, we make a distinction between {\em text editors} and {\em
word processors}. The early kinds of text editing programs dealt with
unformatted text files, while the other kind worked with binary files of
formatted text.} it is possible to achieve a reasonably good
typographically prepared document. But word processors, generally
speaking, are not designed for typesetting and the results, although
they may be correct, are not comparable to the results obtainable with
other tools designed specifically to control the composition of the
document. In fact, word processors are how typewriters evolved, and
their use, to the extent that these tools mask the difference between
the authorship of the text and its typesetting, tends to produce
unstructured and typographically inadequate texts. On the contrary,
tools like \ConTeXt\ have evolved from the printing press; they offer
many more composition possibilities and above all, it is not possible to
learn how to use them without also acquiring, along the way, many
notions relating to typesetting. This is the difference from word
processors, which someone can use for many years without learning a
single thing about typography.

\stopsection

\startsection [title=Markup languages]

In the days before computers, as I said before, the author prepared the
manuscript by hand or typewriter and handed it to the publisher or
printer who was responsible for the transformation of the manuscript
into the final printed text. Although the author had relatively little
involvement in the transformation, he or she did maintain some
intervention by pointing out, for example, that certain lines of the
manuscript were the titles of its various parts (chapters, sections
...), or by indicating that certain things should be highlighted
typographically in some way. These indications were made by the author
in the manuscript itself, sometimes expressly, and at other times
through certain conventions that continued to develop over time. For
example, the chapters always began on a new page by inserting several
blank lines before the title, underlining it, writing it in capital
letters, or framing the text to be highlighted between two underscores,
increasing the indentation of a paragraph, etc.

To put it briefly, the author {\em marked up} the text in order to
provide indications relating to how it should be typeset. Then later,
the editor would handwrite other indications on the text for the
printer, such as, for example, the font to be used, and its size.

Today, in a computerised world, we can continue to do this to generate
electronic documents through what is called a {\em markup language}.
These kinds of languages use a series of {\em marks} or indications that
the program processing the file containing them knows how to interpret.
Probably the best known markup language today is HTML, since most web
pages today are based on it. An HTML page contains the text of a web
page, along with a series of marks that tell the browser program that
loads the page how it should display it. The HTML markup that web
browsers understand, together with instructions about how and where to
use them, is called the \quotation{HTML language}, which is a {\em
markup language}. But as well as HTML, there are many other markup
languages; in fact they are booming, and so XML, which is the markup
language {\em par excellence}, is found everywhere today and is in use
for pretty much everything: for database design, for the creation of
specific languages, transmission of structured data, application
configuration files, etc. There are also markup languages intended for
graphic design (SVG, TikZ or MetaPost), maths formulas (MathML), music
(Lilypond and MusicXML), finance, geomatics, etc. And of course there
are also markup languages aimed at the typographical transformation of
text, and among these, \TeX\ and its derivatives stand out.

With regard to the {\em typographical} markup that indicates how a text
should look, there are two kinds that we can refer to: {\em purely
typographical markup} and {\em conceptual markup} or, if you prefer,
{\em logical markup}. Purely typographical markup is limited to
indicating precisely what typographical resource should be used to
display a certain text; such as when, for example, we indicate that
certain text should be in bold or italics. Conceptual markup, on the
other hand, indicates what function complies with certain text in the
document as a whole, such as when we indicate that something is a title,
or a subtitle, or a quote. In general, documents that prefer to use this
second kind of markup are more consistent and easier to compose, since
once again they point out the difference between authorship and
composition: the author indicates that such and such a line is a title,
or that such and such a fragment is a warning, or a quote; and the
typesetter decides how to typographically highlight all titles, warnings
or quotations; thus, on the one hand, consistency is guaranteed, as all
the fragments that fulfil the same function will look the same, and, on
the other hand it saves time, because the format of each type of
fragment only needs to be indicated once.

\stopsection

\startsection [title=\TeX\ and its derivatives]

\TeX\ was developed towards the end of the 70s by {\sc Donald E. Knuth},
a professor (now emeritus professor) of theoretical computer programming
at Stanford University, who implemented the program to produce his own
publications and as an example of a systematically developed and
annotated program. Along with \TeX, {\sc Knuth} developed an additional
programming language called \MetaFont, created for designing
typographical fonts, and he used it to design a font he baptised as {\em
Computer Modern}, which, along with the usual characters of any font,
also included a complete set of \quotation{glyphs}\footnote{In
typography, a glyph is the graphical representation of a character, a
number of characters or part of a character, and is today's equivalent
of the letter type (the bit engraved with the letter or movable type).}
designed for writing mathematics. To all this he added some additional
utilities and thus the typesetting system called \TeX\ was born, which,
due to its power, quality of results, flexibility of use and broad
possibilities, is considered one of the best computerised systems for
text composition. It was designed for texts in which there was a lot of
mathematics, but it soon became clear that the system's possibilities
made it suitable for all kinds of texts.

\reference[ref:boxes]{}Internally, \TeX\ functions in the same way as
the former compositors would do in a print shop. For \TeX, everything is
a {\em box}: The letters are contained in boxes, the blank spaces are
also boxes, several letters (the boxes containing several letters) form
a new box that encloses the word, and several words, along with the
blank space between them, form a box containing a line, several lines
become a box containing the paragraph ... and so on. All this, moreover,
with extraordinary precision in the handling of measurements. Consider
that the smallest unit that \TeX\ deals with is 65.536 times smaller
than the typographical point with which characters and lines are
measured, which is usually the smallest unit handled by most word
processing programs. This means that the smallest unit handled by \TeX\
is approximately 0.000005356 millimetres.

% I copy-pasted the accented epsilon from "Aprender ConTeXt",
% by Pablo Rodríguez, however for whatever reason it would not process. 
% Therefore I have used \definecharacter to create an accented epsilon

\definecharacter etilde {\buildtextaccent ´ {\lower.2ex\hbox{\epsilon}}}

The name \TeX\ comes from the root of the Greek word
\tau\etilde\chi\nu\eta, written in upper case letters ({\tfx ΤÉΧΝΗ}).
Therefore, the final letter of the word \TeX\ is not a Latin \quote{X},
but  the Greek \quote{\chi}, pronounced -- apparently -- like the
Scottish \quote{ch} in {\em loch}. So \TeX\ should be pronounced as {\em
Tech}. This Greek word, on the other hand, meant both \quotation{art}
and \quotation{technology}, and this is the reason why {\sc Knuth} chose
it to name his system. The purpose of this name -- he wrote --
\quotation{is to remind you that \TeX\ is primarily concerned with high
quality technical manuscripts. Its emphasis is on art and technology, as
in the underlying Greek word}.

Using the convention established by {\sc Knuth}, \TeX\ is to be written:

\startitemize

  \item In typographically formatted texts like this one, using the logo
  that I have been using until now: the three letters in upper case,
  with the central \quote{E} slightly displaced below to facilitate a
  closer alignment between the \quote{T} and the \quote{X}; or in other
  words: \quotation{\TeX}.

  \startSmallPrint

    To facilitate the writing of this logo, {\sc Knuth} included an
    instruction in \TeX\ for writing it in the final document:
    \PlaceMacro{TeX}\tex{TeX}.

  \stopSmallPrint

  \item In unformatted texts (such as an email, or a text file), with
  the \quote{T} and the \quote{X} in upper case, and the central
  \quote{e} in lower case; so: \quotation{TeX}.

\stopitemize

This convention continues to be used in all derivatives of \TeX\ that
include its proper name, as is the case with \ConTeXt. When writing it
in text mode we need to write \quotation{ConTeXt}.

\startsubsection [reference=sec:engines,title=\TeX\ engines]

The \TeX\ program is free {\em libre} software: its source code is
available to the public and anyone can use it or modify it as they wish,
with the only condition that, if modifications are made, the result
cannot be called \quotation{\TeX}. This is why, over time, certain
adaptations of the program have emerged, introducing different
improvements to it, and which are generally referred to as {\em \TeX\
engines}. Apart from the original \TeX\ program, the main engines are,
in chronological order of appearance, \pdfTeX, \eTeX, \XeTeX\ and
\LuaTeX. Each of them is supposed to incorporate the improvements of the
previous one. These improvements, on the other hand, up until the
appearance of \LuaTeX, did not affect the language itself, but only the
input files, the output files, handling of sources and low level
operation of macros.

\startSmallPrint

  The question of which \TeX\ engine to use is a much debated one within
  the \TeX\ universe. I will not develop this question here since
  \ConTeXt\ Mark~IV only works with LuaTeX. In reality, in the \ConTeXt\
  world, discussion on \TeX\ {\em engines} becomes a discussion on
  whether to use Mark~II (that works with PdfTeX and XeTeX) or Mark~IV
  (that only works with LuaTeX).

\stopSmallPrint

\stopsubsection

\startsubsection [title=Formats derived from \TeX]

The core or heart of \TeX\ only understands a set of approximately 300
very basic instructions, called {\em  primitives}, which are suitable
for typesetting operations and programming functions. The great majority
of these instructions are of a very {\em low level}, which, in computer
terminology, means that they are more easily understandable by the
computer than by human beings, since they concern very elementary
operations of the \quotation{shift this character 0.000725 millimetres
upward} kind. Hence {\sc Knuth} saw that \TeX\ would be extensible,
meaning that there should be a mechanism that allows instructions to be
defined at a higher level, more easily understandable by human beings.
These instructions, that are broken down into other simpler instructions
at the time of execution, are called {\em macros}. For example, the
\TeX\ instruction that prints the (\tex{TeX}) logo, is broken down as
follows at the time of execution:

\vbox{\starttyping 
T 
\kern -.1667em 
\lower .5ex 
\hbox {E} 
\kern -.125em 
X
stoptyping}

But for the human being, it is much easier to understand and remember
that the simple command \quotation{\PlaceMacro{TeX}\type{\TeX}} carries
out the typographical operations needed to print the logo.

\startSmallPrint

  The difference between what is a {\em macro} and what is a {\em
  primitive}, really only has importance from the perspective of the
  \TeX\ developer. From the user's perspective they are {\em
  instructions} or, if you prefer, {\em commands}. {\sc Knuth} called
  them {\em control sequences}.

\stopSmallPrint

This possibility of extending the language through {\em macros} is one
of the characteristics that turned \TeX\ into such a powerful tool. In
fact, {\sc Knuth} himself created approximately 600 macros that, along
with the 300 primitives, make up the format called \quotation{Plain
\TeX}. It is quite common to confuse \TeX\ properly so called, with
Plain \TeX\ and, in fact, almost everything usually written or said
about \TeX, is really a reference to Plain \TeX. Books that claim to be
about \TeX\ (including the foundational \quotation{{\em The \TeX
Book}}), really refer to Plain \TeX; and those who believe they are
directly working with \TeX\ are in reality working with Plain \TeX.

Plain \TeX\ is what, in \TeX\ terminology, is called a {\em format},
consisting of a broad set of macros, together with certain rules of
syntax concerning how and in what way to use them. As well as Plain
\TeX, with the passing of time other {\em formats} have been developed,
among which it is worth mentioning \LaTeX, a broad set of macros for
\TeX\ created in 1985 by {\sc Leslie Lamport} and which is probably the
\TeX\ derivative that is most in use in the academic, technological and
mathematical world. \ConTeXt\ is (or has begun to be), on a par with
\LaTeX\ as a format derived from \TeX.

Normally these {\em formats} are accompanied by a programme that loads
the macros that make them up into memory before calling on \MyKey{tex}
(or the actual engine being used for processing) to process the source
file. But even though all these formats are actually running \TeX, as
each of them has different instructions and different syntax rules from
the user's point of view, we can think of them as {\em different
languages}. They all draw their inspiration from \TeX, but are different
from \TeX\ and also different from each other.

\stopsection

\startsection [title=\ConTeXt, reference=sec:ctx]

In reality \ConTeXt, which started out as a {\em format} of \TeX, is
much more than that today. \ConTeXt\ includes:

\startitemize[n]

  \item A very broad set of \TeX\ macros. If Plain \TeX\ has around 900
  instructions, \ConTeXt\ has around 3500; and if we add up the names of
  the different options that these commands support, we are talking
  about a vocabulary of around 4000 words. The vocabulary is this large
  because of the \ConTeXt\ strategy to facilitate its learning, and this
  strategy means the inclusion of any number of synonyms for commands
  and options.

  \startSmallPrint

    The intention is that if a certain effect is to be achieved, then
    for each of the ways an English speaker would call that effect there
    is a command or option that achieves it -- which is supposed to make
    the use of the language easier. For example, to simultaneously get a
    bold and italic letter, \ConTeXt\ has three instructions all of
    which achieve the same result: \type{\bi}, \type{\italicbold} and
    \type{\bolditalic}.

  \stopSmallPrint

  \item A likewise broad set of macros for \MetaPost, a graphical
  programming language derived from \MetaFont, which in turn is a
  language for typeface design that {\sc Knuth} developed jointly with
  \TeX.

  \item Various {\em scripts} developed in {\sc Perl} (the oldest), {\sc
  Ruby} (some also old, others not so old) and {\sc Lua} (the most
  recent).

  \item An interface that integrates \TeX, \MetaPost, {\sc Lua} and XML,
  allowing one to write and process documents in any of these languages,
  or to mix elements from some of them.

\stopitemize

\startSmallPrint

  Perhaps you did not understand much of the previous explanation? Don't
  worry about it. I used a lot of computer jargon in it and mentioned
  many programs and languages. It is not necessary to know all the
  different components to use \ConTeXt. The important thing, at this
  stage of learning, is to stay with the idea that \ConTeXt\ integrates
  many tools from different sources that together make up a {\em
  typesetting system}.

\stopSmallPrint

It is because of this latter feature of integration of tools with
different origins, that we say that \ConTeXt\ is a \quotation{hybrid
technology} intended for typesetting documents. My understanding is that
this turns \ConTeXt\ into an extraordinarily advanced and powerful
system.

Even though \ConTeXt\ is much more than a collection of macros for \TeX,
it continues to be based on \TeX, and this is why this document, that I
claim to be no more than an {\em introduction}, focuses on this.

\ConTeXt, on the other hand, is rather more modern than \TeX. When \TeX\
was created, the emergence of computers was just at the beginning, and
we were far from seeing what the internet and the multimedia world would
be (would become). In this respect, \ConTeXt\ naturally integrates some
of the things that have always been something of a foreign body in \TeX\
such as including external graphics, handling colour, hyperlinks in
electronic documents, assuming a paper size suitable for a document
intended for display on a screen, etc.

\stopsubsection

\startsubsection 
  [ 
    reference=sec:historyctx, 
    title=A short history of \ConTeXt
  ]

\ConTeXt{} was born approximately in 1991. It was created by {\sc Hans
Hagen} and {\sc Ton Otten} in a Dutch document design and processing
company called \quotation{{\em Pragma Advanced Document Engineering}},
usually abbreviated as Pragma ADE. It began by being a collection of
\TeX\ macros that had Dutch names and was unofficially known as {\em
Pragmatex}, aimed at the company's non-technical employees who had to
manage the many details of editing typeset documents and who were not
used to using markup languages or interfaces other than Dutch. Hence the
first version of \ConTeXt{} was written in Dutch. The idea was to create
a sufficient number of macros with a uniform and consistent interface.
Approximately in 1994 the {\em package} was stable enough for a user
manual to be written in Dutch, and in 1996, through the initiative of
{\sc Hans Hagen}, reference to it began taking on the name
\quotation{\ConTeXt{}}. This name claims to mean \quotation{Text with
\TeX} (using the Latin preposition \quote{con} meaning \quote{with}),
but at the same time a wordplay on the English (and Dutch) word
\quotation{Context}. Behind the name, therefore, lies a triple play on
words involving \quotation{\TeX}, \quotation{text} and
\quotation{context}.

\startSmallPrint

  Therefore, since the name is based on wordplay, \ConTeXt\ should be
  pronounced \quote{context} and not \quote{contecht} since this would
  mean losing the play on words.

\stopSmallPrint

The interface began to be translated into English approximately in 2005,
giving rise to the version known as \ConTeXt\ Mark~II, where the
\quote{II} is explained because in the mind of the developers, the
previous version in Dutch was Version~\quote{I}, even though it was
never officially called that. After the interface was translated into
English, the use of the system began to spread beyond the Netherlands,
and the interface was translated into other European languages such as
French, German, Italian and Romanian. The \quotation{official}
documentation for \ConTeXt{}, nevertheless, is normally based on the
English version, and this is the version this document works with.

In its initial version, \ConTeXt\ Mark~II worked with the PdfTeX {\em
\TeX\ engine}. But later, at the appearance of the \XeTeX\ {\em engine},
\ConTeXt\ Mark~II was modified to allow the use of this new engine that
contributed a number of advantages by comparison with PdfTeX. But when
\LuaTeX\ came along some years later, the decision was made to
internally reconfigure how \ConTeXt{} functioned in order to integrate
all the new possibilities that this new engine offered. And so,
\ConTeXt\ Mark~IV was born, and it was presented in 2007, immediately
after the presentation of \LuaTeX. Very probably, one of the influencing
factors in the decision to reconfigure \ConTeXt\ to adapt it to \LuaTeX\
was that two of the three main developers of \ConTeXt{}, {\sc Hans
Hagen} and {\sc Taco Hoekwater}, were also part of the main team
developing \LuaTeX. This is why \ConTeXt\ Mark~IV and \LuaTeX\ were born
at the same time and developed in unison. There is a synergy between
\ConTeXt{} and \LuaTeX\ that does not exist in any other derivative of
\TeX; and I doubt that any of the others can avail themselves of the
advantages of \LuaTeX\ as \ConTeXt{} can.

There are many differences between Mark~II and Mark~IV, although most of
them are {\em internal}, that is, they have to do with how the macro
actually works at a lower level, such that from the user's perspective
the differences are not noticeable: the name and parameters of the macro
remain the same. There are, however, some differences that affect the
interface and force one to do things differently depending on which
version one is working with. These differences are relatively few, but
they do affect very important aspects such as for example, the coding of
the input file, or the handling of fonts installed in the system.

\startSmallPrint

  It would, however, be very welcome if somewhere there were a document
  that explained (or listed) the appreciable differences between Mark~II
  and \Conjecture Mark~IV. In the \ConTeXt\ wiki, for example, for each
  \ConTeXt\ command there are {\em two kinds of syntax} (very often
  identical). I presume one belongs to Mark~II and the other to Mark~IV;
  and based on this assumption, I also presume that the {\em first
  version} is from Mark~II. But the truth is that the wiki tells us
  nothing about this.

\stopSmallPrint

The fact that the differences, at a language level, are relatively few,
means that on many occasions rather than speaking of two versions we are
talking about two \quotation{flavours} of \ConTeXt{}. But whether you
call them one or the other, the fact is that a document prepared for
Mark~II cannot normally be compiled with Mark~IV and vice versa; and if
the document mixes both versions, it will most likely not compile well
with either of them; which implies that the author of the source file
has to start by deciding whether to write for Mark~II or for Mark~IV.

\startSmallPrint

  If we work with the different versions of \ConTeXt{}, a good trick for
  differentiating at first sight between files intended for Mark~II and
  those intended for Mark~IV is to use a different extension for the
  file names. Thus, for example, for any files I have written for
  Mark~II, I put \MyKey{.mkii} as the extension, and \MyKey{.mkiv}
  instead for those written for Mark~IV. It is true that \ConTeXt{}
  expects all source files to have the extension \MyKey{.tex}, but we
  can change the file extension as long as we expressly indicate the
  file extension when applying \ConTeXt{} to the file.

\stopSmallPrint

The \ConTeXt{} distribution installed on the wiki, \suite-, includes
both versions, and to avoid confusion -- I assume -- uses a different
command for each of them to compile a file. Mark~II compiles with the
command \MyKey{texexec} and Mark~IV with the command \MyKey{context}.

\startSmallPrint

  In fact both commands, \MyKey{context} and \MyKey{texexec}, are {\em
  scripts} with different options that run \MyKey{mtxrun}, which in turn
  is a {\sc Lua} {\em script}.

\stopSmallPrint

Today, Mark~II is frozen and Mark~IV continues to be developed, which
means that new versions of the former are only published when errors or
faults are discovered that need to be corrected, while new versions of
Mark~IV continue to be published regularly; sometimes two or three times
a month, even though in most of these cases the \quotation{new versions}
do not introduce perceptible changes in the language but are limited to
somehow improving implementation of a command at low level, or updating
some of the many manuals included with the distribution.  Even so, if we
have installed the development version -- which is what I would
recommend and which is the one installed by default with \suite-{} -- it
makes sense to update our version from time to time (See
\in{Appendix}[installation_suite] for the way of updating the installed
version of \suite-).

\startSmallPrint

\startsubsubsubsubject 
  [title=LMTX and other alternative implementations of Mark~IV]

The developers of \ConTeXt{} are naturally restless, and therefore have
not ceased development of \ConTeXt{} with Mark~IV; new versions are
still being tested and experimented with, although in general these
differ from Mark~IV in very few ways, and do not have the
incompatibility in compiling that exists between Mark~IV and Mark~II.

Thus, certain minor variants of Mark~IV called, respectively, Mark~VI,
Mark~IX and Mark~XI have been developed. Of these, I have only been able
to find a small reference to Mark~VI in the \ConTeXt{} wiki where it
says that the only difference with Mark~IV lies in the possibility of
defining commands by assigning the parameters not a number, as is
traditional in \TeX, but a name, as is usually done in almost all
programming languages.

More important than these small variations, I believe, is the appearance
in the \ConTeXt{} universe (\ConTeXt{}verse?) of a new version called
LMTX, a name which is an acronym of LuaMetaTeX: a new \TeX\ {\em engine}
that is a simplified version of \LuaTeX, developed with a view to saving
computer resources; which means that LMTX requires less memory and less
processing power than \ConTeXt\ Mark~IV.

LMTX was presented in spring 2019 and one assumes that it will not imply
any external change to the Mark~IV language. For the author of the
document there would be no difference at the time of working with it;
but when compiling it, one would need to choose between doing so with
\LuaTeX, or doing so with LuaMetaTeX. In
\in{Appendix}[installation_suite], relating to the installation of
\ConTeXt, a procedure is shown for assigning a different command name to
each of the installations (\in{section}[sec:alias]).

\stopsubsubsubject

\stopSmallPrint

\stopsubsection

\startsubsection [title=\ConTeXt\ versus \LaTeX]

Given that the most popular format derived from \TeX{} is \LaTeX{}, a
comparison between this and \ConTeXt\ is inevitable. Clearly we are
talking about different languages although in some way related to each
other since they both derive from \TeX; the relationship is similar to
that which exists, for example, between Spanish and French: languages
that have a common origin (Latin) which means that their syntax is {\em
similar} and many of the words in each of these languages is mirrored by
a word in the other. But apart from this {\em family resemblance},
\LaTeX\ and \ConTeXt\ differ in their philosophy and implementation,
since the initial aims of both, are, to some degree, the opposite.
\LaTeX\ claims to facilitate the use of \TeX, isolating the author from
the concrete typographical details to help focus on content, leaving the
typesetting details in the hands of \LaTeX. This means that simplifying
the use of \TeX\ takes place at the expense of limiting the immense
flexibility of \TeX, by predefining basic formats and limiting the
number of typographical issues that the author has to decide on. In
contrast to this philosophy, \ConTeXt\ was born within a company
dedicated to typesetting documents. Therefore, far from wanting to
isolate the author from typesetting details, the aim is to give the
author absolute and complete control over them. To achieve this,
\ConTeXt\ provides a uniform and consistent interface which is much
closer to the original spirit of \TeX\ than \LaTeX.

This difference in philosophy and founding objectives then translates,
in turn, into a difference in implementation. \LaTeX, that tends to
simplify things as much as possible, does not need to use all of \TeX's
resources. In some way, its core is rather simple. So when there is a
need to broaden its possibilities, it is necessary to expressly write a
{\em package} to do so. This {\em packaging} associated with \LaTeX\ is
both a virtue and a defect: a virtue, because the tremendous popularity
of \LaTeX, together with the generosity of its users, means that almost
any need we are likely to have has been met by someone before, and that
there is a package to achieve it; but it is also a defect because these
packages are often incompatible with each other, and their syntax is not
always uniform. This means that working with \LaTeX\ requires one to
constantly be searching through thousands of already existing packages
to fulfil one's needs and ensure that they all work together.

By contrast with the simplicity of the \LaTeX\ core, which is
complemented by its extensibility through packages, \ConTeXt\ is
designed to have within it all -- or almost all -- the typographical
possibilities of \TeX, so its conception is much more monolithic, but at
the same time it is also more modular. The \ConTeXt\ core allows us to
do almost everything, and we are guaranteed that there will be no
incompatibilities between its different utilities, no need to
investigate extensions for what we need, and the syntax of the language
does not change just because we need a particular utility.

It is true that \ConTeXt\ has what are called extension {\em modules}
that some might consider as carrying out a function similar to the
\LaTeX\ packages, but in real terms they both work differently:
\ConTeXt\ modules are designed exclusively to include additional
utilities that, because they are still in an experimental stage, have
not yet been incorporated into the core, or to allow access to
extensions authored by someone outside the \ConTeXt\ development team.

I do not believe that either one of these two {\em philosophies} is
preferable to the other. The question depends rather on the user's
profile and what he or she wants. If the user does not want to deal with
typographical issues but simply produce very high quality standardised
documents, it would probably be preferable to opt for a system like
\LaTeX; on the other hand, the user who likes to experiment, or who
needs to control every last detail of the document, or someone who has
to devise a special layout for a document, would probably be better off
using a system like \ConTeXt, where the author has all the control in
their hands; with the risk, of course, of not knowing how to use this
control correctly.

\stopsubsection

\startsubsection 
  [title=A good understanding of the dynamics of working with \ConTeXt]

When we work with \ConTeXt, we always begin by writing a text file
(which we call a {\em source file}), in which, along with the actual
content of our final document, we will include the instructions (in
\ConTeXt-speak) that indicate exactly how we want the document to be
formatted: the general appearance we want its pages and paragraphs to
have, the margins we want to apply to certain paragraphs, the font we
want to display, the snippets we want shown in a different font, etc.
Once we have written the source file, we apply the \MyKey{context}
program from a terminal, which will process it, and will generate a
different file from it in which the contents of our document will be
formatted in accordance with the instructions included in the source
file for this purpose. This new file could be sent to a (commercial)
printer, displayed on screen, placed on the internet or distributed
among contacts, friends, clients, teachers, pupils ... or in other
words, to anyone for whom we wrote the document.

This means that when working with \ConTeXt\ the author is working with a
file whose appearance has nothing to do with the final document: the
file the author is directly working on is a text file that is not
formatted typographically. So \ConTeXt\ works in a different way than do
programs known as {\em word processors} that show the final appearance
of the edited document at the same time we are writing it. For those
accustomed to word processors, the way of working with \ConTeXt\ will
initially feel strange, and it may even take some time to get used to
it. However, once one gets used to it, one understands that in reality
this other way of working, differentiating between the work file and the
final result, is actually an advantage for many reasons, among which I
will highlight here, without following any particular order, the
following:

\startitemize[n,broad]

  \item Because text files are \quote{lighter} to handle than word
  processor binary files, and editing them requires less computer
  memory, they are less likely to be corrupted, and they do not become
  unintelligible when we change the version of the program we are
  creating them with. They are also compatible with any operating
  system, and can be edited with many text editors, so that in order to
  work with them it is not necessary for the computer system to have the
  program the file was created with installed on it: any other editing
  program will do; and in every computer system there is always some
  text editing program.

  \item Because differentiating between the working document and the
  final document helps to distinguish what the actual content of the
  document is from what its appearance will be, allowing the author to
  concentrate on the content in the creation phase, and to focus on the
  appearance in the typesetting phase.

  \item Because it allows one to quickly and accurately change the
  appearance of the document, since this is determined by \ConTeXt\
  commands that can be easily identified.

  \item Because this facility for changing the appearance, on the other
  hand, allows us to easily generate two (or more) different versions
  from a single content: Perhaps one version optimised for printing on
  paper, and another designed to be displayed on screen, adjusted to the
  size of the latter and perhaps including hyperlinks that make no sense
  in a paper document.

  \item Because typographical errors (typos) that are common in word
  processors, such as extending the italics beyond the last character of
  a word, are also easily avoided.

  \item Because while the work file is not distributed and is \quote{for
  our eyes only}, it is possible to incorporate annotations and
  observations, comments and warnings for ourselves for subsequent
  revisions or versions, with the peace of mind in knowing that these
  will not appear in the formatted file to be distributed.

  \item Because the quality that can be obtained by processing the whole
  document simultaneously is much higher than that which can be achieved
  with a program that has to make typographical decisions as the
  document is being written.

  \item Etcetera.

\stopitemize

All of the above means that on the one hand when working with \ConTeXt,
once we have got the hang of it, we are more efficient and productive,
and that on the other hand, the typographical quality we can obtain is
much superior to what can be obtained with so-called {\em word
processors}. And although it is true that the latter are easier to use,
in point of fact they are not that {\em much} easier to use. Because
while it is true that \ConTeXt{}, as we have said before, contains 3500
instructions, a normal user of \ConTeXt{} will not need to know them
all. To do what is usually done with word processors, we only need to
know the instructions that allow us to indicate the structure of the
document, a few instructions concerning common typographical resources,
such as bold or italics, and perhaps how to generate a list, or a
footnote. In total, no more than 15 or 20 instructions will allow us to
do almost all the things that are done with a word processor. The rest
of the instructions allow us to do different things that we normally
cannot do with a word processor, or are very difficult to achieve. We
can say that while learning to use \ConTeXt{} is more difficult than
learning to use a word processor, this is because we can do a lot more
with \ConTeXt{}.

\stopsubsection

\startsubsection 
  [title=Getting help with \ConTeXt]

\adaptlayout[line=+2]

While we are new to it, the best place for getting help with \ConTeXt\ is, undoubtedly, on the \goto{wiki}[url(wiki)], which abounds in examples and has a good search engine, especially if one understands English well. We can also find help on the internet, of course, but here the play on words in the name \ConTeXt\ will play tricks on us because searching on the word \quotation{context} will return millions of results most of which will have nothing to do with what we are looking for. To find information on \ConTeXt\ you need to add something to the word \quotation{context}; for example, \quotation{tex}, or \quotation{Mark IV} or \quotation{Hans Hagen} (one of the creators of \ConTeXt) or \quotation{Pragma ADE}, or something similar. It could also be useful to seek information using the wiki name: \quotation{contextgarden}.

When we have learned something more about \ConTeXt, we can consult some
of the many documents included in \suite-, or even seek help in
\goto{TeX -- LaTeX Stack Exchange}
[url(https://tex.stackexchange.com/)], or on the mailing list for
\ConTeXt\
(\goto{NTG-context}[url(https://mailman.ntg.nl/mailman/listinfo/ntg-context)]).
The latter involves the people who know the most about \ConTeXt, but the
rules of good cyber-etiquette demand that before asking a question, we
should have tried hard beforehand to find the answer ourselves.

\stopsubsection

\stopsubsection

\stopchapter

\stopcomponent

%%% Local Variables:
%%% mode: ConTeXt
%%% mode: auto-fill
%%% TeX-master: "../introCTX.mkiv"
%%% coding: utf-8-unix
%%% End:
%%% vim: tw=72:
