% -*- coding: utf-8 -*-
%%% File:        b03_Commands.mkiv
%%% Author:      Joaquín Ataz-López
%%% Begun:       April 2020
%%% Concluded:   April 2020

% Contents:    This is the central chapter for understanding the essence
%              of ConTeXt. Some of its contents are based on TeX.  The
%              use of reserved characters and the distinction between
%              control symbols and control words. Knuth insists on the
%              difference, but neither the ConTeXt documentation, nor in
%              general, LaTeX documentation do so (despite a brief
%              reference in Kopka). Yet I believe the distinction is
%              important for understanding why, in TeX the command names
%              cannot mix letters and non-letters.  As for the
%              explanation of commands, there are decisions on
%              «start-stop» constructions that I still hesitate to call
%              «environments». But at times they also do so in the wiki.
%              I am satisfied, on the other hand, with my explanation of
%              \setup + \define using the \framed command as an example.
%              Amon other additional concepts I would have also liked to
%              introduce counters, but I have not found any documentation
%              on the name of the main counters in ConTeXt. For example I
%              have done various tests manipulating chapter counters by
%              hand and.... zilch! Rien de rien. Nothing. This
%              information, central for controlling ConTeXt, should be
%              made explicit somewhere           lugar.

%%% Edited with: Emacs + AuTeX - And at times with vim + context-plugin
%%%

\environment ../introCTX_env.mkiv

\startcomponent b03_Commands.mkiv

\startchapter
  [
    reference=cap:commands,
    title=Příkazy a další základní koncepty\\ \ConTeXt{}u,
    bookmark=Commands and other fundamental concepts of ConTeXt,
  ]

\TocChap

Již jsme si ukázali, že ve zdrojovém souboru, stejně jako v textu,
obsaženém v našeho budoucím formátovaném dokumentu, najdeme instrukce
potřebné k tomu, abychom \ConTeXt{u} sdělili jak chceme, aby náš
rukopis vypadal po zpracování. Tyto instrukce můžeme nazývat
\quotation{příkazy}, \quotation{makra} nebo \quotation{řídící sekvence}.

\startSmallPrint

  Z hlediska vnitřního fungování \ConTeXt{}u (resp. fungování \TeX{}u)
  je rozdíl mezi {\em primitivy} a {\em makry}. Primitiva jsou
  jednoduché instrukce, které nemohou být rozděleny na další jednodušší
  instrukce. Makra jsou instrukce, které lze rozdělit na další jednodušší
  instrukce, které lze případně rozložit na další instrukce, a tak dále
  a tak dále. Většina instrukcí \ConTeXt{}u jsou ve skutečnosti makra.
  Z pohledu programátora je mezi makry a primitivy důležitý rozdíl.
  Z pohledu uživatele však tento rozdíl tak důležitý není: v obou
  případech máme k dispozici instrukce, které budou provedeny, aniž
  bychom se museli starat o to, jak fungují na vnitřní úrovni. Proto se
  v dokumentaci \ConTeXt{}u běžně hovoří o {\em příkazu}, v případě kdy
  jde o sdělení z pohledu uživatele a o {\em makru} v případě, když jde
  o sdělení pro programátora. Jelikož v tomto návodu zohledňujeme
  primárně uživatelský pohled, budu používat oba termíny jako synonyma.

  Pomocí {\em příkazů} rozkazujeme \ConTeXt{}u něco udělat; {\em
  řídíme} tak, s jejich pomocí, běh programu. Proto když \cap{Knuth}, otec
  \TeX{}u, hovoří o primitivech nebo makrech, používá termín {\em řídící
  sekvence}. Domnívám se, že to je nejpřesnější termín ze všech
  zmíněných. Budu ho používat v případech, kdy bude třeba rozlišit mezi
  {\em řídícími znaky} a {\em řídícími slovy}.

\stopSmallPrint

Instrukce \ConTeXt{}u jsou dvou základních typů: vyhrazené znaky
a příkazy, které se tak ostatně jmenují.

\startsection
  [
    reference=sec:reserved characters,
    title=Vyhrazené znaky \ConTeXt{}u,
  ]

Když \ConTeXt\ čte zdrojový soubor, který je složený pouze z textových znaků (protože se jedná o
čistě textový soubor), musí nějak rozlišit, co je skutečný text, který má být formátován,
a co jsou instrukce, které má provést. Vyhrazené znaky \ConTeXt{}u
umožňují toto rozlišení. \ConTeXt\ v zásadě předpokládá, že že každý znak ve zdrojovém
souboru je text, který má být zpracován, pokud se nejedná o jeden z 11 znaků rezervovaných
znaků, které se považují za instrukce.
Pouze 11 instrukcí? Ne. Existuje pouze 11 vyhrazených znaků; ale protože jeden z nich
\quotation{{\tt\backslash}}, mění bezprostředně následující znak, nebo znaky, na instrukci,
je skutečně možný počet příkazů je neomezený. \ConTeXt má přibližně 3000 příkazů (sečteme-li
příkazy pouze pro Mark II, pro Mark IV a příkazy společné pro obě verze).

Vyhrazené znaky jsou následující:

{
  \switchtobodyfont[25pt]
  \midaligned{\cmd{ \% \{ \} \# \lettertilde\ \| \$ \_ \letterhat\ \&}}
}

\ConTeXt\ je interpetuje následujícím způsobem:

\semitable{\backslash}

Tento znak je pro nás nejdůležitější ze všech: naznačuje, že to, co
následuje bezprostředně za ním, nesmí být interpretováno jako text,
ale jako instrukce. Říká se mu \quotation{Escape znak} nebo \quotation{Escape
sekvence} (i když nemá nic společného s klávesou \quotation{Esc}, kterou najdete na většině
klávesnic).\footnote{V počítačové terminologii se klíč, který ovlivňuje interpretaci
následujícího znaku, nazývá {\em escape znak}. Naproti tomu klávesa escape na klávesnicích
se tak nazývá proto, že generuje tzv. znak 27 v kódu ASCII, který se v tomto
kódování používá jako znak escape. Dnes se používá klávesy Escape spojeno spíše
s myšlenkou zrušení probíhající akce.}

\semitable{\%}

Říká \ConTeXt{}u, že to, co následuje až do konce řádku, je komentář.
který nesmí být zpracován ani zahrnut do konečného formátovaného souboru.
Zavedení komentářů do zdrojového souboru je velmi užitečné. Komentář
může pomoci vysvětlit, proč bylo něco provedeno určitým způsobem, a to
je velmi užitečné v dokončených zdrojových souborech s ohledem na pozdější revize.
kdy si někdy nemůžeme vzpomenout, proč jsme udělali to, co jsme udělali; nebo to také může
pomoci jako připomínka něčeho, co bychom mohli potřebovat revidovat.
Může být dokonce použita jako pomůcka k nalezení příčiny určité chyby ve
zdrojovém souboru, protože umístěním značky komentáře na začátek řádku,
vyloučíme tento řádek z kompilace a můžeme zjistit, zda to byl právě on.
řádek byl příčinou chyby; lze jej také použít k uložení dvou
různých verzí téhož makra, a tak získat různé výsledky.
nebo zabránit zkompilování úryvku, u něhož si nejsme jisti, že ho chceme
smazat, protože je možné, že jej budeme chtít do souboru vrátit... atd.
Jakmile jsme otevřeli možnost, že náš zdrojový soubor obsahuje text,
který neuvidí nikdo jiný než my sami, je naše použití tohoto znaku omezeno
pouze naší vlastní představivostí. Přiznám se, že toto je jedwn z nástrojů,
které mi nejvíce chybí, když jediným prostředkem pro psaní textu je textový procesor.

\semitable{\{}

Tento znak zahajuje skupinu. Skupiny jsou bloky textu ovlivněné
určitými funkcemi. Budeme o nich hovořit v \in{section}[sec:groups].

\semitable{\}}

Tento znak ukončuje skupinu dříve otevřenou pomocí {\tt \{}.

\semitable{\#}

Tento znak se používá pro definování maker. Odkazuje na argumenty makroa.
Viz \in{section}[sec:define] dálev kapitole.

\semitable{\lettertilde}

Vloží do dokumentu nezlomitenou mezeru, což znamená, že dvě slova oddělená
znakem \type{~} zůstanou na stejném řádku. O této instrukci a o tom, kde
by měla být použita více v \in{section}[sec:lettertilde].

\semitable{\|}

Tento znak se používá ve složených slovech s oddělovačem. Tato slova mohou být
rozdělena na slabiky pouze v první části složeného slova. Viz
\in{section}[sec:compound words].

\semitable{\$}

Tento znak je přepínačem matematického režimu. Nastavuje tento režim, pokud
nebyl nastevn, a ukončuje jej, pokud nastaven byl. V matematickém režimu
\ConTeXt\ používá některá písma a pravidla, která se liší od běžných,
s cílem optimalizovat zápis matematických vzorců. Přestože
psaní matematiky je velmi důležitým použitím \ConTeXt{}u, nebudu se mu
v tomto úvodu věnovat. Abych byl přesný: necítím se na to.

\semitable{\_}

Tento znak se používá v matematickém módu a indikuje, že následující znaky
jsou spodním indexem. Například, pro zápis $x_1$, použijeme \type{$x_1$}.

\semitable{\letterhat}

Tento znak se používá v matematickém módu a indikuje, že následující znaky
jsou spodním indexem. Například, pro zápis $(x+i)^{n^3}$ použijeme
\type{$(x+i)^{n^3}$}.

\semitable{\&}

V dokumentaci \ConTeXt{}u se píše, že se jedná o vyhrazené znak,
ale neuvádí se proč. V obyčejném \TeX{}u má tento znak
v podstatě dvě použití: používá se k zarovnání sloupců v prostředí
základních tabulek a v matematickém kontextu tak, aby to, co následuje, bylo
považováno za normální text. Je uveden i v úvodní příručce \quotation{\ConTeXt\
Mark~IV, an Excursion}, ačkoli se neuvádí, k čemu slouží. A jsou uvedeny
příklady jeho použití v matematických vzorcích, i když ne toho druhu, jakým je
v používán vběžném \TeX{}u, ale k zarovnání sloupců v rámci \Doubt komplexních
funkcí. Jelikož jsem spíš spisovatel, nemám pocit, že bych mohl provádět
další testy, abych zjistil, jaké je přesné použití tohoto vyhrazeného znaku a
k čemu slouží.

Lze předpokládat, že při výběru vyhrazených znaků byly zvoleny ty,
které jsou dostupné na většině klávesnic, ale současně nejsou často používány
v běžném textu.
Nicméně se může stát, byť ne příliš často, že některý z nich budeme potřebovat
v textu použít. Například když budeme chtít napsat, že něco stojí
100 dolarů (\$100), nebo že ve Španělsku procento řidičů starších 65 let v roce
2018 činilo 16\,\%. V těchto případech nemůžeme psát vyhrazený znak přímo,
ale použít {\em příkaz}, který vypíše ve výsledném dokumentu vyhrazený znak samotný.
Příkaz pro každý z vyhrazených znaků najdete \in{table}[Reserved characters].

\placetable
  [here]
  [Vyhrazené znaky]
  {\tfx Psaní vyhrazených znaků}
{\starttabulate[|c|l|]
  \HL
  \NC {\bf Vyhrazený znak} \NC {\bf Příkaz, který jej vygeneruje}\NR
  \HL
  \NC{\tt
    \backslash}\NC\PlaceMacro{backslash}\tex{backslash}\NR\macro{reserved characters+\backslash backslash}
  \NC{\tt \%}\NC{\cmd{\%}}\NR\macro{reserved characters+\backslash \%}
  \NC{\tt \{}\NC\cmd{\{}\NR\macro{reserved characters+\backslash \{}
  \NC{\tt \}}\NC\cmd{\}}\NR\macro{reserved characters+\backslash \}}
  \NC{\tt \#}\NC\cmd{\#}\NR\macro{reserved characters+\backslash \#}
  \NC{\tt \lettertilde}\NC\PlaceMacro{lettertilde}\tex{lettertilde}\NR\macro{reserved characters+\backslash lettertilde}
  \NC{\tt \|}\NC\cmd{\|}\NR\macro{reserved characters+\backslash \|}
  \NC{\tt \$}\NC\cmd{\$}\NR\macro{reserved characters+\backslash \$}
  \NC{\tt \_}\NC\cmd{\_}\NR\macro{reserved characters+\backslash \_}
  \NC{\tt \letterhat}\NC\PlaceMacro{letterhat}\tex{letterhat}\NR\macro{reserved characters+\backslash letterhat}
  \NC{\tt \&}\NC\cmd{\&}\macro{reserved characters+\backslash \&}\NR
  \HL
\stoptabulate}

Dalším způsobem, jak dostat vyhrazené znaky do výstupu, je použití
příkazu \tex{type}. Tento příkaz odešle to, co převezme jako
argument, bez jakéhokoli zpracování či interpretace. V konečném
dokumentu je text, zapsaný v příkazu \tex{type}, zobrazen neproporciálním
písmem typickým pro počítačové terminály a psací stroje.

\startSmallPrint

  Normálně bychom text, který má \tex{type} zobrazit, zapsali do
  složených závorek. Pokud však tento text sám o sobě obsahuje
  složené závorky, musíme místo nich zapsat text
  mezi dva stejné znaky, které nejsou součástí textu, který
  je argumentem příkazu \tex{type}. Například: \cmd{type*\{*},
  or \cmd{type+\}+}.

\stopSmallPrint

Pokud omylem použijeme některý z vyhrazených znaků přímo, jinak než k účelu,
ke kterému je určen (protože jsme zapomněli, že se jedná o vyhrazený znak a
nelze jej použít jako normální znak), mohou se stát tři věci:

\startitemize[n]

  \item Nejčastěji dojde v průběhu zpracovnání k chybě.

  \item Výsledek bude jiný, než jsme čekali. Speciálně k tomu dochází v případě
  \MyKey{\lettertilde} a \MyKey{\%}. V prvním případě bude v textu místo
  znaku \MyKey{\lettertilde}, který očekáváme, jen mezera. V druhém případě nebude
  v textu nic, co bylo na na stejném řádku za znakem \MyKey{\%}. I nesprávné použití
  \quotation{{\tt\backslash}} může způsobit nečekaný výstup. To v případě,
  kdy za ním následuje znak nebo text, který \ConTeXt\ \quotation{umí}.
  Nicméně nesprávné použití \quotation{{\tt\backslash}} obvykle způsobí chybu
  při zpracování.

  \item Nedojde k problému. U vyhrazených znaků, které se používají v prostředí
  matematických výrazů ({\tt _ ^ &}), pokud jsou použity mimo toto prostředí
  jsou zpracovány jako běžné znaky.

  \startSmallPrint

    Bod 3 je můj závěr. Pravdou je, že jsem nikde v dokumentaci \ConTeXt{}u
    nenašel uvedeno, že tytto třei znaky mohou být uvedeny \Conjecture{} přímo.
    Nicméně při svých testech jsem při jejich použití nenarazil na žádnou chybu.
    Narozdíl od \LaTeX{}u například.

  \stopSmallPrint

\stopitemize

\stopsection

\startsection
  [
    title=Samotné příkazy,
    reference=sec:commands themselves,
  ]

Samotné příklady začínají vždy znakem \quotation{{\tt\backslash}}.
V závislosti na tom, co následuje bezprostředně za ním, rozlišujeme:

\startitemize[a]

  \item {\bf Řídící znaky.} Řídící znak začíná escape znakem
  (\quotation{{\tt\backslash}}) a skládá se výhradně z jiných znaků,
  než z písmen. Jako například \quotation{\tex{,}},
  \quotation{\tex{1}}, \quotation{\tex{'}} nebo \quotation{\type{\%}}. Jakýkoliv
  znak nebo symbol, který není písmenem v doslovném slova smyslu, může být řídícími
  znakem. Včetně čísel, interpuknčních znamének nebo dokonce mezer.
  V tomto dokumentu, pokud potřebuji zviditelnit mezeru (bílý znak) tam,
  kde je potřeba její přítomnost zdůraznit, používám symbol \textvisiblespace.
  Ve skutečnosti je \quotation{\cmd{\textvisiblespace}} (zpětné lomítko
  následované mezerou) běžne používaný řídící znak, jak brzy uvidíme.

  \startSmallPrint\reference[note:invisible space]{}

    Mezera, nebo tzv. bílý znak, je \quotation{neviditelný} znak, což
    je, v dokumentu jako je tento, problém. Protože někdy potřebujeme
    jasně zobrazit, že ve zdrojovém souboru má být mezera zapsána. {\sc
    Knuth} si byl tohoto problému vědom a v jeho knize \quotation{The
    \TeX Book} zavedl zvyk zobrazovat významné mezery pomocí symbolu
    \quotation{\textvisiblespace}. Takže pokud je například třeba,
    aby ve zdrojovém souboru byla zapsána dvě slova oddělená dvěmi mezerami,
    můžeme to zapsat jako \quotation{slovo1\textvisiblespace\textvisiblespace slovo2}.

  \stopSmallPrint

  \item {\bf Řídící slova.} Pokud je bezprostředně za zpětným lomítkem
  písmeno, příkazem je {\em řídící slovo}. Tato skupina příkazů je nejpočetnější.
  Její vlastností je, že jméno příkazu může být tvořeno pouze písmeny.
  Čísla, interpukční znaménka, ani jiné symboly nejsou ve jméně příkazu povolena.
  Pouze malá a velké písmena. A je třeba mít na paměti, \ConTeXt\ rozlišuje
  mezi malými a velkými písmeny. Což znamená, že příkazy \tex{mycommand} and
  \tex{MyCommand} jsou dva různé příkazy. Na druhou stranu příkazy \tex{MyCommand1}
  a \tex{MyCommand2} budou vyhodnoceny jako shodné, protože znaky \quote{1}
  a \quote{2} nejsou písmena a nejsou tudíž zpracovány jako součást názvu příkazu.

  \startSmallPrint

    Referenční příručka \ConTeXt{}u neobsahuje popis pravidel
    pro tvorbu jmen příkazů. Stejně tak žádný z dalších \quotation{manuálů}
    zahrnutých v \Conjecture\suite-. To co uvádím v předchozím odstavci
    vychází z toho, co platí v \TeX{}u (kde tedy, mimochodem, nejsou znaky
    s interpukncí, které nejsou součástí anglické abecedy, zahrnuty
    mezi \quotation{písmena}). Tato pravidla také nabízí dobré vysvětlení
    mizení mezery za názvem příkazu.

  \stopSmallPrint

\stopitemize

Když \ConTeXt\ čte zdrojový soubor a narazí na escape znak (\quotation{{\tt\backslash}})
ví, že bude následovat příkaz. Pak přečte první nasledující znak. Pokud to není písmeno,
znamená to, že příkaz je řídicí symbol a skládá se pouze z tohoto prvního symbolu.
Ale na druhou stranu, pokud první znak po escape sekvenci je písmeno, pak pokračuje
\ConTeXt\ ve čtení dalších znaků do doby, než narazí na první znak, který není písmenem.
V tom okamžiku ví, že jméno příkazu skončilo. To je důvod, proč názvy příkazů,
které jsou řídicími slovy, nemohou obsahovat znaky, které nejsou písmeny.

Pokud je \quotation{ne-písmenný} znak na konci příkazu mezera předpokládá se,
že není součástí jména příkazu, ale že je v kódu jenom proto, aby oddělila
jméno příkazu od zbytku textu. Takže \ConTeXt\ tuto mezeru zahodí.
Výsledný efekt překvapuje \ConTeXt{}ové začátečníky. Protože ve výstupu je
výsledek příkazu spojen s následujícím textem.
Například, tyto dvě věty:

{\switchtobodyfont[small]
\starttyping
Znát \TeX pomůže, pokud se chcete naučit \ConTeXt.
Znát \TeX, jakkoliv to není nezbytné, pomůže, pokud se chcete naučit \ConTeXt
\stoptyping
}

dopadne ve výstupu takto:

{
\tfx\color[red]{Znát \TeX pomůže, pokud se chcete naučit \ConTeXt.\\ %
Znát \TeX, jakkoliv to není nezbytné, pomůže, pokud se chcete naučit \ConTeXt.%
}\footnote{{\bf Poznámka:} v případech, kdy chci demonstrovat zapsaný kód
zdrojového souboru a výsledek ve výstupu, používám v tomto dokumentu dva způsoby:
buď jsou zdroj a výstup zapsány vedle sebe ve dvousloupcovém odstavci, nebo je zdrojový
kód zapsán \color[puprple]{fialovou barvou}, kterou je obecně používána pro
zápis příklazů \ConTeXt{}u, a výsledný výstup je zobrazen červeně.}}

Všimněte si, že v prvním případě je slovo \quotation{\TeX} spojeno s následujícím
slovem. Ve druhém případě tomu tak není. Je to tím, že v prvním případě je prvním
\quotation{ne-písmenným} znakem mezera, kterou \ConTeXt\ potlačí, protože
předpokládá, že jen značí konec příkazz. Ve druhém případě je tím znakem čárka, jejíž
výstup potlačen není.

A tento problém nevyřeší přidání mezery navíc. Tedy například zápis:

\color[darkmagenta]{{\tt Znát \backslash TeX\textvisiblespace\textvisiblespace
pomůže, pokud se chcete naučit \backslash ConTeXt}}\footnote{Pro vysvětlení symbolu \quotation{\textvisiblespace} viz poznámku na \at{straně}[note:invisible space].}.

není řešením, protože jiné pravidlo \ConTeXt{}u (které je vysvětleno v
 \in{kapitole}[sec:spaces]) je takové, že prázdný znak pohltí všechny následující
 prázdné znaky a tabelátory. Tento problém, naštěstí, nenastává příliš často a řešením
 je to, aby první \quotation{ne-písmenný} znak za názvem příkazu nebyla mezera.
 A kandidáti na takový znak jsou dva:

\startitemize[1]

  \item Vyhrazený znak \MyKey{\{\}}. Ten, jak jsme si řekli, zahajuje skupinu.
  A znak \MyKey{\}} skupinu uzavírá. Takže posloupnost znaků \MyKey{\{\}}
  představuje prázdnou skupinu. Ta se nijak neprojeví ve výsledném dokumentu,
  ale pomůže \ConTeXt{}u odlišit konec jména příkazu. Je také možné vytvořit
  skupinu obsahující příkaz, například \quotation{\{\tex{TeX}\}}. Ani v jednom
  případě není prvním \quotation{ne-písmenným} znakem za jménem příkazu mezera.

  \item Řídící znak \quotation{\cmd{\textvisiblespace}} (zpětné
  lomítko následované mezerou - viz poznámku na \at{str.}[note:invisible space]).
  Tento řídící znak vloží do výstupu mezeru. Pro správné pochopení logiky
  \ConTeXt{}u stojí za to podívat se v klidu na to, co se děje, když po sobě
  následuje řídící sloco (například \tex{TeX}) a řídící znak
  (např. \quotation{\cmd{\textvisiblespace}}):

  \startitemize[2, packed]

    \item \ConTeXt\ zaregistruje znak \backslash\ následovaný
    \quote{T}. Tudíž ví, že jde o řídící slovo a pokračuje ve čtení další
    znaků, dokud nenarazí na \quotation{ne-písmenný} znak. K tomu dojde,
    když narazí nad další znak \backslash\, uvozující kontrolní znak.

    \item Jakmile zjistí, že jméno příkazu je \tex{TeX}, zpracuje příkaz
    a vytiskne \TeX\ do výsledného dokumentu. Poté se vrátí do bodu, kde
    skončil a přešte další znak následující za druhým zpětným lomítkem.

    \item Zjistí, že následuje mezera, tedy \quotation{ne-písmenný} znak.
    A to znamená, že jde o žídící znak, který už je přečtený. Tudíž
    zapíše do výsledného souboru mezeru.

    \item Nakonec se vrátí zpět ke čtení zdrojového souboru a pokračuje dál.

  \stopitemize

\stopitemize

Tento mechanismus jsem podrobně vysvětlil, protože vypuštění mezer
často překvapuje nováčky. Je však třeba poznamenat, že tento problém je
relativně málo častý. Protože řídící slova se obvykle nezapisují do výsledného
dokumentu, ale ovlivňují jeho formát a struktur. Naproti tomu řídící znaky
obvykle do výsledného dokumentu zapisují.

\startSmallPrint

  Třetím způsobem, jak se vyhnout problému s prázdným znakem. A tím je definice
  vlastního příkazu, který již bude obsahovat \quotation{ne-písmenný} znak na konci
  názvu. Například zápis:

  \type{\def\txt-{\TeX}}

vytvoří příkaz nazvaný \tex{txt}, který udělé to stejné, jako příkaz
\tex{TeX} a funguje správně jen v případě, že je volán s pomlčkou na konci
\tex{txt-}. Pomlčka není, technicky, součástí jména příkazu, ale ten nebude
fungovat, pokud nebude uvedena. Má to co dělat s mechanismem definice maker
v \TeX{}u a ta je příliš komplikovaná na to, abych ji vysvětloval zde.
Ale funguje to: jakmile je příkaz definován, pokaždé, když použijeme \tex{txt-},
\ConTeXt\ jej nahradí \tex{TeX} a odstraní pomlčku, ale vnitřně podle ní pozná
konec jména příkazu. Takže následující mezera nebude vynechán.

Tento \quote{trik} nebude správně fungovat v příkazu \tex{define}, což je specifický
příkaz \ConTeXt{}u pro definic maker.

\stopSmallPrint

\stopsection

\startsection
  [title=Rozsah působnosti příkazů]

\startsubsection
  [
    reference=sec:command scope,
    title=Příkazy vyžadující a nevyžadující specifikaci rozsahu,
  ]

Mnoho příkazů \ConTeXt{}u, zejména ty, které ovlivňují formátování
(tučné písmo, kurzíva, malá písmena atd.), nastavují určité chování do doby,
dokud se neobjeví jiný příkaz, který ji zakáže nebo který nastaví jinou funkci,
která s ní není kompatibilní. Například příkaz \tex{bf} povoluje tučné písmo a
zůstane aktivní, dokud nenarazí na {\em nekompatibilní} příkaz, jako např.
\tex{tf} nebo \tex{it}.

Tyto typy příkazů nepotřebují žádné argumenty, protože nejsou určeny pouze pro
určitý text. Jejich funkce je limitována pouze na {\em nastavení} nějaké vlastnosti
(tučné písmo, kurzíva, bezpatkové písmo, určitá velikost písma atd.).

When these commands are executed within a {\em group} (see
\in{section}[sec:groups]), they also lose their effectiveness when the
group they are executed in is closed. Therefore, often in order to make
these commands affect only a portion of text, what is done is to
generate a group containing that command and the text we want it to
affect. A group is created by enclosing it between curly brackets.
Therefore, the following text

Pokud jsou tyto příkazy prováděny v rámci skupiny (viz \in{kapitolu}[sec:groups]),
končí jejich účinek také v okamžiku, kdy skončí skupina, v níž jsou prováděny.
Proto se často, aby tyto příkazy ovlivnily pouze část textu, dělá to, že se vytvoří
skupina, která obsahuje daný příkaz. Takto text

\startDoubleExample

\starttyping
V {\it The \TeX Book}, {\sc Knuth}
vysvětluje vše, co potřebujete
vědět o \TeX{}u.
\stoptyping

V {\it The \TeX Book}, {\sc Knuth} vysvětluje vše, co potřebujete
vědět o \TeX{}u.

\stopDoubleExample

vytváří dvě skupiny. Jedna definuje rozsah příkazu \tex{it} (kurzíva)
a druhá příkazu \tex{sc} (kapitálky).

Naopak potom existují příkazy, které kvůli svému účinku
nebo z jiných důvodů vyžadují výslovné uvedení toho, na jaký text mají být
aplikovány. V těchto případech je text, který má být příkazem ovlivněn,
uveden ve složených závorkách {\em bezprostředně za příkazem}. Jako příklad
můžeme uvést příkaz \tex{framed}: tento příkaz vykreslí rámeček kolem textu, který
je uveden jako argument. Například:

{\tfx\type{\framed{Tweedledum and Tweedledee}}}

vygeneruje\blank

\example{\framed{Tweedledum and Tweedledee}}

Všimněte si, že ačkoli se v první skupině příkazů (těch, které vyžadují argument)
někdy k určení rozsahu působnosti používají složené závorky, nejsou nutné k tomu,
aby příkaz fungoval. Příkaz je určen k použití od místa, kde se objeví.
Při určování jeho rozsahu  působnosti pomocí je příkaz umístěn {\em uvnitř těchto závorek}.
V druhém přáadě závorky rámují text, na který se má příkaz vztahovat, a jsou uvedeny
až {\em za příkazem}.

U příkazu \tex{framed} je zřejmé, že efekt, který způsobí, potřebuje argument
-- text, na který má být aplikován.
V principu ale záleželo na programátorovi, o který typ příkazu půjde.
Například příkazy \tex{it} a \tex{color} dělají podobnou věc: nastavují nějakou
vlastnost (formát nebo barvu. Ale z rozhodnutí vývojářů se první používá bez
argumentu a druhý s argumentem.

\stopsubsection

\startsubsection
  [title=Příkazy vyžadující výslovné označení místa, kde začínají a končí (prostředí)]

Existují určité příkazy, které určují svůj rozsah přesným uvedením
bodu, ve kterém začíná jejich platnost, a bodu, ve kterém končí.
Tyto příkazy se tedy vyskytují ve dvojicích: jeden označuje začátek
platnosti a druhý konec. \quotation{start}, za kterým následuje
názvem příkazu, se používá k označení začátku akce, a \quotation{stop},
následovaný názvem příkazu, k označení konce. Takže například příkaz
\MyKey{itemize} je představován \tex{startitemize} na začátku platnosti
a \tex{stopitemize} označujícím její konec.

V oficiální dokumentaci \ConTeXt{}u nemají tyto párové příkazy žádné speciální
označení. Referenční příručka je označuje jednoduše jako
\quotation{start ... stop}. Někdy jsou nazývány jako {\em prostředí},
což je ale název, který používá \LaTeX\ pro podobnou konstrukci.
Nevýhodou tohoto označení je také to, že v \ConTeXt{}u je termín
\quotation{prostředí} používán pro něco jiného (speciální druh souboru,
o kterém bude řeč u vícesouborových projektů v \in{kapitole}[sec-projects]).
Přesto, jelikož pojem prostředí je jasný a srozumitelný, a z kontextu
bude vždy zřejmé, zda je řeč o {\em příkazech prostředí} ne {\em souborech prostředí},
budu toto označení používat.

Prostředí se tedy skládají z příkazu, který je otevírá nebo spouští, a
příkazu, který je uzavírá nebo ukončuje. Pokud zdrojový soubor obsahuje
příkaz pro otevření prostředí, který není později uzavřen, bude obvykle
vygenerována chyba.\footnote{I když ne vždy; záleží na prostředí a na
zbytku dokumentu. \ConTeXt\ se v tomto liší od \LaTeX{}u, který je při
kontrole mnohem striktnějšíin this regard, which is much
stricter.}
Tento druh chyb obtížnější nalézt, protože chyba může vzniknout
daleko za místem, kde se příkaz k otevření vyskytuje. Někdy lze v \MyKey{.log}
souboru nalézt řádek, na kterém začíná nesprávně uzavřené prostředí.
Jindy se ale chyba neuzavření prostředí znamená, že \ConTeXt\ špatně
interpretuje určitou pasáž a ne v tomto chybném prostředí, což znamená,
že \MyKey{.log} soubor v totmo případě nepomůže zjistit, jde je problém.

Prostředí mohou být vnořená, což znamená, že další prostředí může být otevřeno
v rámci existujícího prostředí. Je však nutné dodržet hierarchii a vnořené
prostředí musí být uzavřeno uvnitř prostředí, ve kterém bylo otevřeno.
Jinými slovy pořadí uzavírání prostředí musí odpovídat pořadí jejich otevírání
(v opačném pořadí). Domnívám se, že by to mělo být zřejmé z následujícího
příkladu:

{\switchtobodyfont[small]
\startframedtext
\starttyping
\startSomething
  ...
  \startSomethingElse
    ...
    \startAnotherSomethingElse
      ...
    \stopAnotherSomethingElse
  \stopSomethingElse
\stopSomething
\stoptyping
\stopframedtext
}

V příkladu je vidět, že prostředí \MyKey{AnotherSomethingElse}
bylo otevřeno uvnitř prostředí  \MyKey{SomethingElse} a musí
v něm být také uzavřeno. V opačném případě vznikne při zpracování
chyba.

Obecně lze říci, že příkazy koncipované jako prostředí jsou takové,
které jsou určeny k definici nějaké vlastnosti na text o velikosti
minimálně odstavce. Například prostředí \MyKey{narrower}, které mění
okraje, má smysl pouze v případě, že se jedná o odstavec. Nebo
prostředí \MyKey{framedtext}, které orámuje jeden nebo více
odstavců. Toto druhé prostředí nám může pomoci pochopit, proč některé
příkazy jsou navrženy jako prostředí a jiné jako jednotlivé příkazy:
pokud chceme orámovat jedno nebo více slov na stejném řádku, použijeme příkaz
\tex{framed}, ale pokud chceme orámovat celý odstavec (nebo několik odstavců),
použijeme příkaz pak použijeme prostředí \MyKey{framedtext}.

Dále je třeba pochopit, že text umístěný v nějakém prostředí, současně
vytváří {\em skupinu} (viz \in{kapitolu}[sec:groups]). A tedy pokud
je uvnitř prostředí aktivován nějaký příkaz, aplikuje se pouze na
zbývající část tohoto prostředí a jeho účinek končí s koncem prostředí.
A \ConTeXt\ má, de facto, nepojmenované {\em prostředí} začínající
příkazem \tex{start} (bez dalšího textu, prostě {\em start}. Proto
ho nazývám {\em nepojmenované prostředí}) a končí příkazem \tex{stop}.
Předpokládám, že jeho jediná funkce je vytvořit skupin.

\startSmallPrint

  V dokumentaci \Conjecture\ConTeXt\ jsem se nikde nedočetl,
  že by jedním z efektů prostředí bylo i skupinování jejich obsahu.
  Ale je to výsledek mých testů z řadou předdefinovaných prostředí,
  byť musím přiznat, že nebyly zdaleka vyčerpávající. Zkrátka jsem
  vyzkoušel nějaká náhodně vybraná prostředí.
  Moje testy nicméně ukázaly, že je tato skutečnost platná jen pro
  některá předdefinovaná prostředí. Prostředí vytvořená příkazem
  \tex{definestartstop} (popsaná v \in{kapitole}[sec:startstop])
  nevytváří skupiny, pokud do vytvoření prostředí nezahrneme i vytvoření
  skupiny (viz \in{kapitolu}[sec:groups]).

  Také chování prostředí, které nazývám {\em nepojmenované} (\tex{start})
  je pouze mým předpokladem. Skupinu vytváří, ale jestli má i jiné využití
  nevím. Je to jeden z příkazů, nezdokumentovaných v referenční příručce.

\stopSmallPrint

\stopsubsection

\stopsection

\startsection
  [
    title=Parametry provádění příkazů,
    reference=sec:command options,
  ]

\startsubsection
  [title=Příkazy, které mohou fungovat několika různými způsoby]

Řada příkazů může fungovat více než jedním způsobem. V takových případech
existuje vždy výchozí chování, který lze změnit pomocí parametrů požadovné
operace uvedených v hranatých závorkách za názvem příkazu.

Dobrým příkladem toho, co jsem právě uvedl, je \tex{framed},
zmíněném v předchozí kapitole. Tento příkaz vykreslí rámeček
kolem textu, který bere jako argument. Ve výchozím nastavení má rámeček
výšku a šířku dle textu, na který je aplikován. Můžeme však uvést i
jinou výšku a šířku. Můžeme tak vidět rozdíl mezi tím, jak
funguje ve výchozím nastavení \tex{framed}:

\startDoubleExample

  \type{\framed{Tweedledum}}

  \framed{Tweedledum}

\stopDoubleExample

a při upraveném spuštění:

\startDoubleExample

\starttyping
\framed
  [width=3cm, height=1cm]
  {Tweedledum}
\stoptyping

\framed
  [width=3cm, height=1cm]
  {Tweedledum}

\stopDoubleExample

Ve druhém příkladu jsme v hranatých závorkách uvedli konkrétní šířku a
výšku rámečku, který obklopuje text v předanáém argumentu. Uvnitř závorek
jsou uvedeny různé možnosti odděleny čárkou. Prázdné znaky a dokonce i
zalomení řádků (pokud nejde o dvojité odřádkování) nejsou zohledňovány.
Takže například další čtyři varianty příkazu vedou k naprosto stejnému
výsledku:

{\switchtobodyfont[small]
\startframedtext
\starttyping
\framed[width=3cm,height=1cm]{Tweedledum}

\framed[width=3cm,    height=1cm]{Tweedledum}

\framed
  [width=3cm, height=1cm]
  {Tweedledum}

\framed
  [width=3cm,
    height=1cm]
  {Tweedledum}

\stoptyping
\stopframedtext
}

Je myslím zřejmé, že poslední varianta je nejlépe čitelná. Na první pohled vidíme
kolik parametrů existuje a jak se používají. V příkladu, jako je tento, s
pouze dvěma možnostmi, by se to možná nemuselo zdát tak důležit, ale v případech,
kdy existuje dlouhý seznam parametrů, je pro uživatele pochopit, co má být vysledkem
v \ConTeXt{}u, pokud je každý zapsán na vlastní řádek. I odhalení případné chyby
je v takovém případě snazší. Proto je tento formát (nebo podobný) pro
zápisu příkazů uživateli "preferován".

Pokud jde o syntaxi konfiguračních voleb, podívejte se dále do \in{kapitoly}[sec:syntax].

\stopsubsection

\startsubsection
  [title=Příkazy upravující chování jiných\\příkazů (\cmd{setupSomething})]

Ukázali jsme si, že příkazy, které podporují parametry, mají vždy definovány
jejich výchozí hodnoty, tj. své výchozí chování. Pokud jeden z těchto
příkazů voláme několikrát a chtěli bychom změnit toto výchozí chování pro všechna
volání místo toho, abychom jej měnili při každém volání, existuje snažší a eketivnější
způsob, jak to udělat. K tomuto účelu je skoro vždy k dispozici příkaz, jehož název
začíná \tex{setup}, a následuje název přílazu, jehož výchozí nastavení chceme změnit.

I v tomto případě můžeme jako dobrý příklad použí příkaz  \tex{framed}. Pokud tedy
máme v dokumentu spoustu rámečků, u nichž potřebujeme použít přesný rozměr, můžeme
nastavit chování příkazu \tex{framed} pomocí příkazu \tex{setupframed}. Takže

{\switchtobodyfont[small]
\starttyping
\setupframed
  [
    width=3cm,
    height=1cm
  ]
\stoptyping
}

zajistí, že pokaždé, když použijeme příkaz \tex{framed} bez dalších parametrů,
vygeneruje rámeček široký 3 cm a vysoký 1 cm, bez toho, abychom mu to muselo pokaždé
nastavovat.

\ConTeXt\ obsahuje okolo 300 příkazů umožňujících definovat chování jiných příkazů.
Můžeme tak konfigurovat výchozí chování rámečků (\tex{framed}), výčtů (\MyKey{itemize}),
názvů kapitol (\tex{chapter}), částí (\tex{section}), atd.

\stopsubsection

\startsubsection
  [title=Vytváření vlastních variant konfigurovatelných příkazů (\cmd{defineSomething})]

Když bychom pokračovali s příkladem příkazu \tex{framed}, tak je obvyklé,
že v dokumentu používáme různé typy rámečků a každý má jiné rozměry. Bylo
by tedy idální, kdybychom si mohli {\em předdefinovat} různé varianty
a nastavení  \tex{framed} a pojmenovat je. A následně používat tyto pojmenované
varianty. To lze v \ConTeXt{}u provést pomocí příkazu \tex{defineframed} se syntaxí:

\type{\defineframed[Jmeno][Nastaveni]}

kde {\em Jmeno} je naše vlastní pojmenování konkrétní varianty nastavení daného
typu rámečku a {\em Nastaveni} je konkértní nastavení parametrů pro tuto variantu.

Výsledkem bude, že uvedená konfigurace bude spojena se jménem, které jsme
vytvořili a bude fungovat tak, jako by to byl nový příkaz. Přičemž jej můžeme
použít v jakémkoli místě, kde bychom mohli použít původní příkaz \tex{framed}.

Tato možnost samozřejmě existuje nejen konkrétně pro příkaz \tex{framed},
ale pro spoustu příkazů, které mají svou \tex{setup} variantu. Kombinace
\tex{defineSomething} + \tex{setupSomething} je mechanismus který dává \ConTeXt{}u
jeho extrémní sílu a flexibilitu. Pokud bychom podrobně zkoumali, co přesně
vykonání příkazu \tex{defineSomething} dělá, zjistilo bychom následující:

\startitemize[packed]

  \item Nejprve vytvoří kopii konkrétního příkaz se všemi jeho dostupnými parametry.

  \item Následně spojí tuto kopii s požadovaným jménem.

  \item A nakonec nastaví požadované výchozí chování, odlišné od původního příkazu.

\stopitemize

V příkladu, který jsme si uvedli, jsme konfigurovali náš speciální rámeček současně
s jeho definicí. Můžeme jej však také nejprve vytvořit a nakonfigurovat později.
Protože, jak jsem již uvedl, jakmile je klon vytvořen, lze jej použít všude tam, kde by
mohl být použit původnípříkaz. Pokud jsme tedy například vytvořili rámeček s názvem
\MyKey{MySpecialFrame}, můžeme jej nakonfigurovat pomocí \tex{setupframed} s uvedením
konkrétního jména rámečku. V tomto případě příkaz \tex{setup} přijme nový argument
s názvem rámečku, který má být nakonfigurován:

{\switchtobodyfont[small]
\vbox{\starttyping
\defineframed[MySpecialFrame]

\setupframed
  [MySpecialFrame]
  [ ... ]
\stoptyping
}}

\stopsubsection

\stopsection

\startsection
  [
    reference=sec:syntax,
    title={Shrnutí syntaxe příkazů,\\ jejich parametrůoptions a použití hranatých a složených závorek při jejich volání},
  ]
  % this section is especially dedicated to LaTeX users, so
  % they can understand the different use of such brackets.

Shrneme-li si to, co jsme dosud viděli, vidíme, že v \ConTeXt{u}

\startitemize

  \item Příkazy, jako takové, začínají vždy znakem \quotation{{\tt\backslash}}.

  \item Některé příkazy mohou mít jeden nebo více argumentů.

  \item Parametry, které sdělují příkazu {\em jak} má proběhnout nebo
  jinak ovlivňují to, jak se chová, se uvádějí v hranatých závorkách.

  \item Parametry určující na kterou část textu má být příkaz aplikován
  se uvádějí ve složených závorkách.

  \startSmallPrint

    Pokud se příkaz provádí pouze na jednom znaku, jako například
    v případě příkazu \tex{buildtextcedilla} (jen pro vysvětlení
    -- the \quote{ç} se často užívá v katalánštině), složené závorky
    kolem parametru (textu) můžeme vynechat: příkaz se provede
    na prvním znaku, který není mezera.

  \stopSmallPrint

  \item Některé parametry mohou být nepovinné, v takovém případě je můžeme
  vynechat. Nikdy však nemůžeme měnit pořadí argumentů, které příkaz očekává.

\stopitemize

  Parametry předáváne v hranatých závorkách mohou být několika typů.
  Zejména:

\startitemize

  \item Mohou nabývat pouze jediné hodnoty, kterou bude téměř vždy jedno
    slovo nebo fráze

  \item Mohou nabývat více hodnot. V takovém případě může jít o:

  \startitemize

    \item Jedno slovo - symbolické označení (jehož význam \ConTeXt\ zná),
    míru nebo rozměr, číslo, název jiného příkazu apod.

    \item Názvy proměnných, které musí mít přiřazenu hodnotu. V takovém
    případě definice příkazu (viz \in{kapitolu}[sec:qrc-setup-en]) vždy
    uvádí, jaký typ hodnoty každý parametr očekává.

    \startitemize

      \item Pokud je očekávanou hodnotou text, může obsahovat i prázdné
      znaky nebo další příkazy. V těchto případech je někdy vhodné zapsat
      hodnotu ve složených závorkách.

      \item Pokud je očekávanou hodnotou parametru příkaz, můžeme obvykle
      jako hodnotu uvést více než jeden příkaz. Někdy je potřeba všechny
      takto předávané příkazy uzavřít do složených závorek. Do složených
      závorek je třeba pramatr zapsat i v případě, že předáváný příka
      obsahuje paramtery v hranatých závorkách.

    \stopitemize

  \stopitemize

  Pokud je do parametru předáváno více hodnout, vždy se oddělují čárkami.
  Bílé znaky a konce řádků (ale ne dvojnásobné) jsou pro zpracování
  vynechávány. Stejně tak jsou vynechávány prézdné znaky a jednoduché
  konce řádků mezi jednotlivými parametry.

  \item Závěrem je dobré říci, že v \ConTeXt{}u se nikdy nestane, že by
  příkaz měl parametry jak typu jednoho slova, tak s očekávanou hodnou
  nebo proměnnou. Jinými slovy můžeme najítpříkazy jako

  \type{\command[Option1, Option2, ...]}

  a jiné jako

  \type{\command[Variable1=value, Variable2=value, ...]}

  ale nikdy nenajdeme kombinaci

  \type{\command[Option1, Variable1=value, ...]}

\stopitemize

\stopsection

\startsection
  [
    title=Oficiální seznam příkazů \ConTeXt{}u,
    reference=sec:qrc-setup-en
  ]
  % This section is written only to be able to refer to it each
  % time there is mention of an "official list" of
  % commands.

Mezi dokumentací \ConTeXt{}u je zvláště důležitý dokument se seznamem
všech příkazů, kde je u každého z nich uvedeno, kolik argumentů
očekávají a jakého druhu, a také různé předpokládané volby a jejich
povolené hodnoty. Tento dokument se nazývá \MyKey{setup|-|en.pdf} a
je automaticky generován pro každou novou verzi \ConTeXt{}u. Najdete
ho v adresáři
\MyKey{tex/texmf|-|context/doc/context/documents/general/qrcs}.

\startSmallPrint

  Ve skutečnosti má  \MyKey{qrc} sedm verzí tohoto dokumentu, jednu
  pro každý z jazyků, které mají rozhraní \ConTeXt{}u: německy, česky,
  francouzsky, holandsky, anglicky, italsky a rumunsky. Pro každý z těchto
  jazyků jsou v adresáři dva dokumenty: jeden s názvem \MyKey{setup-LangCode}
  (kde LangCode je písmenný kód pro identifikaci dvou mezinárodních jazyků)
  a druhý dokument s názvem \MyKey{setup-mapping-LangCode}. Tento druhý
  dokument obsahuje seznam příkazů v abecedním pořadí a uvádí pouze prototyp
  příkazu bez dalších informací o hodnotách argumentů.

\stopSmallPrint

Tento dokument je zásadní pro učení se \ConTeXt{}u, protože v něm můžeme
zjistit, zda určitý příkaz existuje, nebo ne. Speciálně je pak dobré mít
na paměti kombinaci{\sc command} (or {\sc environment})
+ setup {\sc command} + define{\sc command}.
Například pokud víme, že se prázdný řádek vloží příkazem \tex{blank},
můžeme v dokumentu snadno zjistit, zda existuje i příkaz \tex{setupblank},
kterým můžeme modifikovat jeho výchozí chování a příkaz \tex{defineblank},
kterým můžeme vytvořit jeho pojmenované varianty.

\startSmallPrint

  \MyKey{setup-en.pdf} je tedy základem pro učení se ConTEXtu. Ale já
  bych byl opravdu raději, kdyby nám především řekl, zda daný příkaz
  funguje pouze v  Mark~II nebo Mark~IV a zejména pak, kdyby místo pouhého
  výčtu parametrů a jejich typů obsahoval i informace, k čemu tyto parametry
  slouží. Tím by se značně omezily nedostatky dokumentace \ConTeXt{}u.
  Existují i příkazy, které umožňují použití volitelných argumenty,
  které ale v tomto úvodu ani nezmiňuji, protože nevím, k čemu slouží.
  A jelikož jsou nepovinné, není nutné se o nich zmiňovat.
  A to je nesmírně frustrující.

\stopSmallPrint

\stopsection

\startsection
  [
    reference=sec:definingcommands,
    title=Definování nových příkazů,
  ]

\startsubsection
  [
    reference=sec:define,
    title=Základní postup definování\\ nových příkazů,
  ]
\PlaceMacro{define}

We have just seen how, with \tex{defineSomething} we can clone a
pre-existing command and develop a new version of it from there, that
will function, to all intents and purposes, as a new command.
Již jsme si ukázalo, jak můžeme pomocí \tex{defineSomething}
klonovat existující příkazy a vytvořit z nich novou verzi, která bude
fungovat, ve všech ohledech, jako nový příkaz.

Vedle této možnosti, která je dostupná pouze pro některé specifické
příkazy (docela dost, ale ne všechny), má \ConTeXt\ obecný mechanismus
pro definování nových příkazů, který je mimořádně výkonný. I když v některých
případech použití také poměrně složitý. V textu, který je jako je tento
určený začátečníkům, považuji za nejlepší představit jej tím, že začneme
některými z jeho jednodušších použití. Nejjednodušší ze všech je
spojení nějakáho slova (názvu) s úryvkem textu. Takže pokaždé, když
použijete vybrané slovo, bude nahrazeno textem, který je s ním spojen.
To nám na jedné straně umožní ušetřit spoustu času při psaní, a na druhé
straně to, jako další výhoda, snižuje možnost udělat při psaní chyby.
Současně tím dosáhneme toho, že bude daný text zapsán vždy stejným způsobem.

Představme si například, že píšeme pojednání o aliteracích v latinských
textech, kde často citujeme latinskou větu \quotation{{\em O Tite tute
Tati tibi tanta tyranne tulisti}} (O Tite Tati, ty tyrane, tolik jsi
toho na sebe přivolal!). Je to poměrně dlouhá věta, v níž dvě slova
jsou vlastními jmény a začínají velkými písmeny, a kde, přiznejme si,
jakkoli můžeme mít latinskou poezii rádi, při jejím zápisu snadno
"zakopneme". V tomto případě jsme mohli jednoduše vložit do úvodu
našeho zdrojového souboru:

{\tfx \type{\define\Tite{\quotation{O Tite tute Tati tibi tanta tyranne tulisti}}}}

Na základě takové definice bude pokaždé, když se v našem zdrojovém
souboru objeví příkaz \tex{Tite}, nahrazen uvedenou větou a bude také
v uvozovkách, stejně jako v původní definici, což nám umožní zajistit,
že způsob, jakým se tato věta objeví, bude vždy stejný. Mohli bychom ji
také napsat kurzívou, větší velikostí písma... jak chceme. Důležité je,
že ji musíme napsat jen jednou a v celém textu bude reprodukována přesně
tak, jak byla napsána, tak často, jak budeme chtít. Mohli bychom také
vytvořit dvě verze příkazu, nazvané \tex{Tite} a \tex{tite}, podle toho,
zda je třeba větu napsat velkými písmeny, nebo ne. Náhradní text může
být čistě text nebo může obsahovat příkazy nebo tvořit matematické výrazy
v nichž se často chybuje (alespoň v mém případě). Například pokud se má
v našem textu pravidelně objevovat výraz $(x_1,\ldots,x_n)$, mohli bychom

{\tfx\type{\define\xvec{$(x_1,\ldots,x_n)$}}}

takže vždy, když se v textu objeví \tex{xvec}, bude nahrazen uvedeným
výrazem.

Základní syntaxe příkazu \tex{define} je následující:

\type{\define[PocetParametru]\NazevPrikazu{TextProNahrazeni}}

% {\ttx \color[maincolor]{\backslash define[{\em
%         NumArguments}]\backslash {\em CommandName}\{{\em
%       TextoReplace}\}}}

kde

\startitemize[packed]

  \item {\tt\bf PocetParametru} definuje, kolik parametrů příkaz má.
  Pokud nemá žádné, jako ve výše uvedených příkladech, můžeme tuto
  část vynechat.

  \item {\tt\bf NazevPrikazu} je název nového příkazu. Ten musí
  odpovídat obecnému providlu pro názvy příkazů. Tj. může jít
  buď o jeden znak, který není písmenem anglické abecedy, nebo
  o jedno nebo více písmen základní anglické abecedy (a-zA-Z).

  \item {\tt\bf TextProNahrazeni} obsahuje to, čím bude nahrazeno
  ve výstupu uméno příkazu, kdykoliv se ve zdrojovém textu objeví.

\stopitemize

Možnost zadávat nové příkazy s argumenty v jejich definici dává
tomuto mechanismu velkou flexibilitu, protože umožňuje nahradit
text proměnnou, předanou v parametru.

Představme si, například, že chceme nadefinovat příkaz, který
vytvoří úvod obchodního dopisu. Hodně jednoduchá verze může být::

{\switchtobodyfont[small]
\starttyping
\define\\ZahlaviDopisu{
  \rightaligned{Eda Pádlo}\par
  \rightaligned{konzultant}\par
  Horní Věrolomná, \date\par
  Vážený pane,\par
  }
\stoptyping
}

but it would be preferable to have a version of the command that would
write the name of the recipient in the header. This would require the
use of a parameter that communicates the name of the recipient to the
new command. This would require redefining the command as follows:
ale bylo by vhodnější mít verzi příkazu, která by zapisovala i jméno
příjemce v záhlaví. To by vyžaduje použití parametru, kterým novému
příkazu předáme jméno příjemce. Příkaz můžeme předefinovat takto:

{\switchtobodyfont[small]
\starttyping
\define[1]\ZahlaviDopisu{
  \rightaligned{Eda Pádlo}\par
  \rightaligned{konzultant}\par
  Horní Věrolomná, \date\par
  Vážený pane #1,\par
  }
\stoptyping
}

Všimněte si, že jsme v definici provedli dvě změny. Především jsme mezi
klíčové slovo \tex{define} a název příkazu vložili jedničku do hranatých
závorek ([1]). Tím říkáme \ConTeXt{}u, že příkaz, který definujeme, bude
mít jeden argument. Dále jsme v posledním řádku definice příkazu napsali
\quotation{{\tt Vážený pane \#1,}}, přičemž jsme použili vyhrazený znak
\MyKey{\#}. To znamená, že v místě náhradního textu, kde se objeví
\MyKey{\#1}, bude vložen obsah prvního argumentu. Pokud by měl příkaz dva
parametry, \MyKey{\#1} by odkazovalo na první parametr a \MyKey{\#2} na
druhý. Aby bylo možné příkaz zavolat (ve zdrojovém souboru), za názvem
příkazu, musí být argumenty uvedeny ve složených závorkácg (každý
argument s vlastními). Příkaz, který jsme právě definovali, by tedy měl
být v textu našeho zdrojového souboru volán následujícím způsobem:

\type{\ZahlaviDopisu{Jméno příjemce}}

% {\ttx \color[maincolor]{\backslash LetterHeading\{{\em Addressee
% name}\} }}

Například tey: \cmd{ZahlaviDopisu\{Nevěřící Tomáši\}}.

Předchozí funkci bychom mohli ještě vylepšit, protože předpokládá, že
dopis bude odeslán muži (vloží \quotation{Vážený pane} Takže bychom
mohli přidat další parametr, který by rozlišoval mezi oslovením muže
a ženy. Například tedy:
{\switchtobodyfont[small]
\starttyping
\define[1]\ZahlaviDopisu{
  \rightaligned{Eda Pádlo}\par
  \rightaligned{konzultant}\par
  Horní Věrolomná, \date\par
  #1\ #2,\par
  }
\stoptyping
}

takže by potom volání vypadalo například takto

{\tfx\type{\ZahlaviDopisu{Vážená paní}{Liduško Muzikantová}}}

although this is not very elegant (from a programming point of view). It
would be preferable for symbolic values to be defined for the first
argument (man/woman; 0/1; m/f) so that the macro itself would choose the
appropriate text according to this value. But explaining how to achieve
this requires us to get into more depth than I think the novice reader
can understand at this stage.
ačkoli to není příliš elegantní (z programátorského hlediska). Vhodnější
by to bylo, aby byly pro první argument definovány symbolické hodnoty
(muz/zena; 0/1; m/f), takže makro by následně samo vybralo vhodný text
podle této hodnoty. Vysvětlení, jak toho dosáhnout, však vyžaduje,
abychom se dostali do větší hloubky než si myslím, že začínající
čtenář může v této fázi pochopit.

\stopsubsection

\startsubsection
  [
    reference=sec:startstop,
    title=Vytváření nových prostředí,
  ]

\PlaceMacro{definestartstop}

K vytvoření nového prostředí nabízí\ConTeXt\ příkat
\tex{definestartstop} s následující syntaxí:

\type{\definestartstop[Jmeno][Kongfigurace]}

% {\ttx \color[maincolor]{\backslash definestartstop[{\em Name}][{\em
% Options}]}}

\startSmallPrint

  V {\em oficiální} definici \tex{definestartstop} (viz \in{kapitolu}
  [sec:qrc-setup-en]) je další parametr, který jsem výše neuvedl,
  protože je nepovinný a nepodařilo se mi zjistit, k čemu \Doubt slouží.
  Nevysvětluje ho ani úvodní \ConTeXt\ \quotation{Excursion}, ani neúplná
  referenční příručka. Předpokládal jsem, že tento argument (který
  se musí zadat mezi název a konfiguraci) by mohl být názvem nějakého
  existujícího prostředí, které by mohlo sloužit jako výchozí model
  pro nové prostředí. Ale mé testy ukázaly, že tento předpoklad byl
  mylný. Podíval jsem se do poštovní konference \ConTeXt\ a neviděl
  jsem žádné použití tohoto parametru.
\stopSmallPrint

kde

\startitemize

  \item {\bf Jmeno} je jmeno definovaného prostředí.

  \item {\bf Konfigurace} nám dovoluje nastavit chování nového prostředí.
  K dispozici máme následující hodnoty:

  \startitemize

    \item {\tt before} -- Příkazy, které mají být provedeny před zahájením
    prostředí.

    \item {\tt after} -- Příkazy, které mají být provedeny po ukončení
    prostředí.

    \item {\tt style} -- Styl, který má být textu v prostředí nastaven.

    \item {\tt setups} -- Sada příkazů vytvářená pomocí
      \PlaceMacro{startsetups}\tex{startsetups ... \stopsetups}. Tyto
      příkazy a jejich užití nebudu v tomto dokumentu vysvětlovat.

    \item {\tt color, inbetween, left, right} -- Nedokumentované možnosti,
    které se mi nepodařilo \Doubt zprovoznit. Co některé dělají se dá
    odhadovat z jejich názvu, například barva, ale z více testů, které jsem
    provedl, vyplývá, že při zadání různých hodnot tohoto parametru,
    nevidím v prostředí žádnou změnu.

  \stopitemize

\stopitemize

Příklad definice prostředí:

{\switchtobodyfont[small]
\starttyping
\definestartstop
  [TextWithBar]
  [before=\startmarginrule\noindentation,
    after=\stopmarginrule,
    style=\ss\sl
  ]

\starttext

První dva základní zákony lidské hlouposti jednoznačně říkají že:

\startTextWithBar

  \startitemize[n,broad]

    \item Vždy a nevyhnutelně podceňujeme počet existujících hloupých jedinců.

    \item Pravděpodobnost, že daná osoba je hloupá, je nezávislá na na jiných
    vlastnostech téže osoby.

  \stopitemize

\stopTextWithBar

\stoptext
\stoptyping
}

Výsledek bude:

\definestartstop
  [TextWithBar]
  [before=\startmarginrule\noindentation,
    after=\stopmarginrule,
    style=\sl,
  ]

\startframedtext[frame=off]

  \color[red]{První dva základní zákony lidské hlouposti jednoznačně říkají že:

\startTextWithBar

\startitemize[n,broad]

  \item Vždy a nevyhnutelně podceňujeme počet existujících hloupých jedinců.

  \item Pravděpodobnost, že daná osoba je hloupá, je nezávislá na na jiných
    vlastnostech téže osoby.

\stopitemize

\stopTextWithBar}
\stopframedtext

Pokud chceme, aby naše nové prostředí bylo současně skupinou
(\in{kapitola}[sec:groups]), tj. aby jakákoli změna normálního fungování
\ConTeXt{}u, ke které v něm dojde, skončila při opuštění prostředí, musíme
do parametru \MyKey{before} uvést příkaz \PlaceMacro{bgroup}\tex{bgroup}
a do parametru \MyKey{after} příkaz \PlaceMacro{egroup}\tex{egroup}.

\stopsubsection

\stopsection

\startsection
  [title=Další základní concepty]

Kromě příkazů existují i další pojmy, které jsou zásadní pro pochopení
logiky fungování \ConTeXt{}u. Některé z nich, díky své složitosti, nejsou
vhodné pro úvod, a proto se jimi nebudeme zabývat. Dva z nich ale nyní
prozkoumáme podrobněji: skupiny a rozměry.

\startsubsection
  [reference=sec:groups, title=Skupiny]

Skupina je přesně ohraničený fragment zdrojového souboru, který \ConTeXt\
používá jako {\em pracovní jednotku} (co to znamená, je vysvětleno níže).
Každá skupina má explixitně uvedený začátek a konec. Skupina začíná

\startitemize[packed]

  \item Vyhrazeným znakem \MyKey{\{} nebo příkazem
    \PlaceMacro{bgroup}\tex{bgroup}.

  \item Příkazem \PlaceMacro{begingroup}\tex{begingroup}

  \item Příkazem \PlaceMacro{start}\tex{start}

  \item Zahájením některých prostředí (příkaz \tex{startSomething}).

  \item Zahájením prostředí matematiky (rezervovaným znakem «\$»).

\stopitemize

a je ukončena

\startitemize[packed]

  \item Vyhrazeným znakem \MyKey{\}} nebo příkazem
    \PlaceMacro{egroup}\tex{egroup}.

  \item Příkazem \PlaceMacro{endgroup}\tex{endgroup}

  \item Příkazem \PlaceMacro{stop}\tex{stop}

  \item Ukončením prostředí (příkaz \tex{stopSomething}).

  \item Ukončením prostředí matematiky (rezervovaným znakem «\$»).

\stopitemize

Skupiny generují automaticky také některé příkazy, například \tex{hbox},
\tex{vbox} (obecně příkazy spojené s vytvářením boxů)\footnote{Pojem
{\em box} je také jeden z obecných pojmů \ConTeXt{}u, ale jeho vysvětlení
není součístí tohoto úvodz}. Mimo tyto případy (skupiny automaticky
generované určitými příkazy), musí být vždy způsob
uzavření skupiny v souladu se způsobem jejího otevření. To znamená,
že skupina která je zahájená znakem \MyKey{\{}, musí končit znakem \MyKey{\}}
a skupina zahájená příkazem \tex{begingroup} musí být končit příkazem
\tex{endgroup}. Toto pravidlo má pouze jednu výjimku, a to že skupina
začínající \MyKey{\{} může končit pomocí \tex{egroup} a naopak (kombinace
\tex{bgroup} a \MyKey{\}}). Což znamená, že dvojice \MyKey{\{} a \tex{bgroup}
a dvojice \MyKey{\}} a \tex{egroup} jsou vzájemně zcela nahraditelné.

\startSmallPrint

  Příkazy \PlaceMacro{bgroup}\tex{bgroup} a \PlaceMacro{egroup}\tex{egroup}
  byly navrženy tak, aby bylo možné definovat příkazy pro otevření skupiny
  a jiné pro její uzavření. Proto z vnitřních důvodů syntaxe \TeX{}u
  nebylo možné tyto skupiny otevírat a zavírat pomocí složených závorek,
  protože by to generovalo nepárové složené závorky a při kompilaci vžd
  končilo chybou.

  Naproti tomu příkazy \PlaceMacro{begingroup}\tex{begingroup} a
  \PlaceMacro{endgroup}\tex{endgroup} nejsou zaměnitelné
  se složenými závorkami nebo příkazy \tex{bgroup  ... \egroup}.
  Skupina započatá příkazem \tex{begingroup} musí být uzavřena pomocí
  \tex{endgroup}. Tyto poslední příkazy byly navrženy tak, aby
  umožňovaly důkladnější kontrolu chyb. Běžní uživatelé je obecně nemusí
  používat.

\stopSmallPrint

Můžeme mít i vnořené skupiny (skupina v jiné skupině). V takovém případě
musí být pořadí uzavírání skupin odpovídat jejich otevírání.
Tj. každá podskupina musí být uzavřena v rámci skupiny, ve které začala.
Mohou existovat i prázdné skupiny generované pomocí  \MyKey{\{\}}.
Prázdná skupina nemá na výsledný dokument žádný vliv, ale může být
užitečná například pro označení konce názvu příkazu.

Hlavním účelem skupin je zapouzdření jejich obsahu: definice, formáty
a přiřazení hodnot, které jsou provedeny v rámci skupiny, jsou
zpravidla \quotation{zapomenuty}, jakmile skupinu opustíme. Pokud
tedy chceme, aby \ConTeXt\ dočasně změnil svůj běžný způsob fungování,
je nejefektivnějším způsobem vytvořit skupinu a v jejím rámci toto
fungování změnit. Jakmile apk skupinu opustíme, všechny hodnoty a
formáty se obnoví do stavu před zahájením skupiny. Několik příkladů
jsme již viděli: \tex{it}, \tex{bf}, \tex{sc}. To se však neděje
pouze u formátovacích příkazů. Skupina určitým způsobem
 {\em izoluje} svůj obsah, takže jakákoli změna v kterékoli z mnoha
vnitřních proměnných, které \ConTeXt\ neustále spravuje, zůstane platná
jen tak dlouho, dokud se nacházíme ve skupině, v níž byla tato změna
provedena. Stejně tak příkaz definovaný v rámci skupiny nebude
znám mimo ni.

Takže pokud vyzkoušíme následující příklad

\starttyping
\define\A{B}
\A
{
  \define\A{C}
  \A
}
\A
\stoptyping

uvidíme, že při prvním spuštění příkazu \tex{A} odpovídá výsledek
jeho původní definici (\quote{B}). Poté jsme vytvořili skupinu a
předefinovali v ní příkaz \tex{A}. Spustíme-li jej nyní v rámci
skupiny, příkaz poskytne výsledek dle nové definice (v uvedeném
příkladu tedy \quote{C}). Ale když opustíme skupinu, ve které
byl příkaz \quote{B}. Redefinice příkazu provedená ve skupině
je skutečně  \quote{zapomenuta}, jakmile ji opustíme.

Další použití nachází skupin u příkazů nebo instrukcí, které jsou
určeny výhradně pro znak, který je za nimi napsán. V tomto případě,
pokud chceme, aby se příkaz vztahoval na více než jeden znak,
musíme znaky, na které chceme, aby se příkaz nebo instrukce
vztahovaly, uzavřít do skupiny. Tak například rezervovaný znak
\MyKey{\letterhat}, který, jak již víme, při použití uvnitř
matematického prostředí převede následující znak na horní index.
Pokud tedy napíšeme například \MyKey{\$4^2x\$}, dostaneme
\quotation{$4^2x$ }. Pokud však napíšeme \MyKey{\$4^\{2x\}\$},
\quotation{$4^{2x}$}.

A konečně: třetím použitím skupiny je případ, kdy  \ConTeXt{}u sdělujeme,
že to, co je ve skupině uzavřeno, musí být považováno za jeden celek.
To je důvod, proč jem dříve (\in{kapitola}[sec:syntax]) uváděl, že v některých
případech je lepší příkazy předávané jako parametr jiného příkazu
uzavřít do složených závorek.
\stopsubsection

\startsubsection
  [reference=sec:dimensions, title=Rozměry]

Ačkoli bychom mohli \ConTeXt\ používat bez obav o rozměry, nemohli
bychom využít všech jeho možnosti, bez zvážení úráce s nimo. Protože
typografická dokonalost dosažená \TeX{}em a jeho odvozeninami spočívá,
do značné míry, ve velké pozornosti, kterou systém vnitřně věnuje
rozměrům. Své rozměry maji znaky, mezery mezi slovy, řádky nebo odstavci,
samotné řádky, okraje, záhlaví i zápatí. Téměř pro každý prvek na
stránce, který nás napasne, existuje nějaký rozměr.

V \ConTeXt{}u jsou rozměry specifikovány desetinným číslem následovaným
jednotkou. Jednotky, které lze používat, jsou uvedeny
v \in{tabulce}[tbl:measurements].

\placetable
  [here]
  [tbl:measurements]
  {Rozměrové jednotky použitelné v \ConTeXt{}u}
{\starttabulate[|l|c|l|]
  \NC {\bf Náezv} \NC {\bf Označení v \ConTeXt{}u}\NC {\bf Ekvivalent}\NR
\NC Palec\NC in\NC 1 in $=$ 2.54 cm\NR
\NC Centimetr\NC cm\NC 2.54 cm $=$ 1 inch\NR
\NC Milimetr\NC mm\NC 10 mm $=$ 1 cm\NR
\NC Bod\NC pt\NC 72.27 pt $=$ 1 inch\NR
\NC Big point\NC bp\NC 72 bp $=$ 1 inch\NR
\NC Scaled point\NC sp\NC 65536 sp $=$ 1 point\NR
\NC Pica\NC pc\NC 1 pc $=$ 12 points\NR
\NC Didot\NC dd\NC 1157 dd $=$ 1238 points\NR
\NC Cicero\NC cc\NC 1 cc $=$ 12 didots\NR
\NC\NC em\NR
\NC\NC ex\NR
\stoptabulate
}

První tři jednotky v \in{tabulce}[tbl:measurements] jsou standardní
délkové míry. První z nich je používána v některých částech anglicky
mluvícího světa, zbylé mimo něj. Zbývající jednotky pocházejí ze
světa typografie. Poslední dvě, pro které jsem neuvedl žádný ekvivalent,
jsou relativní měrné jednotky založené na na základě aktuálního písma.
1 \quotation{em} se rovná šířce písmene \quotation{M} a \quotation{ex}
výšce písmene \quotation{x}. Použití měr vztahujících se k velikosti
písma umožňuje vytvářet makra, která vypadají stejně dobře, ať už je
v daném okamžiku použit jakýkoli zdroj. Proto se obecně doporučuje jejich
používání.

Až na velice málo vyjímek můžeme použít jakoukoliv z uvedených měrných
jedotek a \ConTeXt\ si je vnitřně převede. Kdykoli však uvádíme rozměr,
je to povinně uvést i měrnou jednotku. A to i v případě, kdy jde o rozměr
velikosti  \quotation{0}. Musíme uvést např. \quote{0pt} nebo \quote{0cm}.
Mezi číslem a názvem jednotky můžeme, ale nemusíme nechat prázdné místo.
Pokud má jednotka desetinnou část, můžeme použít oddělovač desetinných
míst, buď (.), nebo čárku (,).

Rozměry a míry se obvykle používají jako volitelné parametry příkazů.
Můžeme však také přímo přiřadit hodnotu nějaké interní míře \ConTeXt{}u,
pokud známe její název. Například odsazení prvního řádku běžného odstavce
je interně řízeno \ConTeXt{}em pomocí proměnné nazvané
\PlaceMacro{parindent}\tex{parindent}. Přiřazením hodnoty do této proměnné
změníme rozměr, který \ConTeXt\ bude od tohoto okamžiku používat. A tak
například, pokud chceme odstranit např. odsazení prvního řádku stačí,
když do našeho zdrojového souboru zapíšeme text:

\type{\parindent=0pt}

Můžeme také psát \tex{parindent 0pt} (bez rovnítka)
nebo \tex{parindent0pt} bez mezery mezi názvem rozměru a přiřazovanou
hodnotou.

However, assigning a value directly to an internal measure is considered
\quotation{inelegant}. In general, it is recommended to use the commands
that control that variable, and to do so in the preamble of the source
file. The opposite results in source files that are very difficult to
debug because not all the configuration commands are in the same place,
and it is really difficult to obtain a certain consistency in
typographical characteristics.

Nicméně přiřazení hodnoty přímo internímu rozmětu je považováno za
\quotation{neelegantní}. Obecně se doporučuje používat příkazy, které
danou proměnnou ovládají, v úvodu zdrojového souboru.
Nedodržení tohoto pravidla vede ke zdrojovém kódu, který se velmi obtížně
ladí, protože ne všechny konfigurační příkazy jsou na stejném místě, a pak
hodně obtížné dosáhnout konzistence v ve vztahu k typografickým
vlastnostem.

Některé rozměry používané \ConTeXt{}em jsou \quotation{pružné}. To znamená,
že v závislosti na situaci mohou nabývat různých hodnot. Tyto rozměry
se nastavují s pomocí následující syntaxe:

\type{\MeasureName plus MaxIncrement minus MaxDecrease}

% {\tt\backslash {\em MeasureName} plus {\em MaxIncrement} minus
%   {\em MinDecrease}}

Například

\type{\parskip 3pt plus 2pt minus 1pt}

Tímto příkazem říkáme \ConTeXt{}u, že {\em normální} hodnota
\PlaceMacro{parskip}\tex{parskip} (definující vertikální vzdálenost
mezi odstavci) je 3 body. Ale pokud to aktuální situace na stránce
vyžaduje, může být až 5 bodů (3 plus 2) nebo jen 2 body (3 mínus 1).
Výsledné rozhodnutí tak necháváme na \ConTeXt{}u s tím, že víme,
že výsledná vzdálenost bude mezi 2 a 5 body.

\stopsubsection

\stopsection

\startsection
  [title=\ConTeXt\ pro samouky]

  % Section added at the last moment, when I realised that I myself
  % had become so imbued with the spirit of ConTeXt that I was able to
  % guess the existence of certain commands.

Obrovské množství příkazů a možností \ConTeXt{}u se ukazuje jako skutečně
ohromující a může v nás vyvolat pocit, že se s ním nikdy nenaučíme dobře
pracovat. Tento dojem je ale mylný, protože jednou z výhod \ConTeXt{}u
je jednotný způsob, jakým pracuje se všemi svými strukturami. Stačí se
tak dobře naučit několik málo kostrukcí a víceméně víme, k čemu slouží
ty zbývající. A když budeme potřebovat použít nějakou novou vlastnost,
bude relativně snadné naučit se ji používat. Proto chápu tento úvod,
jako jakési {\em školení}, které vás připraví na to, abyste mohli
vyrazit na vlastní průzkum.

Abyste mohli vytvořit dokument v \ConTeXt{}u, mělo by vám stačit znát
těchto pět základních věcí (můžeme je nazývat \ConTeXt{}í {\em
Top Five}):

\startitemize[n]

  \item Vědět, jak vytvořit zdrojový soubor nebo projekt. Popsáno je to
  v \in{kapitole}[cap:sourcefile].

  \item Umět nastavit hlavní font a znát základní příkazy pro změnu fontu
  a barvy - viz (\in{kapitolu}[sec:fontscol]).

  \item Znát základní příkazy pro definici struktury dokumentu, jako jsou
  části, kapitoly, podkapitoly poad. To vše vysvětluje
  \in{kapitola}[cap:structure].

  \item Možná také vědět něco o tom, jak procovat s prostředím {\em itemize},
  což vysvětluje do detailů \in{kapitola}[sec:itemize].

  \item ...a něco málo dalšího.

\stopitemize

For the rest, all we need to know is that it is possible. Certainly no
one will use a utility if they do not know that it exists. Many of them
are explained in this introduction; but, above all, we can repeatedly
watch how \ConTeXt\ always acts when faced with a certain type of
construction:

O ostatním stačí vědět, že je to možné. Určitě nikdo nebude používat
nástroje, o kterých neví, že existují. Mnoho z nich je vysvětleno v tomto
úvodu, především však můžeme opakovaně vidět, jak se \ConTeXt\ chová
u většiny těchto nástrojů:

\startitemize

  \item Za prvé bude existovat příkaz, který něco umožní.

  \item Za druhé, skoro vždy zde bude i příkaz, který nám umožní
  nastavit a definovat výchozí chování v dané situaci. Tento příkaz
  zaříná \tex{setup} a obvykle odpovídá názvu nastavovaného příkazu.

  \item A nakonec je obvykle i možnost nadefinovat si nový příkaz
  pro spouštění dané úlohy s různými nastaveními.

\stopitemize

Chcete-li zjistit, zda tyto příkazy existují, vyhledejte si oficiální
seznam příkazů (viz \in{kapitolu}[sec:qrc-setup-en]), kde najdete také
parametry, které příkaz má. A přestože se na první pohled mohou zdát názvy
těchto parametrů poněkud záhadné, brzy zjistíte, že jsou parametry, které
se opakují v mnoha dalších příkazech a ve všech fungují stejně nebo
velmi podobně. Pokud máte pochybnosti o tom, co některý parametr dělá
nebo jak funguje, stačí si vygenerovat dokument a vyzkoušet jej. Můžete
se také podívat do bohaté dokumentace \ConTeXt{}u. Jak je ve světě
svobodného softwaru obvyklé, \suite- obsahuje, mimo jiné, zdrojové kódy
téměř veškeré své dokumentace. Nástroj jako \MyKey{grep} (pro systémy
GNU Linux) vám může pomoci vyhledat, zda příkaz nebo volba, u které
máte pochybnosti, je použit v některém z těchto zdrojových souborů.
Takže můžete mít k dispozici i příklad použití.

Takto jsem výuku \ConTeXt{}u koncipoval: v tomto \quotation{úvodu} je do
podrobna vysvětleno pět (ve skutečnosti čtyři) základních konceptů, na které
kladu důraz, a řada dalších. Při čtení by se vám měl v hlavě poskládat
jasný obraz posloupnosti: {\em příkaz, který něco provádí} -- {\em příkaz,
který jej konfiguruje} -- {\em příkaz který nám umožní vytvořit
podobný příkaz}. Naučíme se také některé ještě něco o hlavní strukturách
\ConTeXt{}u a ukážeme si, k čemu slouží.

\stopsection

\stopchapter

\stopcomponent

%%% Local Variables:
%%% mode: ConTeXt
%%% mode: auto-fill
%%% TeX-master: "../introCTX.mkiv"
%%% coding: utf-8-unix
%%% End:
%%% vim:set filetype=context tw=72 : %%%
}