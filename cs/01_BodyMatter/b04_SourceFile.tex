\environment ../introCTX_env.mkiv

\startcomponent b04_SourceFile.mkiv

\startchapter
  [reference=cap:sourcefile, title=Zdrojové soubory a~projekty]

\TocChap

Jak již víme, při práci s~\ConTeXt{}em vždy začínáme textovým souborem 
ve kterém je spolu s~obsahem výsledného dokumentu již zahrnuta řada pokynů, 
které informují \ConTeXt\ o~transformacích, 
které musí použít k~vygenerování našeho finálně správně naformátovaného 
dokumentu v~PDF ze zdrojového souboru.

Myslíme na čtenáře, kteří dosud uměli pracovat pouze se slovem
procesory, myslím, že stojí za to strávit nějaký čas se zdrojovým souborem
samotným. Nebo spíše se zdrojovými soubory, protože jsou chvíle, kdy existuje pouze
jeden zdrojový soubor a~další, když k~průchodu použijeme více zdrojových souborů
u~závěrečného dokumentu. V~tomto posledním případě můžeme mluvit
o~\quotation{vícenásobné projekty}. 

\startsection
  [title=Kódování zdrojových souborů, reference=sec:encoding]

Zdrojové soubory musí být textové soubory. V~počítačové terminologii je to pak
název souboru, který obsahuje pouze text čitelný pro člověka, který
neobsahuje binární kód. Tyto soubory se také nazývají 
{\em jednoduché textové} nebo {\em prosté textové} soubory.

Protože interně počítačové systémy zpracovávají pouze binární čísla, textový soubor
se ve skutečnosti skládá z~{\em čísel}, která představují {\em znaky}. A~{\em tabulka} 
se používá ke spojení čísla se znakem. Pro textové soubory tam
je několik možných tabulek. Termín {\em kódování textového souboru} odkazuje
na konkrétní znak|-|odpovídající tabulce, kterou konkrétní textový soubor používá. 

\startSmallPrint

  Existence různých kódovacích tabulek pro textové soubory je
  důsledek historie samotné informatiky. Na začátku
  vývojové fáze, kdy paměť a~úložná kapacita počítače
  zařízení byla vzácná, bylo rozhodnuto použít tabulku nazvanou ASCII (což
  znamená \quotation{{\em Americký standardní kód pro výměnu informací}})
  který umožňoval pouze 128 znaků a~byl založen v~roce 1963 v~USA
  výbor pro standardy. Je zřejmé, že 128 znaků je málo pro
  představu všech znaků a~symbolů používaných ve všech jazycích
  světa; ale na reprezentaci angličtiny to bylo víc než dost
  Západní jazyk, ten, který má méně znaků, protože
  nepoužívá žádnou diakritiku (přízvuky a~jiné značky nad nebo pod nebo skrz
  jiná písmena). Výhodou použití ASCII bylo, že textové soubory
  zabírají velmi málo místa, protože 127 (nejvyšší číslo v~tabulce) může
  být reprezentováno 7místným binárním číslem a~prvními použitými počítači
  používající byte jako jednotku pro měření paměti, 8místné binární číslo. Žádný
  znak v~tabulce by se nevešel do jednoho bajtu. Protože bajt má 8
  číslice a~ASCII používaly pouze 7 číslic, zbylo dokonce místo na přidání
  některých dalších znaků reprezentující jiné jazyky. 

  Ale když se používání počítačů rozšířilo, ASCII se stala nedostatečnou
  a~bylo nutné vyvinout {\em alternativní} tabulky, které budou
  zahrnovat znaky neznámé anglické abecedě, jako je španělština
  \quote{ñ}, samohlásky s~diakritikou, katalánština nebo francouzština \quote{ç} atd. Na
  na druhé straně neexistovala žádná počáteční shoda ohledně toho, co tyto 
  {\em alternativní tabulky} ASCII by měly být, takže jiný specializované
  počítačové firmy se s~problémem postupně vypořádaly samy. Proto nejen
  byly vytvořeny specifické tabulky pro různé jazyky nebo skupiny
  jazyků, ale i~různé tabulky podle firmy, které je
  vytvořili (Microsoft, Apple, IBM atd.).

  Bylo to pouze s~nárůstem paměti počítače a~nižšími náklady
  úložných zařízení a~tomu odpovídajícímu zvýšení kapacity, že myšlenka
  vytvoření jedné tabulky, která by mohla být použita pro všechny jazyky.
  Ale znovu, ve skutečnosti to nebyla jediná tabulka obsahující všechny
  znaky, které byly vytvořeny, ale standardní kódování (nazývané 
  {\sc Unicode}) spolu s~různými způsoby jeho znázornění (UTF-8, UTF-16,
  UTF-32 atd.) Ze všech těchto systémů se nakonec stal
  de facto standardem jen UTF-8, což umožňuje reprezentovat
  prakticky jakýkoli živý jazyk a~mnoho již zaniklých jazyků,
  stejně jako další čtenné  symboly, všechny používající čísla proměnné délky
  (mezi 1 a~4 byty), což zase pomáhá optimalizovat velikost
  textových souborů. Tato velikost se nezvětšila {\em příliš} ve srovnání se soubory
  pomocí čistého ASCII. 

  Než se objevil \XeTeX\, systémy založené na \TeX{}u\ -- který se také zrodil
  v~USA, a~proto má angličtinu jako svůj rodný jazyk -- předpokládá se, 
  že kódování bylo v~čistém ASCII; takže pro použití jiného kódování jste to měli
  nějak indikovat ve zdrojovém souboru. 

\stopSmallPrint

\dontleavehmode\ConTeXt\ Mark~IV předpokládá, že kódování bude UTF-8.
V~méně aktuálních počítačových systémech však může být jiné kódování
stále používané ve výchozím nastavení. Nejsem si příliš jistý výchozím kódováním
které Windows používá, vzhledem k~tomu, že strategie Microsoftu pro oslovování
širší veřejnosti spočívá ve skrytí složitosti (ale i~když je skrytá, je
neznameno, že by zmizela!). K~dispozici není mnoho informací
(nebo jsem je nemohl najít) ohledně kódovacího systému, který používá
ve výchozím stavu.

V~každém případě, bez ohledu na výchozí kódování, jakýkoli textový editor vám umožní
uložit soubor v~požadovaném kódování. Zdrojové soubory které mají být
zpracované \ConTeXt\ Mark IV musí být uloženy v~UTF-8, pokud ovšem
existuje velmi dobrý důvod pro použití jiného kódování (i~když já si
nedokážu představit, jaký by mohl být tento důvod).

Pokud chceme zapsat soubor napsaný v~jiném kódování (možná starém
soubor) můžeme

\startitemize[a]

\item Převeďte soubor na UTF-8, doporučené možnosti, a~existující
různé nástroje k~tomu; v~Linuxu například příkazy 
{\tt iconv} nebo {\tt recode}.

\item Řekněte \ConTeXt{}u\ ve zdrojovém souboru, že kódování není UTF-8. Na
to musíme použít příkaz \tex{enableregime}, jehož syntaxe
je:

\PlaceMacro{enableregime}{\tt \color[hlavní barva]{\backslash enableregime[{\em Encoding}]}}

kde {\em Encoding} odkazuje na jméno, pod kterým \ConTeXt\ zná skutečné
kódování daného souboru. V~\in{table}[kódování] musíte
najít různá kódování a~názvy, pod kterými je \ConTeXt\ zná.

\stopitemize

{\switchtobodyfont[small]
  \placetable
    [here]
    [encodings]
    {Main encodings in \ConTeXt}
    {
      \starttabulate[|l|l|l|]
        \HL
        \NC {\bf Encoding} \NC {\bf Name(s) in \ConTeXt} \NC{\bf Notes}\NR
        \HL
        \NC Windows CP 1250\NC cp1250, windows-1250\NC Western Europe\NR
        \NC Windows CP 1251\NC cp1251, windows-1251\NC Cyrillic\NR
        \NC Windows CP 1252\NC cp1252, win, windows-1252\NC Western Europe\NR
        \NC Windows CP 1253\NC cp1253, windows-1253\NC Greek\NR
        \NC Windows CP 1254\NC cp1254, windows-1254\NC Turkish\NR
        \NC Windows CP 1257\NC cp1257, windows-1257\NC Baltic\NR
        \NC ISO-8859-1, ISO Latin 1\NC iso-8859-1, latin1, il1\NC Western Europe\NR
        \NC ISO-8859-2, ISO Latin 2\NC iso-8859-2, latin2, il2\NC Western Europe\NR
        \NC ISO-8859-15, ISO Latin 9\NC iso-8859-15,  latin9,  il9\NC Western Europe\NR
        \NC ISO-8859-7\NC iso-8859-7,  grk\NC Greek\NR
        \NC Mac Roman\NC mac\NC Western Europe\NR
        \NC IBM PC DOS\NC ibm\NC Western Europe\NR
        \NC UTF-8\NC utf\NC Unicode\NR
        \NC VISCII\NC vis,  viscii\NC Vietnamese\NR
        \NC DOS CP 866\NC cp866, cp866nav\NC Cyrillic\NR
        \NC KOI8\NC koi8-r, koi8-u, koi8-ru\NC Cyrillic\NR
        \NC Mac Cyrillic\NC maccyr, macukr\NC Cyrillic\NR
        \NC Others\NC cp855, cp866av, cp866mav, cp866tat, \NC Various\NR
        \NC \NC ctt, dbk, iso88595, isoir111, mik, mls, \NC\NR
        \NC \NC mnk, mos, ncc\NC\NR
        \HL
      \stoptabulate
    }
}

\ConTeXt\ MkIV důrazně doporučuje použití UTF-8. Souhlasím s~tímto
doporučením. Od tohoto okamžiku v~tomto úvodu můžeme předpokládat, že
kódování je vždy UTF-8.    

\startSmallPrint

   Spolu s~\tex{enableregime} \ConTeXt\ obsahuje příkaz
   \PlaceMacro{useregime}\tex{useregime} což nám umožňuje používat kód
   jednoho nebo jiného kódování jako argument. Nenašel jsem žádné informace
o~tento příkazu ani to, jak se liší od \tex{enableregime}, pouze některé
   příklady jeho použití \Conjecture. Mám podezření, že \tex{useregime} je
   navržen pro složité projekty, které používají mnoho zdrojových souborů,
s~očekáváním, že ne všechny budou mít stejné kódování. Ale je to jen odhad.

\stopSmallPrint

\stopsection

\startsection
  [title=Znaky ve zdrojovém souboru (souborech), se kterými \ConTeXt\ zachází zvláštním způsobem] 

{\em Speciální znaky} je označení, které dám skupině znaků,
které se liší od {\em vyhrazených znaků}. Jak je vidět
v~\in{kapitole}[sec:rezervované znaky] jsou takové, které mají speciální
význam pro \ConTeXt\ a~nelze je tedy přímo použít jako znaky
v~zdrojových souborech. Spolu s~nimi existuje další skupina znaků,
ačkoliv s~nimi \ConTeXt\ zachází, když je najde ve zdroji,
podle zvláštních pravidel. Tato skupina obsahuje prázdné
mezery (bílá místa), tabulátory, zalomení řádků a~pomlčky.

\startsubsection
  [title=Prázdné mezery (prázdné místo) a~tabulátory, reference=sec:spaces]

S~tabulátory a~mezerami se ve zdrojovém souboru pro všechny zachází se
stejnými záměry a~účely. Znak tabulátoru (klávesa Tab na klávesnici) bude
transformován na bílé místo pomocí \ConTeXt{}u. A~prázdná místa jsou pohlcena
do jakéhokoli jiného prázdného místa (nebo karty) bezprostředně za nimi. Tedy to
není absolutně žádný rozdíl ve zdrojovém souboru k~zápisu

\type{Tweedledum a~Tweedledee.}

nebo

\type{Tweedledum a~Tweedledee.} 

\ConTeXt{} považuje tyto dvě věty za naprosto stejné. Proto,
pokud chceme vložit další mezeru mezi slova,
je potřeba použít nějaké \ConTeXt{} příkazy, které to dělají. Normálně to bude fungovat
s~\quotation{\cmd{\textvisiblespace}}, což znamená {\tt\backslash}
znak následovaný mezerou. Existují ale i~jiné postupy
na které se podíváme v~\in{kapitole}[sec:horizontální prostor1] týkající se
horizontálních prostorů.

Absorpce po sobě jdoucích prázdných míst nám umožňuje napsat zdrojový
soubor vizuálně zvýrazněním částím, které bychom chtěli zvýraznit, jednoduše tím
zvýšíme nebo snížíme použité odsazení, s~klidem na duši
s~vědomím, že to žádným způsobem neovlivní konečný dokument. Tady
v~následujícím příkladu

\starttyping

Hudební skupina z~Madridu na konci sedmdesátých let {\em La Romántica
   Banda Local} napsala písně eklektického stylu, které bylo velmi obtížné
kategorizovat. O~svém synovi „El Egipcio“ například řekli:
\quotation{{\em Esto es una farsa más que una comedia, página muy seria
   de la histeria musical; sueños de princesa, vicios de gitano pueden en
   su mano acariciar la verdad}}, míchání slov, frází prostě,
mají vnitřní rytmus (comedia-histeria-seria, gitano-mano).

\stoptyping

můžete vidět, jak jsou některé řádky mírně odsazeny doprava. Tyto
řádky, které jsou součástí bitů, se zobrazí kurzívou. Mít tyto
odsazení pomáhá (autorovi) vidět, kde končí kurzíva.

\startSmallPrint

   Někdo by si mohl myslet, jaký nepořádek! Musím se obtěžovat s~odsazením řádků?
   Pravdou je, že toto speciální odsazení se u~mého textu děje automaticky
   pomocí editoru (GNU Emacs), když upravuje zdrojový soubor \ConTeXt{}u\. Tohle je
   taková malá nápověda, díky které se rozhodnete pracovat s~tímto určitým
   textovým editorem a~ne s~jinými.

\stopSmallPrint

Pravidlo, že prázdná místa jsou absorbována, platí výhradně pro po sobě jdoucí
prázdná místa ve zdrojovém souboru. Pokud tedy prázdná skupina
(\MyKey{\{\}}), je umístěna ve zdrojovém souboru mezi dvě prázdná místa,
ačkoli prázdná skupina ve výsledném souboru nic nevytvoří, jeho
přítomnost zajistí, že tyto dva polotovary nebudou po sobě jdoucí. Například,
pokud napíšeme \MyKey{Tweedledum \{\} a~Tweedledee}, dostaneme
\quotation{\color[red]{Tweedledum {} a~Tweedledee}}, kde, když se podíváte
dostatečně blízko, uvidíte dvě po sobě jdoucí mezery mezi prvními dvěma
slovy.

Totéž se však stane s~vyhrazeným znakem \MyKey{\lettertilde},
jeho efektem je generování prázdného místa, i~když tady ve skutečnosti není:
mezera následovaná znakem \lettertilde\ toto nepohltí
a~prázdné místo za \lettertilde\ také nebude absorbováno.

\stopsubsection

\startsubsection
  [reference=sec:linebreaks, title=Konce řádků]

Ve většině textových editorů, když řádek překročí maximální šířku, dojde
automaticky k~zalomení řádku. Můžeme také výslovně vložit zalomení řádku
stisknutím klávesy \quotation{Enter} nebo \quotation{Return}.

\ConTeXt{} aplikuje na zalomení řádků následující pravidla:

\startitemize[a, broad]

\item Zalomení jednoho řádku se ve všech směrech rovná prázdnému místu. 
Pokud tedy bezprostředně před nebo po zalomení řádku existuje
jakékoli prázdné místo nebo tabulátor, budou pohlceny zalomením řádku nebo
první prázdné místo a~v~konečném dokumentu bude jednoduché prázdné místo
vloženo.

\item Dva nebo více po sobě jdoucích zalomení řádku vytvoří zalomení odstavce. Pro
tyto dva konce řádků jsou považovány za po sobě jdoucí, pokud tam není
mezera nebo tabulátor mezi prvním a~druhým zalomením řádku (protože
ty jsou pohlceny zalomením prvního řádku); což ve zkratce znamená
jeden nebo více po sobě jdoucích řádků, které jsou ve zdrojovém souboru zcela prázdné
(bez jakéhokoli znaku nebo pouze s~prázdnými mezerami nebo tabulátory) se stanou
koncem odstavce.

\stopitemize

Všimněte si, že jsem řekl \quotation{dva nebo více po sobě jdoucích zalomení řádku} a~pak
\quotation{jeden nebo více prázdných po sobě jdoucích řádků}, což znamená, že pokud chceme
zvětšit separaci mezi odstavci, neděláme tak jednoduše
vložení dalšího konce řádku. K~tomu musíme použít příkaz který
zvětšuje vertikální prostor. Pokud chceme jen jednu linii oddělení navíc,
můžeme použít příkaz \PlaceMacro{blank}\tex{blank}. Ale jsou i~další
postupy pro zvětšení vertikálního prostoru. odkazuji na
\in{section}[sec:verticalspace].

\startSmallPrint

   V~některých případech, když se konec řádku stane prázdným místem, můžeme skončit
   s~nějakým nežádoucím a~neočekávaným bílým místem. Zvlášť když jsme
   napsali makro, kde je snadné pro prázdné místo \quotation{se propašovat
   v} aniž bychom si to uvědomovali. Abychom tomu zabránili, můžeme použít
   vyhrazený znak \MyKey{\%}, který, jak víme, způsobí že řádek, kde se objeví
   nezpracovat, znamená, že konce řádku také nebudou zpracovány. Tedy například příkaz

\vbox{
\starttyping
\define[3]\Test{
  {\em #1}
  {\bf #2}
  {\sc #3}
}
\stoptyping
}

který píše svůj první argument kurzívou, druhý tučně a~třetí
malými písmeny, vloží mezi každý z~těchto argumentů mezeru,
zatímco

\starttyping
\define[3]\Test{%
  {\em #1}%
  {\bf #2}%
  {\sc #3}%
}
\stoptyping

nevloží mezi ně žádné prázdné místo, protože vyhrazený znak
\% zabrání zpracování zalomení řádků a~stane se pouze prázdným místem.

\stopSmallPrint

\stopsubsection

\startsubsection
  [reference=sec:dashes, title=Pravidla/pomlčky]

Pomlčky jsou dobrým příkladem rozdílu mezi počítačovou klávesnicí
a~tištěný text. Na normální klávesnici je obvykle pouze jeden znak
pomlčky (nebo pravidlo, typograficky), kterou nazýváme pomlčka nebo
(\MyKey{-}); ale tištěný text používá až čtyři různé délky
pravidla:

\startitemize[1,broad]

\item Krátká pravidla (pomlčky), jako jsou ta, která se používají k~oddělení slabik
dělení slov na konci řádku (-).

\item Středně velká pravidla (en pomlčky nebo en pravidla), o~něco delší než
předchozí (--). Mají řadu použití, včetně některých Evropských
jazyků (méně v~angličtině) začátek řádku dialogu, nebo pro
oddělení menších od větších číslic v~rozsahu dat nebo stránek;
\quotation{pp. 12--33}.

\item Dlouhá pravidla (em pomlčky nebo em pravidla) (---), používané jako závorky
zahrnout jednu větu do druhé.

\item Znaménko mínus ($-$) představuje odečítání nebo záporné číslo.

\stopitemize

Dnes jsou všechny výše uvedené a~další kromě toho dostupné v~kódování UTF-8.
Ale protože je nelze všechny vygenerovat jedinou klávesou na klávesnici,
není je tak snadné vytvořit ve zdrojovém souboru. Naštěstí \TeX\ vidí
do našeho konečného dokumentu a~musí zahrnout více pravidel/pomlček, než by mohlo být
vytvořené klávesnicí a~jednoduchý postup k~tomu navržený.
\ConTeXt{} doplnil tento postup také přidáním příkazů, které
vytváří tyto různé druhy pravidel. Můžeme použít dva přístupy
generování čtyř druhů pravidel: buď běžným způsobem \ConTeXt{}
příkazem nebo přímo z~klávesnice. Tyto postupy jsou uvedeny
v~\in{table}[tbl:rules]:

{\switchtobodyfont[small]
\placetable
  [here]
  [tbl:rules]
  {Rules/dashes in \ConTeXt}
  {\starttabulate[|l|c|c|l|]
    \HL
    \NC {\bf Type of rule}
    \NC {\bf Appearance}
    \NC {\bf Written directly}
    \NC {\bf Command}
    \NR
    \HL
    \NC Hyphen
    \NC -
    \NC {\tt -}
    \NC \PlaceMacro{hyphen}\tex{hyphen}
    \NR
    \NC En rule
    \NC --
    \NC {\tt --}
    \NC \PlaceMacro{endash}\tex{endash}
    \NR
    \NC Em rule
    \NC ---
    \NC {\tt ---}
    \NC \PlaceMacro{emdash}\tex{emdash}
    \NR
    \NC Minus sign
    \NC $-$
    \NC {\tt \$-\$}
    \NC \PlaceMacro{minus}\tex{minus}
    \NR
    \HL
  \stoptabulate}
}

Názvy příkazů \tex{hyphen} a~\tex{minus} jsou normálně používané
v~angličtině. Zatímco mnozí v~polygrafickém průmyslu jim říkají \quote{pravidla},
Termíny \TeX{}u, jmenovitě \tex{endash} a~\tex{emdash} jsou také běžné
v~sázecí terminologie. \quotation{{\em en}} a~\quotation{{\em em}}
jsou názvy měrných jednotek používaných v~typografii. \quotation{cs}
představuje šířku \quote{n}, zatímco \quote{em} je šířka
\quote{m} v~používaném písmu.

\stopsection

\startsection
  [title=Jednoduché a~vícesouborové projekty]

V~\ConTeXt{}u\ můžeme použít pouze jeden zdrojový soubor, který obsahuje úplně
všechen obsah našeho konečného dokumentu, jakož i~všechny související podrobnosti
v~tom případě mluvíme o~\quotation{jednoduchých projektech}, nebo, bychom
oproti tomu mohli použít řadu zdrojových souborů, které sdílejí obsah
našeho závěrečného dokumentu, o~kterém v~tomto případě mluvíme jako
o~\quotation{vícenásobních projektech}.

Scénáře, kdy je typické pracovat s~více než jedním zdrojovým souborem
jsou následující:

\startitemize

\item Pokud píšeme dokument, který má několik spolupracujících autorů, 
každý se svou částí odlišnou od ostatních;
například, pokud píšeme festschrift s~příspěvky od různých
autorů, nebo číslo časopisu atd.

\item Pokud píšeme dlouhý dokument, kde má každá část (kapitola)
relativní autonomii, tak jejich konečné uspořádání umožňuje
několik možností a~bude rozhodnuto na konci. K~tomu dochází
s~relativní četností pro mnoho akademických textů (příručky, úvody
a~podobné), kde se pořadí kapitol může lišit.

\item Pokud píšeme řadu souvisejících dokumentů, které sdílejí nějaké
stylové vlastnosti.

\item Pokud, zjednodušeně řečeno, dokument, na kterém pracujeme, je velký, např
počítač se zpomalí buď při jeho editaci nebo kompilaci; v~tomto případě,
rozdělení materiálu do několika zdrojových souborů značně urychlí kompilaci pro každou z~nich.

\item Také, pokud jsme napsali řadu maker, která chceme použít
některé (nebo všechny) na naše dokumenty, nebo pokud jsme vygenerovali šablonu, která
ovládá nebo upravuje naše dokumenty a~my je na ně chceme aplikovat atd.

\stopitemize

\stopsection

\startsection
  [title=Struktura zdrojového souboru v~jednoduchých projektech,
  reference=sec:structure]

V~jednoduchých projektech vyvinutých v~jediném zdrojovém souboru je struktura velmi
jednoduchá a~točí se kolem prostředí \MyKey{textu}, který se musí
v~podstatě objevit ve stejném souboru. Rozlišujeme následující
části tohoto souboru:

\startitemize

\item {\bf Preambule dokumentu}: vše od prvního řádku
v~souboru až na začátek prostředí \MyKey{textu}
(\PlaceMacro{starttext}\PlaceMacro{stoptext}\tex{starttext}).

\item {\bf Tělo dokumentu}: toto je
prostředí dokumentu \MyKey{text}; nebo jinými slovy všechno mezi
\tex{starttext} a~\tex{stoptext}.

\stopitemize

\placefigure
   [here]
   [img:ProyectoSimple]
   {\tfx soubor obsahující jednoduchý projekt}
{\startframedtext
\starttyping
% První řádek dokumentu

% oblasti preambule:
% Obsahuje globální konfiguraci
% příkazy pro dokument

\starttext % Zde začíná tělo dokumentu

...
... % Obsah dokumentu
...

\stoptext % Konec dokumentu

\stoptyping
\stopframedtext}

V~\in{figure}[img:ProyectoSimple] vidíme velmi jednoduchý zdrojový soubor.
Absolutně vše před příkazem \tex{starttext} (který je
v~obrázku na řádku 5, počítají se pouze ty s~nějakým textem), tvoří
preambule; vše mezi \tex{starttext} a~\tex{stoptext} tvoří
tělo dokumentu. Cokoli po stoptextu bude ignorováno.

{\bf Preambule} se používá k~zahrnutí příkazů, které mají ovlivnit
dokument jako celek, tedy ty, které určují jeho celkovou konfiguraci. Tady
není podstatné psát žádný příkaz do preambule. Pokud žádný není,
\ConTeXt\ převezme výchozí konfiguraci, která není příliš vyvinutá,
ale může to udělat pro mnoho dokumentů. V~dobře naplánovaném dokumentu preambule
bude obsahovat všechny příkazy ovlivňující dokument jako celek, např
makra a~přizpůsobené příkazy pro použití ve zdrojovém souboru. V~typické
preambule by to mohlo zahrnovat následující:

\startitemize[zabaleno]

\item Označení hlavního jazyka dokumentu (viz
   \in{section}[sec:langdoc]).

\item Označení velikosti papíru
   (\in{section}[sec:papersize]) a~rozložení stránky
   (\in{section}[sec:pagelayout]).

\item Vlastnosti hlavního písma dokumentů
   (\in{section}[sec:mainfont]).

\item Přizpůsobení příkazů sekce, které mají být použity
   (\in{section}[sec:setuphead]) a~v~případě potřeby definice
   nové příkazy sekce (\in{section}[sec:definehead]).

\item Rozložení záhlaví a~zápatí
   (\in{section}[sec:headerfooter]).

\item Nastavení pro naše vlastní makra
   (\in{section}[sec:definingcommands]).

\item atd.

\stopitemize

Preambule je určena pro celkovou konfiguraci dokumentu;
tedy nic, co má společného s~{\em obsahem} dokumentu, popř
měl by tam být zpracovatelný text. Teoreticky zahrnuje jakýkoli zpracovatelný text
v~preambuli bude ignorován, i~když někdy, pokud tam je,
způsobí chybu při kompilaci.

{\bf Tělo dokumentu}, zarámované mezi \tex{starttext}
a~příkazy \tex{stoptext} zahrnují skutečný obsah, což znamená zpracovatelný
text spolu s~příkazy \ConTeXt{}u\, které by neměly ovlivnit celek
dokumentu.

\stopsection

\startsection
  [title=Správa více souborů ve stylu \TeX{}u]

Aby bylo možné pracovat s~více než jedním zdrojovým souborem, \TeX\ obsahoval
prvotní soubor nazvaný \tex{input}, která také funguje v~\ConTeXt{}u, ačkoliv
později obsahuje dva specifické příkazy, které do jisté míry zdokonalují způsob
funkce \tex{input}.

\stopsection

\startsubsection[title={Příkaz input}, reference={input},]
\PlaceMacro{input}

Příkaz \tex{input} vloží obsah souboru, který označuje. Jeho
formát je:

\type{\input Název souboru}

kde {\em FileName} je název souboru, který se má vložit. Všimněte si, že není
nutné, aby byl název souboru uzavřen ve složených závorkách,
i~když to nevyvolá chybu, pokud se to udělá. Nemělo by však nikdy
umístit do hranatých závorek. Pokud je přípona souboru
\quotation{\type{.tex}}, lze jej vynechat.

Když \ConTeXt\ kompiluje dokument a~najde příkaz \tex{input},
vyhledá to uvedený soubor a~pokračuje v~kompilaci, jako by tento soubor byl
část souboru, který jej nazval. Po dokončení kompilace se vrátí
do původního souboru a~pokračuje od místa, kde skončil; praktickým
výsledkem je tedy, že obsah souboru volaný pomocí
\tex{input} se vloží do bodu, kde se to volá. Soubor nazvaný
s~\tex{input} musí mít platný název v~našem operačním systému a~ne
prázdná místa v~názvu. \ConTeXt\ to bude hledat v~pracovním
adresáři, a~pokud jej tam nenajde, vyhledá jej
v~adresáři obsaženém v~proměnné prostředí TEXROOT. Pokud
soubor nebyl nakonec nalezen, způsobí chybu kompilace.

Nejběžnější použití příkazu \tex{input} je následující: soubor je
nazvěme to \MyKey{principal.tex} a~toto bude použito jako
kontejner pro volání různých souborů pomocí příkazu \tex{input}
které tvoří náš projekt. To je znázorněno na následujícím příkladu:

\startframedtext\switchtobodyfont[small]
\starttyping
% Obecné konfigurační příkazy:

  \input MyConfiguration

\starttext

  \input PageTitle
  \input Preface
  \input Chap1
  \input Chap2
  \input Chap3

  ...

\stoptext
\stoptyping
\stopframedtext

Všimněte si, jak jsme pro obecnou konfiguraci dokumentu nazvali
soubor \quotation{MyConfiguration.tex}, o~kterém předpokládáme, že obsahuje globální
příkazy, které chceme použít. Poté mezi příkazy \tex{starttext}
a~\tex{stoptext} nazýváme několik souborů, které obsahují obsah souboru
různé části našeho dokumentu. Pokud v~danou chvíli pro urychlení
kompilační proces, chceme vynechat kompilaci některých souborů, vše, co potřebujeme
udělat, je umístit značku komentáře na začátek řádku volajícího nebo
ty soubory. Například když píšeme třetí kapitolu a~chceme
zkompilovat jej jednoduše, abychom zkontrolovali, že v~něm nejsou žádné chyby, nepotřebujeme
zkompilovat zbytek, a~proto můžeme psát:

\startframedtext\switchtobodyfont[small]
\starttyping
% Obecné konfigurační příkazy:

  \input MyConfiguration

\starttext

  % \input PageTitle
  % \input Preface
  % \input Chap1
  % \input Chap2

  \input Chap3

  ...

\stoptext
\stoptyping
\stopframedtext

a~bude zkompilována pouze kapitola 3. Všimněte si, jak se naopak mění
pořadí kapitol je stejně jednoduché jako změna volání pořadí řádků.

\startSmallPrint

   Když vyloučíme soubor ve vícesouborovém projektu z~kompilace,
   získáme na rychlosti zpracování, ale v~důsledku toho všechny reference, které
   část, která je zkompilována, převede na jiné části, které ještě nebyly zkompilovány
   již nebude fungovat. Viz \in{section}[sec:reference].

\stopSmallPrint

Je důležité, aby bylo jasné, že když pracujeme pouze s~\tex{input}
hlavního souboru, ten, který volá všechny ostatní, musí obsahovat
\tex{starttext} a~\tex{stoptext} příkazy, protože pokud ostatní soubory
jej zahrnují, dojde k~chybě. To na druhou stranu znamená, že
nemůžeme přímo kompilovat různé soubory, které tvoří projekt,
ale musí je nutně zkompilovat z~hlavního souboru, což je ten, který
obsahuje základní strukturu dokumentu.

\stopsubsection

\startsubsection
  [title=\tex{ReadFile} a~\tex{readfile}]
\PlaceMacro{ReadFile}\PlaceMacro{readfile}

Jak jsme právě viděli, pokud \ConTeXt\ nenalezne soubor volaný
s~\tex{input}, vygeneruje chybu. Pro situaci, kdy chceme
vložit soubor, pouze pokud existuje, ale s~možností, že existovat
nemusí, \ConTeXt\ nabízí variantu příkazu \tex{input}. To je

\type{\ReadFile{FileName}}

Tento příkaz je ve všech ohledech podobný \tex{input} s~jedinou
výjimkou, že pokud soubor, který se má vložit, není nalezen, bude pokračovat
kompilace bez generování jakékoli chyby. Také se liší od
\tex{input} ve své syntaxi, protože víme, že s~\tex{input} tomu není
nutné vložit název souboru, který má být vložen mezi složené
závorky. Ale s~\tex{ReadFile} to je nutné. Pokud nepoužijeme složené
závorky, \ConTeXt\ si bude myslet, že název hledaného souboru je
stejný jako první znak, který následuje za příkazem \tex{ReadFile},
následuje přípona \type{.tex}. Pokud tedy například píšeme

\type{\ReadFile MyFile}

\ConTeXt\ pochopí, že soubor ke čtení je volán
\quotation{\type{M.tex}}, protože znak bezprostředně za příkazem
\tex{ReadFile} (s~výjimkou prázdných mezer, které jsou, jak víme, na
konci názvu příkazem ignorovány) je \quote{M}. Protože \ConTeXt\ nemůže normálně
najít soubor s~názvem \quotation{\type{M.tex}} a~\tex{ReadFile}
pokud soubor nenajde, vygeneruje chybu, \ConTeXt\ bude pokračovat
kompilace za \quote{M} v~\quotation{\type{MyFile}} a~vloží
text \quotation{\type{yFile}}.

Jemnější verze \tex{ReadFile} je \tex{readfile}, jejíž formát je stejný

\type{\readfile{Název souboru}{TextIfExists}{TextIfNotExists}}

První argument je podobný \tex{ReadFile}: název souboru
uzavřeno mezi složené závorky. Druhý argument obsahuje text který má 
být zapsán, pokud soubor existuje, před vložením obsahu souboru.
Třetí argument obsahuje text, který má být zapsán, pokud jde o~nenalezený soubor. 
To znamená, že v~závislosti na tom, zda byl či nebyl soubor zadán
jak je nalezen první argument, druhý argument (pokud soubor existuje) nebo
třetí (pokud soubor neexistuje) bude spuštěn.

\stopsubsection

\stopsection 

\startsection
   [title=\ConTeXt\ projekty jako takové,
   reference=sec-projects]

Třetí mechanismus, který \ConTeXt\ nabízí pro vícesouborové projekty, je více
komplexní a~kompletní: začíná rozlišováním mezi soubory projektu,
soubory produktu, soubory součástí a~soubory prostředí. Pro pochopení
vztahy a~fungování každého z~těchto typů souborů, myslím, že ano
nejlépe je vysvětlit každý zvlášť:

\startsubsection
   [reference=environments, title={{\em Environment} soubory}]
\PlaceMacro{startenvironment}\PlaceMacro{environment}

Soubor prostředí je soubor, který ukládá makra a~konfigurace
specifického styl, který má být aplikován na několik dokumentů,
zda se jedná o~zcela nezávislé dokumenty nebo součásti komplexu
dokumentů. Soubor prostředí tedy může obsahovat vše, co bychom chtěli
normálně napsat před \tex{starttext}; to je: obecná konfigurace
dokumentu.

\startSmallPrint

Pro tyto druhy jsem zachoval termín \quotation{soubory prostředí}, 
abychom se neodchýlili od oficiální terminologie \ConTeXt{}u\, dokonce
i~když věřím, že lepší termín by byl pravděpodobně \quotation{format
soubory} nebo \quotation{globální konfigurační soubory}.

\stopSmallPrint

Stejně jako všechny zdrojové soubory \ConTeXt{}u\ jsou soubory prostředí textové soubory
a~předpokládejme, že rozšíření bude \quotation{\type{.tex}}, i~když pokud bychom
chtěli můžeme to změnit, třeba na \quotation{\type{.env}}. Obvykle to tak
není však provedeno v~\ConTeXt{}u\. Nejčastěji je soubor prostředí
identifikován tak, že název začíná nebo končí \quote{env}. Pro
příklad:\quotation{\type{MyMacros_env.tex}} nebo
\quotation{\type{env_MyMacros.tex}}. Uvnitř takového souboru prostředí
by vypadalo nějak takto:

\startframedtext\switchtobodyfont[malé]
\starttyping

\startenvironment MyEnvironment

  \mainlanguage[en]

  \setupbodyfont
    [modern]

  \setupwhitespace
    [big]

  ...

\stopenvironment

\stoptyping
\stopframedtext

Nebo jinými slovy, definice a~konfigurační příkazy jsou součástí
\tex{startenvironment} a~\tex{stopenvironment}. Okamžitě následuje
\tex{startenvironment} napíšeme jméno, kterým chceme identifikovat
prostředí a~poté zahrneme všechny příkazy, které bychom chtěli
v~našem prostředí, ze kterého se má skládat.

\startSmallPrint

S~ohledem na název prostředí podle mých testů název
přidaný bezprostředně za \tex{startenvironment} je pouze orientační
a~kdybychom to nepojmenovali, pak se nic (špatného) nestane.

\stopSmallPrint

Soubory prostředí byly určeny pro práci s~komponentami a~produkty
(vysvětleno v~další části). To je důvod, proč může jedno nebo více prostředí
být voláno z~komponenty nebo produktu pomocí \tex{prostředí}
příkazu. Tento příkaz ale také funguje, pokud je použit v~konfiguraci
oblast (preambule) libovolného zdrojového souboru \ConTeXt{}u\, i~když to není
zdrojový soubor určený k~sestavení po částech.

Příkaz \tex{environment} lze volat pomocí kteréhokoli z~těchto dvou
následující formátů:

\type{\environment soubor}

\type{\environment[Soubor]}

V~obou případech bude účinek tohoto příkazu spočívat v~načtení obsahu
souboru brán jako argument. Pokud tento soubor není nalezen, bude pokračovat
kompilace normálním způsobem bez generování jakékoli chyby. Pokud
přípona souboru je \quotation{\type{.tex}}, lze ji vynechat.

\stopsubsection 

\startsubsection
  [reference=components and products, title=Komponenty a~produkty]
\PlaceMacro{startcomponent}\PlaceMacro{startproduct}\PlaceMacro{product}

Pokud si představíme knihu, kde je každá kapitola v~jiném zdrojovém souboru,
pak bychom řekli, že kapitoly jsou {\em komponenty} a~kniha je
{\em produkt}. To znamená, že {\em komponenta} je autonomní
součást {\em produktu}, která může mít svůj vlastní styl a~může být sestavena
nezávisle. Každá komponenta bude mít jiný soubor a~navíc
bude soubor produktu, který spojí všechny součásti dohromady.

Typický soubor součásti by byl následující

\startframedtext\switchtobodyfont[small]
\starttyping

\environment MyEnvironment
\environment MyMacros

\startcomponent Chapter1

  \startchapter[title={Chapter 1}]

  ...

\stopcomponent

\stoptyping
\stopframedtext

A~soubor produktu by vypadal následovně: 

\startframedtext\switchtobodyfont[small]
\starttyping

\environment MyEnvironment
\environment MyMacros

\startproduct MyBook

  \component Chapter1
  \component Chapter2
  \component Chapter3

  ...

\stopproduct

\stoptyping
\stopframedtext

Upozorňujeme, že skutečný obsah našeho dokumentu bude distribuován mezi
různé \quote{component} soubory a~soubor produktu je omezen na
stanovení pořadí komponenty. Na druhou stranu,
(jednotlivé) komponenty a~produkty lze sestavit přímo.
Kompilace produktu vygeneruje soubor PDF obsahující všechny komponenty
tohoto produktu. Pokud je naopak jedna z~komponent sestavena
jednotlivě vygeneruje soubor pdf obsahující pouze zkompilované
komponenty.

V~souboru komponentů a~před příkazem \tex{startcomponent} se
může volat jeden nebo více souborů prostředí pomocí \tex{environment
Název prostředí}. Totéž můžeme udělat dříve v~souboru produktu
\tex{startproduct}. Několik souborů prostředí lze načíst současně.
Můžeme mít například naši oblíbenou sbírku maker
a~různé styly, které aplikujeme na naše dokumenty, všechny v~různých souborech. Prosím
všimněte si však, že když používáme dvě nebo více prostředí, ta se načítají
v~pořadí, ve kterém jsou volány, takže pokud je stejná konfigurace,
příkaz byl zahrnut ve více než jednom prostředí a~také ano
jiné hodnoty, budou platit hodnoty naposledy načteného prostředí. Na
na druhou stranu jsou soubory prostředí načteny pouze jednou, takže
v~předchozích příkladech, ve kterých je prostředí voláno ze souboru produktu
a~konkrétní dílčí soubory, pokud zkompilujeme produkt, to je čas, kdy
prostředí se načtou a~v~uvedeném pořadí; když
prostředí je voláno z~kterékoli komponenty, \ConTeXt\ zkontroluje, zda
toto prostředí je již načteno, v~takovém případě neudělá nic.


Název komponenty, která je volána z~produktu, musí být názvem souboru, 
který obsahuje danou komponentu, ačkoli pokud je přípona tohoto souboru \quotation{\type{.tex}}, 
lze ji vynechat.

\stopsubsection

\startsubsection
  [title=Projekty jako takové]
\PlaceMacro{startproject}\PlaceMacro{project}

Rozlišení mezi produkty a~komponenty je ve většině případů dostatečné. 
Stejně tak \ConTeXt\ má ještě vyšší úroveň, 
kde můžeme seskupit řadu produktů: toto je {\em project}.

Typický soubor projektu by vypadal víceméně následovně

\startframedtext\switchtobodyfont[small]
\starttyping

\startproject MyCollection

  \environment MyEnvironment
  \environment MyMacros

  \product Book01
  \product Book02
  \product Book03

  ...

\stopproject

\stoptyping
\stopframedtext

Scénář, kdy bychom potřebovali projekt, by byl například kde
potřebujeme upravit sbírku knih, všechny ve stejném formátu
specifikace; nebo kdybychom editovali časopis: sbírku knih,
nebo časopis jako takový by byl projekt; každá kniha nebo každý deník,
problém by byl produkt; a~každá kapitola knihy nebo každý článek
v~deníku by byl komponentou.

Na druhou stranu projekty nejsou určeny k~přímému sestavování.
Vezměte v~úvahu, že podle definice každý produkt patřící do projektu (každý
kniha ve sborníku nebo každé číslo časopisu) může být kompilován 
samostatně a~vygenerovat si vlastní PDF. Proto příkaz \tex{product}
je zahrnut v~něm, aby bylo uvedeno, jaké produkty skutečně patří do
projektu, nedělá nic: je to pouze připomínka pro autora.

Je jasné, že by se někteří mohli ptát, proč máme projekty, když je nelze zkompilovat:
odpověď je, že soubor projektu propojuje určitá prostředí s~projektem.
To je důvod, proč, pokud zahrneme projekt \PlaceMacro{project}\tex{projekt
ProjectName} v~souboru komponenty nebo produktu, \ConTeXt\ přečte
soubor projektu a~automaticky načte prostředí s~ním spojená.
Proto musí následovat příkaz \tex{environment} v~projektech
\tex{startproject}; nicméně v~produktech a~komponentách \tex{prostředí}
musí přijít {\em dříve} \tex{startproduct} nebo \tex{startcomponent}

Stejně jako u~příkazů \tex{environment} a~\tex{component},
i~příkaz \tex{project} nám umožňuje zadat název projektu buď uvnitř
hranaté závorky nebo hranaté závorky nepoužívat vůbec. Tohle znamená tamto
\tex{project FileName} a~\tex{Project[FileName]} jsou ekvivalentní příkazy.

{\bf Souhrn různých způsobů načítání prostředí}

Z~výše uvedeného vyplývá, že prostředí může být načteno kterýmkoli
z~následující postupy:

\startitemize[a, board]

\item Před vložením příkazu \tex{environment EnvironmentName}
\tex{starttext} nebo \tex{startcomponent}. Tím se zatíží prostředí pro
pouze kompilace příslušného souboru.

\item Vložením příkazu \tex{environment EnvironmentName} do
souborů produktu před \tex{startproduct}. Tím dojde k~zatížení prostředí, když
produkt je zkompilován, ale ne, pokud jsou zkompilovány jeho součásti
jednotlivě.

\item Vložením příkazu \tex{project} do produktu nebo prostředí:
tím se načtou všechna prostředí spojená s~projektem (v~projektu
soubor).

\stopitemize

\stopsubsection

\startsubsection
  [title={Společné aspekty prostředí,\\ komponenty, produkty a~projekty}]

\startdescription{Názvy prostředí, komponent, produktů a~projektů:}

Již jsme viděli, že u~všech těchto prvků po \tex{start}
příkazu, který spouští určité prostředí, komponentu, produkt nebo
projekt, se jeho název zadává přímo. Toto jméno se zpravidla musí shodovat
s~názvem souboru obsahujícího prostředí, komponentu nebo produkt
protože například když \ConTeXt\ kompiluje produkt
a~do souboru produktu se musí načíst prostředí nebo komponenta, nemáme žádnou možnost
vědět, který soubor je toto prostředí nebo komponenta, pokud soubor nemá
stejný název jako prvek, který se má načíst.

Jinak podle mých testů název napsaný za \tex{startproduct}
nebo \tex{startenvironment} v~souborech produktu a~prostředí je pouze
orientační. Pokud je vynechán nebo se neshoduje s~názvem souboru,
nic zlého se neděje. V~případě komponentů je však důležité,
že název komponenty odpovídá názvu souboru, který to obsahuje.

\stopdescription

\description{Struktura adresářů projektu:}

Víme, že \ConTeXt\ standardně hledá soubory v~pracovním adresáři
a~v~cestě označené proměnnou TEXROOT. Když však použijeme
příkazy \tex{project}, \tex{product}, \tex{component} nebo \tex{environment}
předpokládá se, že projekt má adresářovou strukturu, ve které se společné
prvky nacházejí v~nadřazeném adresáři a~ty specifické v~některých
podřízený adresářích. Pokud tedy soubor uvedený v~pracovním adresáři není
nalezen, bude vyhledán v~nadřazeném adresáři a~pokud není
také tam,pak v~adresáři rodičů tohoto adresáře a~tak dále. 

\stopsubsection

\stopsection

\stopchapter

\stopcomponent