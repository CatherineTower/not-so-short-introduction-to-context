\startcomponent 01-03_Credit

\environment introCTX_env_00

%-------------------------------------------------------------------------------
                                                                 % Credit titles


  {
    \page[yes]
    \page[blank]\parindent0pt
    \strut\vfill

    \switchtobodyfont[small]
    {\bf Une courte (?) introduction à \ConTeXt} (\cap{MKIV} et \cap{LMTX})\\
    Version 2.0alpha [\RevisionDate]\\ 
    \blank

% I wanted  to use  the copyleft  symbol, but  I have  not found  out how  to
% generate  it. With  \mirror{\copyright} the  symbol  is generated,  but a
% rather annoying  linebreak is inserted after  it. Copy and paste  from the
% internet doesn't work and  I don't know why. (It is not  the first time this
% happens).

    {\copyright} 2020-2021, Joaquín Ataz-López and Garulfo\\
    \blank
    Titre original: Una introducción (no demasiado breve) a \ConTeXt\ Mark~IV\\
    Traduction française: Un bon ami qui souhaite rester anonyme.

    L'auteur du  présent texte, ainsi  que ses traducteurs anglais  et français,
    autorisent sa libre  distribution et utilisation, ce qui inclue  le droit de
    copier et de redistribuer ce document  sur support numérique à condition que
    l'auteur soit cité, et que le document  ne soit inclus ni dans un paquet, ni
    dans une  suite logicielle,  ni dans une  documentation dont  les conditions
    d'utilisation ou de distribution ne prévoient pas le droit de copie et de
    redistribution libre par  ses destinataires.  La traduction  du document est
    également autorisée,  à condition  que la paternité  du texte  original soit
    indiquée,  que son  statut  de traduction  soit indiquée,  et  que le  texte
    traduit soit distribué sous la licence  FDL de la Fondation pour le Logiciel
    Libre (Free Software Foundation), une  licence Creative Commons qui autorise
    la copie et la redistribution, ou autre licence similaire.

    Nonobstant  ce qui  précède, la  publication et  la commercialisation  de ce
    document, ou  sa traduction,  nécessitera l'autorisation écrite  expresse de
    l'auteur.

% It is not  a totally free licence since  it does not allow it to  be marketed n
% paper form. The point is to pevent some "smartass" from publishing this document
% as a  book. However,  apart from  this restriction, I  believe that  the licence
% continues to reflect  quite well the spirit of the  FDL licence, in considerably
% less space.

    \setupnotation[numberconversion=set 0]
    Historique des versions :

    \startitemize[packed]\smallbodyfont

      \item \date[y=2020, m=8, d=18]  : Version 1.0 (Uniquement en espagnol) :
      Document original.

      \item \date[y=2020, m=8, d=23]  : Version 1.1 (Uniquement en espagnol) :
      Correction de petites erreurs de frappe et de malentendus de l'auteur.

      \item \date[y=2020, m=9, d=3]   : Version 1.15 (Uniquement en espagnol) :
      Autres corrections de petites erreurs de frappe et de malentendus.

      \item \date[y=2020, m=9, d=5]   : Version 1.16 (Uniquement en espagnol) :
      Autres corrections  de petites erreurs  de frappe et de  malentendus ainsi
      que  des petites  modifications  qui  améliorent la  clarté  du texte  (je
      crois).

      \item \date[y=2020, m=9, d=6]   : Version 1.17 (Uniquement en espagnol) :
      C'est incroyable le  nombre de petites erreurs que je  trouve.  Si je veux
      m'arrêter, je dois arrêter de relire le document.

      \item \date[y=2020, m=10, d=21] : Version 1.5 (Uniquement en espagnol) :
      Introduction de  suggestions et correction  des erreurs signalées  par les
      utilisateurs de la liste de diffusion \from[ntg-context].

      \item \date[y=2021, m=7, d=4]   : Version 1.6 : Corrections suggérées après une
      nouvelle lecture du document, à l'occasion de sa traduction en anglais

      \item \RevisionDate: Version 2.0alpha : Réorganisation globale du contenu. 
      Attention : le contenu est encore en cours de révision et d'adaptation
      pour correspondre à cette nouvelle réorganisation.
      
      
    \stopitemize

  }


\stopcomponent

%%% Local Variables:
%%% mode: ConTeXt
%%% mode: auto-fill
%%% TeX-master: "../introCTX_fra.tex"
%%% coding: utf-8-unix
%%% End:
