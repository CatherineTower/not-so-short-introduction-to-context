\startcomponent 01-03_Preface

\environment introCTX_env_00

\setupnotation
  [footnote]
  [numberconversion=set 2]

\startchapter
  [title=Avant-propos%
  \footnote{Cet  avant-propos    a   commencé   avec   l'intention    d'être   une
    traduction|/|adaptation  à \ConTeXt\  de l'avant-propos de  \quotation{The \TeX
      Book}, le document qui explique {\em tout  ce que vous devez savoir sur le
      \TeX}.   En fin  de compte,  j'ai  dû m'en  écarter ;  cependant, j'en  ai
    conservé quelques éléments  qui, je l'espère, pour ceux  qui le connaissent,
    feront écho.},
  bookmark=Avant-propos]

\setupnotation
  [footnote]
  [numberconversion=n]

Chère  lectrice,  Cher lecteur,  voici  un  document concernant  \ConTeXt\,  un
système de composition de document dérivé  de \TeX, qui, lui même également, est
un système de  composition créé entre 1977  et 1982 par {\sc Donald  E. Knuth} à
l'Université de Stanford.

\ConTeXt\  a été  conçu pour  la  création de  documents de  très haute  qualité
typographique --  destinés à  être soit  imprimés sur  papier soit  affichés sur
écran informatique. Il  ne s'agit pas d'un traitement de  texte, ni d'un éditeur
de texte, mais, comme indiqué précédemment,  d'un {\em système}, ou encore d'une
{\em suite d'outils}, destinés à la  composition de documents, c'est-à-dire à la
mise en page  et à la visualisation  des différents éléments du  document sur la
page de papier ou à l'écran. En résumé, \ConTeXt\ vise à fournir tous les outils
nécessaires pour donner  aux documents la meilleure  apparence possible.  L'idée
est  de pouvoir  générer des  documents qui,  en plus  d'être bien  écrits, sont
également \quotation{beaux}.   À cet égard,  nous pouvons mentionner ici  ce que
{\sc Donald E.  Knuth}  a écrit lors de la présentation de  \TeX (le système sur
lequel est basé \ConTeXt\ )~:

\startnarrower
\it
Si vous voulez simplement produire un document passablement bon -- quelque chose
d'acceptable et d'essentiellement  lisible mais pas vraiment beau  -- un système
plus simple suffira généralement. Avec \TeX\  l'objectif est de produire la {\em
  meilleure} qualité~; cela nécessite  de porter plus d'attention  aux détails,
mais finalement  vous ne  trouverez pas  cela beaucoup  plus difficile,  et vous
pourrez être particulièrement fier du produit fini.
\stopnarrower

Lorsque  nous préparons  un document  avec \ConTeXt,  nous indiquons  exactement
comment celui-ci doit  être transformé en pages (ou en  écrans) avec une qualité
typographique et une  précision de composition comparable à celle  que l'on peut
obtenir grâce  aux meilleurs imprimeurs  du monde. Pour  ce faire, une  fois que
nous avons appris le  système, nous n'avons guère besoin de  plus de travail que
ce qui  est normalement nécessaire  pour taper  le document dans  n'importe quel
traitement de texte ou éditeur de texte. En fait, une fois que nous avons acquis
une  certaine aisance  avec \ConTeXt,  au total  notre travail  est probablement
moindre si  nous gardons à l'esprit  que les principaux détails  de formatage du
document sont  décrits globalement dans  \ConTeXt, et que nous  travaillons avec
des fichiers texte qui  sont -- une fois que nous nous y  sommes habitués -- une
façon beaucoup plus naturelle de traiter la création et l'édition de documents~;
d'autant  plus que  ces types  de  fichiers sont  beaucoup plus  légers et  plus
faciles à traiter que les lourds fichiers binaires des traitements de texte.

%-------------------------------------------------------------------------------
\startsubsubsubject{Documentation}

Il existe une documentation considérable  sur \ConTeXt, presque exclusivement en
anglais.  Par  exemple, ce que l'on  considère comme étant la  distribution {\em
  officielle} de \ConTeXt\ -- appelée \suite-
\footnote{Au moment de la  première version de ce texte, ceci  était vrai ; mais
  au printemps 2020,  le Wiki ConTeXt a  été mis à jour et  nous devons supposer
  qu'à partir de ce moment la distribution \quotation{officielle} de ConTeXt est
  devenue LMTX.  Cependant, pour ceux qui  entrent dans le monde de ConTeXt pour
  la première fois, je recommande  quand même d'utiliser \suite-{} puisque c'est
  une  distribution plus  stable.   L'\in{annexe}[installation_suite]  explique
  comment installer l'une ou l'autre des distributions.} --
contient une documentation de quelques 180 fichiers PDF (la majorité en anglais,
mais d'autres  en néerlandais et  en allemand)  avec notamment des  manuels, des
exemples  et  des  articles  techniques~;  et  sur  le  \goto{site  web  Pragma
  ADE}[url(pragma)] (la  société qui a  donné naissance à  \ConTeXt) il y  a (le
jour  où   j'ai  fait  le  décompte   en  mai  2020)  224   documents  librement
téléchargeables, dont la plupart sont  distribués avec la \suite- mais également
quelques   autres.    Cependant,   cette    énorme   documentation   n'est   pas
particulièrement  utile durant  la phase  d'apprentissage de  \ConTeXt, car,  en
général, ces documents ne s'adressent pas à un lecteur désireux d'apprendre mais
novice, qui  ne connaît rien  du système.  Sur les  56 fichiers PDF  que \suite-
appelle \quotation{manuels}, un seul suppose que  le lecteur ne connait rien sur
\ConTeXt.   Il  s'agit du  document  intitulé  \from[excursion] ou  en  français
\quotation{\ConTeXt\ Mark~IV,  une escapade}.  Mais  ce document, comme  son nom
l'indique,  se limite  à  présenter  le système  et  à  expliquer comment  faire
certaines choses  qui peuvent être  faites avec  \ConTeXt.  Ce serait  une bonne
introduction s'il était suivi d'un manuel  de référence un peu plus structuré et
systématique.  Mais  un tel  manuel n'existe  pas et  l'écart entre  le document
\from[excursion] et le reste de la documentation est trop important.

En 2001, un  \goto{manuel de référence}[url(manualref2001)] a été  rédigé~; mais
malgré son titre,  d'une part il n'a  pas été conçu pour être  un manuel complet
(la tâche étant titanesque), et d'autre part il était (est) destiné à la version
précédente de \ConTeXt{} (appelé Mark~II)  et intégre donc des éléments obsolètes,
ce qui perturbe l'apprentissage.

En 2013,  ce manuel  \goto{a été  partiellement mis  à jour}[url(manualref2013)]
mais  beaucoup de  ses  sections n'ont  pas  été réécrites  et  il contient  des
informations relatives  à la fois à  \ConTeXt\ Mark~II et \ConTeXt\  Mark~IV (la
version  actuelle), sans  toujours préciser  clairement quelles  informations se
rapportent à chacune  des versions.  C'est peut-être la raison  pour laquelle ce
manuel ne  se trouve  pas parmi  les documents  inclus dans  \suite-.  Pourtant,
malgré ces défauts, et une fois lu \from[excursion], le manuel reste le meilleur
document  pour continuer  à apprendre  \ConTeXt.  Autres  informations également
toujours  très utiles  pour  démarrer  avec \ConTeXt,  celles  contenues sur le
\from[wiki], qui, au  moment où nous écrivons ces lignes,  est en plein remaniement
et  présente  une  structure  beaucoup  plus  claire,  bien  qu'elles  mélangent
également des  explications qui ne  fonctionnent que dans Mark~II  avec d'autres
pour  Mark~IV ou  pour  les  deux versions.   Ce  manque  de différenciation  se
retrouve également dans la liste officielle des commandes \from[commandes]
\footnote{Pour la liste, voir \in{section}[sec:qrc-setup-fr].}
qui  comprend, pour  chacune d'elles,  l'ensemble des  options de  configuration
possibles mais ne  précise pas quelles commandes ne fonctionnent  que dans l'une
ou l'autre des versions. 

Fondamentalement, cette  introduction a  été rédigée  en s'inspirant  des quatre
sources    d'information    énumérées    préalablement~:    \from[excursion],
\from[manualref2013],  le contenu  de \from[wiki],  et la  liste officielle  des
commandes  \from[commandes],  en  plus,  bien  sûr, de  mes  propres  essais  et
tribulations.  Ainsi,  cette introduction  résulte d'un  effort d'investigation,
et, durant  un temps, j'ai  été tenté de l'appeler  \quotation{Ce que je  sais à
  propos de  \ConTeXt\ Mark~IV} ou  \quotation{Ce que j'ai appris  sur \ConTeXt\
  Mark~IV}. Finalement,  aussi vraie que  soit leur  teneur, ces titres  ont été
écartés car  j'ai pensé  qu'ils risquaient de  dissuader certains  de s'investir
dans  \ConTeXt. Il  est  certain,  malgré que  la  documentation  ait selon  moi
certaines lacunes, que \ConTeXt\  est un  outil vraiment utile  et polyvalent,
pour lequel l'effort d'apprentissage vaut sans aucun doute la peine.  {\bf Grâce
  à  \ConTeXt, nous  pouvons manipuler  et configurer  des documents  texte pour
  réaliser des choses que ceux qui ne connaissent pas le système ne peuvent tout
  simplement pas imaginer}.

\startSmallPrint
  Je ne peux pas empêcher -- parce que  je suis comme je suis -- que mes regrets
  concernant les déficiences d'information ressurgissent  de temps en temps dans
  ce document.  Je ne  veux pas qu'il y ait de malentendu~: je suis immensément
  reconnaissant aux créateurs de \ConTeXt\ d'avoir conçu un outil aussi puissant
  et de l'avoir mis à la disposition du public.  C'est simplement que je ne peux
  m'empêcher  de penser  que  cet outil  serait beaucoup  plus  populaire si  sa
  documentation  était améliorée~:  il  faut investir  beaucoup  de temps  pour
  s'approprier \ConTeXt,  non pas  tant en raison  de sa  difficulté intrinsèque
  (qui existe, mais qui n'est pas supérieure à celle d'autres outils spécialisés
  similaires,  c'est  même  plutôt  le  contraire), mais  en  raison  du  manque
  d'informations claires, complètes et systématiques, qui différencient les deux
  versions de  ConTEXt, expliquent  ce qui fonctionne  dans chacune  d'elles et,
  surtout, précisent à quoi sert chaque commande, argument et option.

  Il est vrai  que ce type d'information exigerait  un fort investissement en
  temps.  Mais comme de nombreuses commandes partagent des options avec des noms
  similaires, on  pourrait peut-être  rédiger une sorte  de {\em  glossaire} des
  options,  ce  qui permettrait  également  de  détecter certaines  incohérences
  produites lorsque  deux options du  même nom  font des choses  différentes, ou
  lorsque  des  noms  d'options  différents sont  utilisés  dans  des  commandes
  différentes pour faire la même chose.

  Quant au  lecteur qui s'intéresse  à \ConTeXt{} pour  la première fois,  que mes
  plaintes ne le dissuadent pas, car s'il est vrai que le manque d'informations,
  claires   complètes  et   systématiques,  augmente   le  temps   nécessaire  à
  l'apprentissage, du  moins pour  les sujets  traités dans  cette introduction,
  j'ai  déjà  investi  ce temps,  de  sorte  que  le  lecteur n'aura  pas  à  le
  refaire. Et  déjà avec  ce que  vous aurez  l'occasion d'apprendre  dans cette
  introduction,  vous   pourrez  produire   des  documents  avec   de  puissants
  utilitaires insoupçonnés.
\stopSmallPrint

\stopsubsubsubject
%-------------------------------------------------------------------------------
\startsubsubsubject{Incertitudes}

Étant donné que ce qui est expliqué dans ce document provient essentiellement de
mes  propres conclusions,  il est  probable que,  bien qu'ayant  personnellement
vérifié  une grande  partie  de ce  qui est  exposé,  certaines déclarations  ou
opinions soient  incorrectes ou non orthodoxes.   Bien évidemment, j'apprécierai
toute correction,  toute nuance ou encore  précisions que vous accepterez  de me
faire  parvenir  à \goto{joaquin@ataz.org}[url(mailto:joaquin@ataz.org)].   Pour
limiter les  occasions d'erreur, j'ai  essayé de ne  pas aborder les  sujets sur
lesquels je  n'ai pas  trouvé d'informations ou  que je n'ai  pas pu  (ou voulu)
vérifier personnellement ; parce que parfois le résultat de mon test n'était pas
concluant, d'autres fois parce que je n'ai  pas toujours tout essayé~: le nombre
de commandes et  d'options de \ConTeXt{} est impressionnant, et  si je devais tout
essayer, je n'aurais jamais réussi  à finaliser cette introduction. Néanmoins, à
certaines occasions je n'ai pas pu  éviter de faire certaines {\em conjectures},
c'est-à-dire une déclaration  que je considère comme  probable mais dont  je ne
suis  pas  totalement  sûr.   Dans  ce cas,  en  marge  du  paragraphe,  l'image
\Conjecture{} qui peut être vue à gauche de cette ligne indiquera visuellement
la présence d'une conjecture%
\footnote{Je  n'ai  pas  dessiné  l'image  moi-même,  elle  provient  d'internet
(\goto{https://es.dreamstime.com/}[url(https://es.dreamstime.com/icono-s\%C3\%B3lido-negro-para-la-conjetura-preocupaci\%C3\%B3n-y-duda-conjeturar-el-logotipo-image146773292)]), où il est indiqué qu'elle est libre de droit.}.
D'autres fois, je n'ai pas eu d'autre choix que d'admettre que je ne sais pas et
que je n'ai même pas d'hypothèse raisonnable  à ce sujet~: dans ce deuxième cas,
l'image utilisée  est celle insérée ici  en marge du paragraphe  \Doubt~ afin de
représenter plus que de simples conjectures, l'ignorance.
Mais n'ayant  jamais été très  doué pour  les représentations graphiques,  je ne
suis pas certain que les images sélectionnées parviennent vraiment à transmettre
ces nuances.

\stopsubsubsubject
%-------------------------------------------------------------------------------
\startsubsubsubject{Public visé}

Autre aspect,  cette introduction a  été écrite pour  un lecteur qui  ne connaît
rien à \TeX\ et \ConTeXt, bien  que j'espère qu'elle pourra également être utile
à ceux  qui s'approchent  pour la  première fois  de \TeX\  or \LaTeX\  (le plus
populaire  des outils  dérivés  de  \TeX).  Je  suis  cependant conscient  qu'en
essayant  de  satisfaire  des  lecteurs  aussi différents,  je  risque  de  n'en
satisfaire aucun. Par  conséquent, en cas de doute, j'ai  toujours été clair sur
le fait que le principal destinataire de ce document est le nouvel arrivant dans
le monde de \ConTeXt, le nouveau venu dans cet écosystème fascinant.

Être  un néophyte  \ConTeXt\ n'implique  pas d'être  également néophyte  dans la
manipulation des outils informatiques. Bien que cette introduction ne présuppose
aucun  niveau  spécifique de  compétence  informatique  chez son  lecteur,  elle
suppose  une  certaine  \quotation{aisance   raisonnable}  en  informatique  qui
implique, par exemple, de connaître dans  les grandes lignes la différence entre
un éditeur de texte  et un traitement de texte, de  savoir comment créer, ouvrir
et manipuler  un fichier  texte, de  savoir comment  installer un  programme, de
savoir comment ouvrir un terminal et y exécuter une commande... et un peu plus.

\startSmallPrint

  En  lisant les  parties de  cette introduction  déjà écrites  au moment  de la
  rédaction, je me rends  compte que parfois, je m'emporte et  me lance dans des
  problèmes  informatiques qui  ne  sont pas  nécessaires  à l'apprentissage  de
  \ConTeXt\ et peuvent  effrayer le néophyte, tandis que d'autres  fois, je suis
  occupé à expliquer  des choses assez évidentes qui peuvent  ennuyer le lecteur
  expérimenté.  Je demande votre indulgence.  Rationnellement, je sais qu'il est
  très difficile pour  un total débutant en traitement de  texte informatique de
  connaître  l'existence même  de \ConTeXt\,  mais,  d'un autre  côté, dans  mon
  environnement  professionnel, je  suis  entouré de  personnes  qui se  battent
  continuellement en  utilisant les logiciels  de traitement de texte,  et elles
  s'en sortent  raisonnablement bien,  mais néanmoins, n'ayant  jamais travaillé
  avec des  fichiers texte, elles  ignorent des aspects aussi  élémentaires que,
  par exemple,  ce qu'est  l'encodage ou  la différence entre  un éditeur  et un
  traitement de texte.

\stopSmallPrint

Le fait que ce manuel ait été conçu pour des personnes qui ne connaissent rien à
\ConTeXt\ ou à \TeX\ implique que  j'ai dû y inclure des informations concernant
non pas \ConTeXt\ mais  \TeX,  le système de  base mais j'ai  compris qu'il
n'était pas  nécessaire de surcharger  le lecteur  avec des informations  qui ne
l'intéressent guère, par exemple de savoir  si une telle commande qui fonctionne
{\em de facto} provient  de \ConTeXt\ ou bien de \TeX.  Par conséquent, ce n'est
qu'en certaines  occasions, lorsque je  l'ai considéré utile, qu'il  est précisé
que certaines commandes appartiennent réellement à \TeX.

\stopsubsubsubject
%-------------------------------------------------------------------------------
\startsubsubsubject{Structuration}

En ce  qui concerne l'organisation  de ce document,  le contenu est  présenté en
trois parties~:

\startitemize

%-------------------------------------------------------------------------------

  \item {\bf  La première partie},  qui comprend les quatre  premiers Chapitres,
    donne une  vue d'ensemble de \ConTeXt,  en expliquant ce que  c'est, comment
    l'utiliser, en montrant un premier  exemple de transformation d'un document,
    pour ensuite expliquer certains concepts fondamentaux de \ConTeXt\ ainsi que
    certaines questions relatives aux fichiers sources de \ConTeXt\.

    Dans l'ensemble  ces chapitres  sont destinés aux  lecteurs qui  ne savaient
    travailler jusqu'à présent qu'avec des  traitements de texte. Un lecteur qui
    connaît déjà l'utilisation des langages de balisage  peut sauter les deux
    premiers  chapitres. Si  le  lecteur  connaît déjà  \TeX{} ou  \LaTeX, il  peut
    également sauter une grande partie du contenu des Chapitres 3 et 4.  Mais je
    vous recommande quand même de lire au moins~:

    \startitemize

      \item   les   informations   relatives    aux   commandes   de   \ConTeXt\
        (\in{Chapitre}[cap:commands]), et en particulier sur la configuration de
        son fonctionnement,  car c'est  là que  réside la  principale différence
        dans les  conceptions et  syntaxes de \LaTeX\  et \ConTeXt.  Comme cette
        introduction ne  concerne que  ce dernier, ces  différences ne  sont pas
        expressément mentionnées comme telles, mais quiconque lit ce chapitre et
        connaît  le  fonctionnement  de  \LaTeX\  comprendra  immédiatement  les
        principales différences dans la syntaxe des deux languages, ainsi que la
        façon  dont  \ConTeXt\  permet  de configurer  et  de  personnaliser  le
        fonctionnement de presque toutes ses commandes.

      \item  Les  informations relatives  à  la  gestion des  projets  \ConTeXt\
        multi-fichiers  (\in{Chapitre}[cap:sourcefile]),  qui se  distingue  de  celles des  autres
        systèmes basés sur \TeX.

    \stopitemize

%-------------------------------------------------------------------------------

  \item {\bf La seconde partie}, qui comprend  les Chapitres 5 à 9, se concentre
    sur ce  que nous  pouvons considérer comme l'aspect général d'un document~:

    \startitemize

      \item  Les  deux aspects  qui  affectent  principalement l'apparence  d'un
        document sont les dimensions et la composition de ses pages ainsi que la
        police utilisée.  Les  chapitres \in{}[cap:pages] et \in{}[cap:fontscol]
        sont consacrés à ces deux questions.

      \startitemize

        \item Le \in{Chapitre}[cap:pages]  se concentre sur les  pages~: taille,
          éléments  qui les  composent, conception  ou composition  (c'est-à-dire
          répartition  des différents  éléments  sur la  page),  etc.  Pour  une
          raison de  systématisme, des  aspects plus spécifiques  sont également
          traités ici, comme ceux relatifs à la pagination et aux mécanismes qui
          nous permettent de l'influencer.

        \item  Le \in{Chapitre}[cap:fontscol]  explique les  commandes relatives
          aux  polices et  à  leur  manipulation.  Une  explication  de base  de
          l'utilisation et de la manipulation des couleurs est également incluse
          ici, car,  bien que les couleurs  ne soient pas, à  proprement parler,
          une {\em caractéristique}  de la police, elles influencent  de la même
          manière l'apparence visuelle du document.

      \stopitemize

      \item Les chapitres \in{}[cap:structure]  et \in{}[cap:toc] se concentrent
        sur  la structure  du document  et  les outils  que \ConTeXt\  met à  la
        disposition de l'auteur pour la  rédaction de documents bien structurés.
        Le \in{Chapitre}[cap:structure] se concentre  sur la structure elle-même
        (divisions structurelles  du document) et le  \in{Chapitre}[cap:toc] sur
        la valorisation de cette structure dans une table des matières.

      \item Enfin, le \in{Chapitre}[cap:refpluslinks]  se concentre sur l'aspect
        clé  de l'utilisation  de références  par un  document, références  vers
        d'autres points du  même document (références internes)  mais aussi vers
        des  documents  externes  (références externes).   Dans  cette  dernière
        situation, nous n'aborderons que le cas de références impliquant un {\em
          liens} vers  le document externe.  Ce  chapitre explique l'utilisation
        de outil  de \ConTeXt{} pour la  gestion de ces références  qui permet des
        {\em sauts} entre différents zones d'informations, internes ou externes,
        et rend ainsi notre document {\em interactif}.

    \stopitemize

  Ces chapitres  n'ont pas  besoin d'être  lus dans  un ordre  particulier, sauf
  peut-être  le   \in{Chapitre}[cap:toc],  qui  est  peut-être   plus  facile  à
  comprendre si vous  avez d'abord lu le  \in{Chapitre}[cap:structure].  En tout
  cas, j'ai  essayé de faire  en sorte que lorsqu'une  question se pose  dans un
  chapitre ou une section, alors qu'elle  est traitée ailleurs dans ce document,
  le texte  mentionne ce  fait et propose  un hyperlien vers  le point  où cette
  question est traitée.  Je  ne suis cependant pas en mesure  de garantir que ce
  sera toujours le cas.

%-------------------------------------------------------------------------------

  \item  Enfin,  {\bf  la   troisième  partie}  (\in{Chapitre}[cap:chartext]  et
    suivants) se concentre sur des  aspects plus concrets, plus spécialisés. Non
    seulement les  chapitres sont  maintenant indépendants  les uns  des autres,
    mais également les  sections les unes des autres au  sein d'un même chapitre
    (sauf, peut-être, dans le dernier chapitre).  Étant donné la grande quantité
    de fonctionnalités proposées par  \ConTeXt\, cette troisième partie pourrait
    être très vaste ; mais comme je  devine qu'en arrivant à ce stade le lecteur
    sera capable de plonger lui-même dans la documentation de \ConTeXt\, je n'ai
    considéré que les chapitres suivants~:

    \startitemize 

      \item  Les   chapitres  \in{}[cap:chartext]   et  \in{}[cap:parlinevspace]
        traitent de ce que l'on pourrait appeler les {\em éléments clés} de tout
        document texte~:  Le texte est constitué de caractères,  qui forment des
        mots,  qui  sont regroupés  en  lignes,  qui  à  leur tour  forment  des
        paragraphes,  qui  sont  séparés  les  uns  des  autres  par  un  espace
        vertical...  Il  est évident que  toutes ces questions auraient  pu être
        incluses dans  un seul chapitre. Mais  comme cela aurait été  trop long,
        j'ai  divisé  cette  question  en  deux  chapitres,  l'un  traitant  des
        caractères, des mots et des espaces horizontaux, et l'autre traitant des
        lignes, des paragraphes et des espaces verticaux.

      \item Le \in{Chapitre}[cap:specialparas] est une sorte de {\em pot-pourri}
        qui traite des éléments et constructions  qui sont communs à de nombreux
        documents,     principalement     s'ils      sont     académiques     ou
        scientifico-techniques~:  notes de  bas  de  page, listes  structurées,
        descriptions, énumérations, etc.

      \item  Enfin, le  \in{Chapitre}[cap:floats]  se concentre  sur les  objets
        flottants insérés  dans un document et  en particulier les deux  cas les
        plus répandus~: les images et les tableaux.

  \stopitemize

%-------------------------------------------------------------------------------

  \item   Cette   introduction  à   \ConTeXt\   se   termine  par   trois 
    \in{{\bf Annexes}.}[part:annexes] L'une concerne l'installation de \ConTeXt\ sur
    un  ordinateur,  une  deuxième   annexe  présentent  plusieurs  dizaines  de
    commandes qui permettent  de générer divers symboles  (principalement, mais
    pas  uniquement,  pour  un  usage mathématique),  et  une  troisième  annexe
    rassemble les commandes \ConTeXt\ expliquées  ou mentionnées tout au long de
    ce texte sous la forme d'une liste alphabétique.

%-------------------------------------------------------------------------------

\stopitemize

\stopsubsubsubject
%-------------------------------------------------------------------------------
\startsubsubsubject{Reste à faire}


De nombreux  points restent  à expliquer~: le traitement  des citations  et des
références  bibliographiques, la  rédaction de  textes spéciaux  (mathématiques,
chimie...), la connexion avec XML, l'interface pour le code Lua, les modes et la
compilation basée sur  les modes, la collaboration avec MetaPost  pour le design
graphique, etc.  Par conséquent, comme ce document ne contient pas d'explication
complète de  \ConTeXt\, et ne  prétend pas le  faire, j'ai intitulé  ce document
\quotation{une courte  (?) introduction à  \ConTeXt\ Mark~IV)}, et  ajouté entre
parenthèses l'observation  \quotation{pas trop  courte} car, de  toute évidence,
elle l'est.  Un texte qui  laisse tant de choses  dans l'encrier et  qui dépasse
pourtant   300    pages   n'est    certainement   pas    une   \quotation{courte
  introduction}.  C'est parce  que j'essaie  de faire  comprendre au  lecteur la
logique de  \ConTeXt\~; ou  du moins  la logique telle  que je la  comprends. Il
n'est  pas  destiné  à  être  un  manuel de  référence,  mais  plutôt  un  guide
d'auto-apprentissage  qui  prépare  le  lecteur  à  réaliser  des  documents  de
complexité moyenne  (ce qui inclut la  plupart des documents possibles)  et qui,
surtout, lui apprend à {\em envisager et imaginer} ce qui peut être fait avec ce
puissant outil  et à {\em  repérer} dans la  documentation comment le  faire. Ce
document  n'est pas  non  plus  un tutoriel.   Les  tutoriels  sont conçus  pour
augmenter progressivement la difficulté, afin que ce qui doit être enseigné soit
appris pas à pas~; j'ai, en ce  sens, préféré, dès la deuxième partie, être plus
systématique au  lieu d'ordonner le  sujet en  fonction de sa  difficulté. Mais,
même si ce n'est pas un tutoriel, j'ai inclus de nombreux exemples.


Il est possible que  le titre de ce document rappelle  à certains lecteurs celui
écrit par {\sc Oetiker, Partl, Hyna  et Schlegl}, en  anglais
\quotation{{\em The Not So Short Introduction to \LaTeX\ $2_{\epsilon}$}}
et en français
\quotation{{\em  Une  courte  (?) introduction à  \LaTeX\  $2_{\epsilon}$}}.
Ce  document,
\goto{disponible en 24 langues}[url(https://ctan.org/tex-archive/info/lshort)],
est l'un des  meilleurs documents pour se familiariser avec  le monde de \LaTeX.
Ce n'est pas une coïncidence, mais un hommage et un acte de gratitude~: grâce au
travail généreux de  ceux qui écrivent de  tels textes, il est  possible pour de
nombreuses personnes de s'initier à des outils utiles et puissants comme \LaTeX\
et \ConTeXt.  Ces auteurs m'ont aidé à me lancer dans \LaTeX\~; j'ai l'intention
de faire de même  pour ceux qui veulent se lancer dans le  \ConTeXt, bien que je
sois  limité au  public hispanophone,  qui,  par ailleurs,  manque tellement  de
documentation dans sa langue.  J'espère que ce document répondra à cet objectif.

\page[no]\blank

\rightaligned{Joaquín Ataz-López}\\
\rightaligned{Eté 2020}

\stopsubsubsubject

\stopchapter

\stopcomponent

%%% Local Variables:
%%% mode: ConTeXt
%%% mode: auto-fill
%%% TeX-master: "../introCTX_fra.tex"
%%% coding: utf-8-unix
%%% End:
%%% vim:set filetype=context tw=72~: %%%
