\startcomponent 02-01_Introduction

\environment introCTX_env_00

\startchapter
  [title=Introduction et\\ vue d'ensemble,
   reference=cap:panorama,
   bookmark=Introduction et vue d'ensemble,]

% * Intro 3 concepts =================================================

  \ConTeXt\ est un {\em système de composition et de mise en page} de documents
  électroniques. Il vise à donner à l'utilisateur un contrôle le plus complet et
  précis possible sur l'apparence finale de ses documents électroniques, qu'ils
  soient imprimés sur papier ou affichés sur écran.

\startitemize
\item
{\em La composition} au sens de l'imprimerie, consiste à mettre en forme chaque
élément constituant le contenu du document afin d'en préparer son impression,
son affichage.
\item
{\em La mise en page}, toujours au sens de l'imprimerie, consiste à réunir,
assembler, positionnner les différents éléments issus de l'étape de composition
pour constituer les pages du document final.
\stopitemize

Pour être à l'aise avec l'utilisation de \ConTeXt , comme avec toute technologie
informatique, il est bon d'en comprendre les concepts clés. \ConTeXt\ s'appuie
sur trois concepts clés~:

\startitemize[n]
\item les fichiers sources,
\item le balisage,
\item la composition et mise en page.
\stopitemize

\startsubsubsubject [title={Les fichiers sources}]

Les fichiers sources sont la donnée d'entrée de \ConTeXt.  Il s'agit de ce qu'il
y a de plus simple en informatique~: des fichiers textes, c'est-à-dire des
fichiers lisibles et modifiables sur tous les ordinateurs du monde, avec un
simple éditeur de texte.

Cette universalité se fait au prix de deux conditions~: (1) ces fichiers ne
contiennent que des suites de caractères (pas de couleur, pas de mise en page,
pas d'images, pas de sons… rien que du texte), et (2) ils utilisent un codage
particulier des caractères (c'est-à-dire une convention de représentation
informatique des caractères) que l'on appelle
\goto{UTF-8}[url(https://fr.wikipedia.org/wiki/UTF-8)]. 
Ce codage particulier est dorénavant utilisé par plus de 97\% des sites web et
permet de représenter les milliers de caractères du répertoire universel des
caractères codés (c'est-à-dire la très grande majorité des caractères utilisés
par les différentes langues du monde).

\ConTeXt\ prend donc en entrée des fichiers textes.
Ces derniers servent de {\em contenant} au {\em contenu} informationnel de votre
document.  Pour l'information textuelles brute de votre document, c'est-à-dire
les mots, les paragraphes, c'est assez facile à comprendre.  Mais la question se
pose pour les autres informations que vous ne manquerez pas de vouloir exprimer,
par exemple 
\quotation{ici, il faut telle image}, 
\quotation{ce texte est un titre de chapitre}, 
\quotation{ce texte doit être en rouge}, 
\quotation{la page doit avoir un format A4}…

Comme le fichier ne contient que des suites de caractères, toutes ces
informations (sémantique, structure, couleur, composition, mise en page…)
doivent être décrites par des suites de caractères. Cela se fait en deux temps
avec les deux concepts suivants.

\stopsubsubsubject

\startsubsubsubject [title={Le balisage}]

L'utilisateur va commencer par faire la distinction entre les différents types
d'information au sein des fichiers sources.  La science informatique a développé
pour cela des techniques et langages dits de {\em balisage} tels que 
\goto{HTML}[url(https://fr.wikipedia.org/wiki/Hypertext_Markup_Language)],   
\goto{Markdown}[url(https://daringfireball.net/projects/markdown/syntax)], 
\goto{Org-mode}[url(https://orgmode.org/)] et 
\goto{\ConTeXt}[url(https://wiki.contextgarden.net)].

Apprendre \ConTeXt\ implique donc d'apprendre le langage de balisage 
de \ConTeXt.

\stopsubsubsubject 

\startsubsubsubject [title={La composition et la mise en page}]

Une fois les différents types d'information repérés / distingués, l'utilisateur
va ensuite définir les règles de composition et de mise en page à appliquer à
chacun.  La science informatique a également développé pour cela des techniques
et langages dédiés tels que les feuilles de style en cascade (Cascading Style
Sheets
\goto{CSS}[url(https://fr.wikipedia.org/wiki/Feuilles_de_style_en_cascade)])
pour les éléments balisés avec HTML et… 
\goto{\ConTeXt}[url(https://wiki.contextgarden.net)].

Le langage \ConTeXt\ permet donc, en complément du balisage, de définir
l'ensemble des règles de composition et de mise en page à appliquer.

Les règles par défaut sont l'une des forces de \ConTeXt\ car elles donnent
immédiatement d'excellents résultats. Il {\em suffit} ensuite de les
personnaliser, les compléter, les détailler. On découvre alors que l'art de la
composition est particulièrement riche et vaste -- \quotation{tout est permis}
et \quotation{l'imagination est la seule limite} -- ce qui se traduit par un
vocabulaire \ConTeXt\ également riche et vaste, même s'il bénéficie d'un effort
continu de standardisation.

On peut estimer au nombre de 100 les mots de vocabulaire qui couvriront la très
grande majorité de vos besoins à la fois de balisage, de composition et de mise
en page, même pour des documents complexes.

Contrairement au couple HTML / CSS qui partage clairement les rôles de balisage
(HTML) et de composition et mise en page (CSS) avec des langages et fichiers
sources distincts, \ConTeXt\ brouille la frontière en proposant un seul et même
langage pour traiter les deux aspects.

Aussi, pour progresser dans votre apprentissage de \ConTeXt, ce document va
tenter de systématiquement vous faire percevoir les deux aspects. Sans être
révolutionnaire, ce document vous propose une structuration en 4 parties un peu
inédite dans le microcosme des livres concernant \TeX, \LaTeX\ et \ConTeXt.

\stopsubsubsubject

% * Intro Partie 1   =================================================

\startsection [title={Les fichiers sources}]

\startcolumns[n=2]

La \in{Partie}[part:src] se concentre sur les fichiers sources.  Pour fixer
rapidement les idées, nous commencerons par un premier exemple au
\in{Chapitre}[cap:firstdoc]. La logique et les règles syntaxiques seront ensuite
explicitées au \in{Chapitre}[cap:commands]. Puis nous verrons comment organiser
nos fichiers sources au \in{Chapitre}[cap:sourcefile] avec notamment~:

\start
\tfx
\bTABLE[frame=off,strut=yes,option=stretch,location=top]
\setupTABLE[r][1][style=\bf,width=4cm,topframe=on]
\bTR \bTD Organisation des fichiers sources \eTD \eTR
\bTR \bTD \tex{starttext}                   \eTD \eTR
\bTR \bTD \tex{startproject}                \eTD \eTR
\bTR \bTD \tex{startproduct}                \eTD \eTR
\bTR \bTD \tex{product}                     \eTD \eTR
\bTR \bTD \tex{startcomponent}              \eTD \eTR
\bTR \bTD \tex{component}                   \eTD \eTR
\bTR \bTD \tex{startenvironment}            \eTD \eTR
\bTR \bTD \tex{environment}                 \eTD \eTR
\bTR \bTD \tex{usepath}                     \eTD \eTR
\eTABLE
\stop
\stopcolumns

\stopsection 

% * Intro Partie 2   =================================================

\startsection [title=La composition d'ensemble]

La \in{Partie}[part:global] concerne les éléments de composition d'ensemble et
de mise en page. Ils sont désignés ainsi car ils affectent l'ensemble de la
composition et mise en page du document final. Il s'agit des 7 sujets suivants~:

\startcolumns[n=2]
\tfx
\bTABLE[frame=off,strut=yes,option=stretch,location=top]
\setupTABLE[r][1][style=\bf,width=4cm,topframe=on,width=5cm]
\bTR \bTD 1. \goto{Mise en page}[cap:pages]                                 \eTD \eTR 
\bTR \bTD \tex{definepapersize}                                             \eTD \eTR
\bTR \bTD \tex{setuppapersize}                                              \eTD \eTR
\bTR \bTD \tex{setuplayout}                                                 \eTD \eTR
\bTR \bTD \tex{setupbackgrounds}                                            \eTD \eTR
\bTR \bTD \tex{showframe}                                                   \eTD \eTR
\eTABLE

\bTABLE[frame=off,strut=yes,option=stretch,location=top]
\setupTABLE[r][1][style=\bf,width=4cm,topframe=on,width=5cm]
\bTR \bTD 2. \goto{En-tête, pied et numérotation de page}[sec:headerfooter] \eTD \eTR
\bTR \bTD \tex{setupfootertexts}                                            \eTD \eTR
\bTR \bTD \tex{setupheadertexts}                                            \eTD \eTR
\bTR \bTD \tex{setuppagenumbering}                                          \eTD \eTR                                           
\eTABLE

\bTABLE[frame=off,strut=yes,option=stretch,location=top]
\setupTABLE[r][1][style=\bf,width=4cm,topframe=on,width=5cm]                     
\bTR \bTD 3. \goto{Paragraphes}[cap:parlinevspace]                          \eTD \eTR                 
\bTR \bTD \tex{par}                                                         \eTD \eTR
\bTR \bTD \tex{crlf}                                                        \eTD \eTR
\bTR \bTD \tex{setupinterlinespace}                                         \eTD \eTR
\bTR \bTD \tex{setupwhitespace}                                             \eTD \eTR
\bTR \bTD \tex{setupalign}                                                  \eTD \eTR
\bTR \bTD \tex{setuptolerance}                                              \eTD \eTR
\eTABLE

\bTABLE[frame=off,strut=yes,option=stretch,location=top]
\setupTABLE[r][1][style=\bf,width=4cm,topframe=on,width=5cm]                     
\bTR \bTD 4. \goto{Polices}[cap:fontscol]                                   \eTD \eTR
\bTR \bTD \tex{setupbodyfont}                                               \eTD \eTR
\bTR \bTD \tex{switchtobodyfont}                                            \eTD \eTR
\bTR \bTD \tex{definedfont}                                                 \eTD \eTR
\bTR \bTD \tex{definefontfeature}                                           \eTD \eTR
\eTABLE

\bTABLE[frame=off,strut=yes,option=stretch,location=top]
\setupTABLE[r][1][style=\bf,width=4cm,topframe=on,width=5cm]                     
\bTR \bTD 5. \goto{Couleurs}[sec:colors]                                    \eTD \eTR
\bTR \bTD \tex{usecolors}                                                   \eTD \eTR
\bTR \bTD \tex{color}                                                       \eTD \eTR
\bTR \bTD \tex{definecolor}                                                 \eTD \eTR
\eTABLE

\bTABLE[frame=off,strut=yes,option=stretch,location=top]
\setupTABLE[r][1][style=\bf,width=4cm,topframe=on,width=5cm]                     
\bTR \bTD 6. \goto{Langue}[sec:langdoc]                                     \eTD \eTR
\bTR \bTD \tex{language}                                                    \eTD \eTR
\bTR \bTD \tex{mainlanguage}                                                \eTD \eTR
\bTR \bTD \tex{setcharacterspacing}                                         \eTD \eTR
\eTABLE

\bTABLE[frame=off,strut=yes,option=stretch,location=top]
\setupTABLE[r][1][style=\bf,width=4cm,topframe=on,width=5cm]                     
\bTR \bTD 7. \goto{Interactivité}[sec:interactivity]                        \eTD \eTR
\bTR \bTD \tex{setupattachments}                                            \eTD \eTR
\bTR \bTD \tex{setupinteraction}                                            \eTD \eTR
\eTABLE
\stopcolumns

\stopsection

% * Intro Partie 3   =================================================

\startsection [title=Éléments du flux principal]

Nous verrons en \in{Partie}[part:mainflow] le balisage et la composition des
éléments du {\bf flux principal d'information} qui se déroule linéairement de
page en page, ou bien d'écran en écran. Il s'agit essentiellement des 17
éléments suivants, en partant du plus petit / moins structurant, au plus
macroscopique / plus structurant~:

\startcolumns[n=2]
\tfx
\bTABLE[frame=off,option=stretch,location=top]
\setupTABLE[r][1][style=\bf,width=4cm,topframe=on,width=5cm]                     
\bTR \bTD 1. \goto{Emphase de mots}[cap:emphword]               \eTD \eTR
\bTR \bTD \tex{em}                                              \eTD \eTR
\bTR \bTD \tex{definehighlight}                                 \eTD \eTR
\bTR \bTD \tex{cap}                                             \eTD \eTR
\eTABLE

\bTABLE[frame=off,option=stretch,location=top]
\setupTABLE[r][1][style=\bf,width=4cm,topframe=on,width=5cm] 
\bTR \bTD 2. \goto{Emphase de paragraphes}[cap:emphpara]        \eTD \eTR
\bTR \bTD \tex{startalign}                                      \eTD \eTR
\bTR \bTD \tex{setupnarrower}                                   \eTD \eTR
\bTR \bTD \tex{startnarrower}                                   \eTD \eTR
\eTABLE

\bTABLE[frame=off,option=stretch,location=top]
\setupTABLE[r][1][style=\bf,width=4cm,topframe=on,width=5cm] 
\bTR \bTD 3. \goto{Encadrement}[cap:framed]                     \eTD \eTR         
\bTR \bTD \tex{startframed}                                     \eTD \eTR  
\eTABLE

\bTABLE[frame=off,option=stretch,location=top]
\setupTABLE[r][1][style=\bf,width=4cm,topframe=on,width=5cm] 
\bTR \bTD 4. \goto{Lignes et traits}[cap:lines]                 \eTD \eTR                 
\bTR \bTD \tex{starttextrule}                                   \eTD \eTR   
\bTR \bTD \tex{blackrule}                                       \eTD \eTR   
\eTABLE

\bTABLE[frame=off,option=stretch,location=top]
\setupTABLE[r][1][style=\bf,width=4cm,topframe=on,width=5cm] 
\bTR \bTD 5. \goto{Citations}[cap:quotation]                    \eTD \eTR        
\bTR \bTD \tex{quote}                                           \eTD \eTR  
\bTR \bTD \tex{quotation}                                       \eTD \eTR  
\bTR \bTD \tex{startquotation}                                  \eTD \eTR  
\eTABLE

\bTABLE[frame=off,option=stretch,location=top]
\setupTABLE[r][1][style=\bf,width=4cm,topframe=on,width=5cm] 
\bTR \bTD 6. \goto{Mathématiques}[cap:maths]  \eTD \eTR
\bTR \bTD \tex{startformula}            \eTD \eTR  
\bTR \bTD \tex{startcases}              \eTD \eTR  
\bTR \bTD \tex{stopitemize}             \eTD \eTR  
\eTABLE

\bTABLE[frame=off,option=stretch,location=top]
\setupTABLE[r][1][style=\bf,width=4cm,topframe=on,width=5cm] 
\bTR \bTD 7. \goto{Listes structurées}[cap:itemize]            \eTD \eTR
\bTR \bTD \tex{startitemize}            \eTD \eTR  
\bTR \bTD \tex{item}                    \eTD \eTR  
\eTABLE

\bTABLE[frame=off,option=stretch,location=top]
\setupTABLE[r][1][style=\bf,width=4cm,topframe=on,width=5cm] 
\bTR \bTD 8. \goto{Description et énumération}[cap:descenum]    \eTD \eTR
\bTR \bTD \tex{definedescription}       \eTD \eTR
\bTR \bTD \tex{defineenumeration}       \eTD \eTR
\eTABLE

\bTABLE[frame=off,option=stretch,location=top]
\setupTABLE[r][1][style=\bf,width=4cm,topframe=on,width=5cm] 
\bTR \bTD 9. \goto{Textes tabulés}[cap:tabulate]                \eTD \eTR             
\bTR \bTD \tex{starttabulate}           \eTD \eTR
\bTR \bTD \tex{NR}                      \eTD \eTR
\bTR \bTD \tex{NC}                      \eTD \eTR
\eTABLE

\bTABLE[frame=off,option=stretch,location=top]
\setupTABLE[r][1][style=\bf,width=4cm,topframe=on,width=5cm] 
\bTR \bTD 10. \goto{Tableaux}[cap:table]                      \eTD \eTR      
\bTR \bTD \tex{bTABLE}                  \eTD \eTR
\bTR \bTD \tex{bTABLEbody}              \eTD \eTR
\bTR \bTD \tex{bTABLEhead}              \eTD \eTR
\bTR \bTD \tex{bTD}                     \eTD \eTR
\bTR \bTD \tex{bTH}                     \eTD \eTR
\bTR \bTD \tex{bTR}                     \eTD \eTR
\eTABLE

\bTABLE[frame=off,option=stretch,location=top]
\setupTABLE[r][1][style=\bf,width=4cm,topframe=on,width=5cm] 
\bTR \bTD 11. \goto{Images et Combinaisons}[cap:figures]        \eTD \eTR
\bTR \bTD \tex{externalfigure}          \eTD \eTR
\bTR \bTD \tex{setupexternalfigure}     \eTD \eTR
\bTR \bTD \tex{startcombination}        \eTD \eTR
\eTABLE

\bTABLE[frame=off,option=stretch,location=top]
\setupTABLE[r][1][style=\bf,width=4cm,topframe=on,width=5cm] 
\bTR \bTD 12. \goto{Objets flottants}[cap:floats]              \eTD \eTR                
\bTR \bTD \tex{startplacetable}         \eTD \eTR
\bTR \bTD \tex{startplacefigure}        \eTD \eTR
\bTR \bTD \tex{setupcaption}            \eTD \eTR
\eTABLE

\bTABLE[frame=off,option=stretch,location=top]
\setupTABLE[r][1][style=\bf,width=4cm,topframe=on,width=5cm] 
\bTR \bTD 13. \goto{Colonnes}[cap:columns]                      \eTD \eTR        
\bTR \bTD \tex{startcolumns}            \eTD \eTR
\eTABLE

\bTABLE[frame=off,option=stretch,location=top]
\setupTABLE[r][1][style=\bf,width=4cm,topframe=on,width=5cm]
\bTR \bTD 14. \goto{Section}[cap:sections]                        \eTD \eTR
\bTR \bTD \tex{startpart}                \eTD \eTR
\bTR \bTD \tex{startchapter}             \eTD \eTR
\bTR \bTD \tex{startsection}             \eTD \eTR
\bTR \bTD \tex{startsubsection}          \eTD \eTR
\bTR \bTD \tex{startsubsubsection}       \eTD \eTR
\bTR \bTD \tex{startsubsubsubsection}    \eTD \eTR
\bTR \bTD \tex{setuphead}                \eTD \eTR
\eTABLE

\bTABLE[frame=off,option=stretch,location=top]
\setupTABLE[r][1][style=\bf,width=4cm,topframe=on,width=5cm]
\bTR \bTD 15. \goto{Macro-structure}[cap:macrostruct]                \eTD \eTR
\bTR \bTD \tex{startfrontmatter}         \eTD \eTR
\bTR \bTD \tex{startbodymatter}          \eTD \eTR
\bTR \bTD \tex{startappendices}          \eTD \eTR
\bTR \bTD \tex{startbackmatter}          \eTD \eTR
\eTABLE

\bTABLE[frame=off,option=stretch,location=top]
\setupTABLE[r][1][style=\bf,width=4cm,topframe=on,width=5cm]
\bTR \bTD 16. \goto{Page de couverture et de titre}[cap:covers] \eTD \eTR
\bTR \bTD \tex{definemakeup}             \eTD \eTR
\bTR \bTD \tex{startmakeup}              \eTD \eTR
\eTABLE

\bTABLE[frame=off,option=stretch,location=top]
\setupTABLE[r][1][style=\bf,width=4cm,topframe=on,width=5cm]
\bTR \bTD 17. \goto{Autres éléments spécialisés}[cap:others]    \eTD \eTR
\bTR \bTD \tex{startbuffer}              \eTD \eTR
\bTR \bTD \tex{getbuffer}                \eTD \eTR
\bTR \bTD \tex{startsetups}              \eTD \eTR
\bTR \bTD \tex{setups}                   \eTD \eTR
\eTABLE

\stopcolumns

\stopsection

% * Intro Partie 4   =================================================

\startsection [title=Compléments au flux principal]

Nous verrons en \in{Partie}[part:complements] le balisage et la composition des
{\bf compléments au flux principal}. Ils permettent de l'enrichir, d'y naviguer,
de faire des connexions complexes à d'autres informations. Ces éléments, au
nombre de 10, sont tout à fait essentiels car ils permettent de sortir des
limites du flux linéaire principal~:

\startcolumns[n=2]
\tfx
\bTABLE[frame=off,strut=yes,option=stretch,location=top]
\setupTABLE[r][1][style=\bf,width=4cm,topframe=on,width=5cm]
\bTR \bTD 1. \goto{Table des matières}[cap:toc]             \eTD \eTR
\bTR \bTD \tex{setupcombinedlist}        \eTD \eTR
\bTR \bTD \tex{placecontent}             \eTD \eTR
\eTABLE

\bTABLE[frame=off,strut=yes,option=stretch,location=top]
\setupTABLE[r][1][style=\bf,width=4cm,topframe=on,width=5cm]
\bTR \bTD 2. \goto{Abréviations et glossaire}[cap:abbr]      \eTD \eTR                         
\bTR \bTD \tex{abbreviation}             \eTD \eTR
\bTR \bTD \tex{placelistofabbreviations} \eTD \eTR
\eTABLE

\bTABLE[frame=off,strut=yes,option=stretch,location=top]
\setupTABLE[r][1][style=\bf,width=4cm,topframe=on,width=5cm]
\bTR \bTD 3. \goto{Notes de bas de page}[cap:notes]           \eTD \eTR                    
\bTR \bTD \tex{footnote}                 \eTD \eTR
\bTR \bTD \tex{setupfootnotedefinition}  \eTD \eTR
\bTR \bTD \tex{setupfootnotes}           \eTD \eTR
\eTABLE

\bTABLE[frame=off,strut=yes,option=stretch,location=top]
\setupTABLE[r][1][style=\bf,width=4cm,topframe=on,width=5cm]
\bTR \bTD 4. \goto{Notes marginales}[cap:margin]               \eTD \eTR                
\bTR \bTD \tex{margintext}               \eTD \eTR
\eTABLE

\bTABLE[frame=off,option=stretch,location=top]
\setupTABLE[r][1][style=\bf,width=4cm,topframe=on,width=5cm]
\bTR \bTD 5. \goto{Pièces jointes}[cap:attachment]                 \eTD \eTR              
\bTR \bTD \tex{attachment}               \eTD \eTR
\eTABLE

\bTABLE[frame=off,strut=yes,option=stretch,location=top]
\setupTABLE[r][1][style=\bf,width=4cm,topframe=on,width=5cm]
\bTR \bTD 6. \goto{Références internes}[cap:refint]            \eTD \eTR                   
\bTR \bTD \tex{reference}                \eTD \eTR
\bTR \bTD \tex{at}                       \eTD \eTR
\bTR \bTD \tex{in}                       \eTD \eTR
\eTABLE

\bTABLE[frame=off,option=stretch,location=top]
\setupTABLE[r][1][style=\bf,width=4cm,topframe=on,width=5cm]
\bTR \bTD 7. \goto{Références externes}[cap:refext]             \eTD \eTR                  
\bTR \bTD \tex{from}                     \eTD \eTR
\bTR \bTD \tex{goto}                     \eTD \eTR
\bTR \bTD \tex{useurl}                   \eTD \eTR
\bTR \bTD \type{url()}                   \eTD \eTR
\eTABLE

\bTABLE[frame=off,strut=yes,option=stretch,location=top]
\setupTABLE[r][1][style=\bf,width=4cm,topframe=on,width=5cm]
\bTR \bTD 8. \goto{Références bibliographiques}[cap:refbiblio]  \eTD \eTR                           
\bTR \bTD \tex{cite}                     \eTD \eTR
\bTR \bTD \tex{placelistofpublications}  \eTD \eTR
\bTR \bTD \tex{usebtxdataset}            \eTD \eTR
\eTABLE

\bTABLE[frame=off,strut=yes,option=stretch,location=top]
\setupTABLE[r][1][style=\bf,width=4cm,topframe=on,width=5cm]
\bTR \bTD 9. \goto{Listes des images, tableaux...}[cap:lists] \eTD \eTR                              
\bTR \bTD \tex{placelistoffigures}       \eTD \eTR
\bTR \bTD \tex{placelistoftables}        \eTD \eTR
\eTABLE

\bTABLE[frame=off,strut=yes,option=stretch,location=top]
\setupTABLE[r][1][style=\bf,width=4cm,topframe=on,width=5cm]
\bTR \bTD 10. \goto{Index}[cap:index]                          \eTD \eTR     
\bTR \bTD \tex{index}                    \eTD \eTR
\bTR \bTD \tex{placeindex}               \eTD \eTR
\eTABLE

\stopcolumns

\stopsection

% * Intro Galerie    =================================================


\startsection[title=Galerie]

Pour finir cette introduction, la galerie suivante vous donnera un aperçu visuel
de l'ensemble des différents éléments évoqués durant cette introduction.

\page

% pdfjam demonstration.pdf  --nup 2x22 --frame true   --delta "1mm 2mm" --outfile resultat.pdf
% pdfcrop resultat.pdf resultat.pdf

\start
\setupexternalfigure[frame=on,height=5.5cm,interaction=yes,background=color,backgroundcolor=white]

\startcombination[2*4]
{\goto{\externalfigure[demonstration.pdf][page=1]}[cap:covers]}         {}
{\goto{\externalfigure[demonstration.pdf][page=2]}[cap:macrostruct]}    {}
{\goto{\externalfigure[demonstration.pdf][page=3]}[cap:toc]}            {}
{\goto{\externalfigure[demonstration.pdf][page=4]}[cap:sections]}       {}
{\goto{\externalfigure[demonstration.pdf][page=5]}[cap:pages]}          {}
{\goto{\externalfigure[demonstration.pdf][page=6]}[sec:headerfooter]}   {}
{\goto{\externalfigure[demonstration.pdf][page=7]}[cap:parlinevspace]}  {}
{\goto{\externalfigure[demonstration.pdf][page=8]}[cap:fontscol]}       {}
\stopcombination

\page

\startcombination[2*4]
{\goto{\externalfigure[demonstration.pdf][page=9]}[sec:colors] }        {}
{\goto{\externalfigure[demonstration.pdf][page=10]}[sec:langdoc]}       {}
{\goto{\externalfigure[demonstration.pdf][page=11]}[sec:interactivity]} {}
{\goto{\externalfigure[demonstration.pdf][page=12]}[cap:sections]}      {}
{\goto{\externalfigure[demonstration.pdf][page=13]}[cap:emphword]}      {}
{\goto{\externalfigure[demonstration.pdf][page=14]}[cap:emphword]}      {}
{\goto{\externalfigure[demonstration.pdf][page=15]}[cap:framed]}        {}
{\goto{\externalfigure[demonstration.pdf][page=16]}[cap:lines]}         {}
\stopcombination

\page

\startcombination[2*4]
{\goto{\externalfigure[demonstration.pdf][page=17]}[cap:quotation]}     {}
{\goto{\externalfigure[demonstration.pdf][page=18]}[cap:maths]}         {}
{\goto{\externalfigure[demonstration.pdf][page=19]}[cap:itemize]} {}
{\goto{\externalfigure[demonstration.pdf][page=20]}[cap:descenum]}{}
{\goto{\externalfigure[demonstration.pdf][page=21]}[cap:tabulate]}{}
{\goto{\externalfigure[demonstration.pdf][page=22]}[cap:table]}   {}
{\goto{\externalfigure[demonstration.pdf][page=23]}[cap:figures]} {}
{\goto{\externalfigure[demonstration.pdf][page=24]}[cap:figures]} {}
\stopcombination

\page

\startcombination[2*4]
{\goto{\externalfigure[demonstration.pdf][page=25]}[cap:floats]}     {}
{\goto{\externalfigure[demonstration.pdf][page=26]}[cap:floats]}     {}
{\goto{\externalfigure[demonstration.pdf][page=27]}[cap:columns]}    {}
{\goto{\externalfigure[demonstration.pdf][page=28]}[cap:sections]}   {}
{\goto{\externalfigure[demonstration.pdf][page=29]}[cap:macrostruct]}{}
{\goto{\externalfigure[demonstration.pdf][page=30]}[cap:covers]}     {}
{\goto{\externalfigure[demonstration.pdf][page=31]}[cap:others]}  {}
{\goto{\externalfigure[demonstration.pdf][page=32]}[cap:sections]}{}
\stopcombination

\page

\startcombination[2*4]
{\goto{\externalfigure[demonstration.pdf][page=33]}[cap:toc]}     {}
{\goto{\externalfigure[demonstration.pdf][page=34]}[cap:abbr]}    {}
{\goto{\externalfigure[demonstration.pdf][page=35]}[cap:notes]}   {}
{\goto{\externalfigure[demonstration.pdf][page=36]}[cap:margin]}  {}
{\goto{\externalfigure[demonstration.pdf][page=37]}[cap:attachment]}{}
{\goto{\externalfigure[demonstration.pdf][page=38]}[cap:refint]}    {}
{\goto{\externalfigure[demonstration.pdf][page=39]}[cap:refext]}    {}
{\goto{\externalfigure[demonstration.pdf][page=40]}[cap:refbiblio]} {}
\stopcombination

\page

\startcombination[2*2]
{\goto{\externalfigure[demonstration.pdf][page=41]}[cap:sections]}  {}
{\goto{\externalfigure[demonstration.pdf][page=42]}[cap:lists]}     {}
{\goto{\externalfigure[demonstration.pdf][page=43]}[cap:index]}     {}
{\goto{\externalfigure[demonstration.pdf][page=44]}[cap:covers]}    {}
\stopcombination

\stop

\stopsection

% * END

\stopchapter                                            
                                                        
\stopcomponent                                          
                                                        
%%% Local Variables:
%%% mode: ConTeXt
%%% mode: auto-fill
%%% TeX-master: "../introCTX_fra.tex"
%%% coding: utf-8-unix
%%% End:

 

