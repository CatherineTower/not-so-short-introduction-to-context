\startcomponent 02-03-05_Typeset_Interactivity

\environment introCTX_env_00

\startchapter
  [
reference=sec:interactivity,
title=Interactivité,
  ]

\TocChap

{\externalfigure[demonstration.pdf][page=11,background=color,backgroundcolor=white]}

% **   Section 3 Interactivity

Seuls les documents électroniques peuvent être interactifs, mais tous les documents {\em électroniques} ne le sont pas.  Un document électronique est un document qui est stocké dans un fichier informatique et qui peut être ouvert et lu directement à l'écran. En revanche, un document électronique équipé de fonctionnalités qui permettent à l'utilisateur d'{\em interagir} avec lui est interactif, c'est-à-dire qu'on peut faire plus que le lire. Il y a interactivité, par exemple, lorsque le document comporte des boutons permettant d'effectuer une action, ou des liens permettant de passer à un autre point du document, ou à un document externe~; ou lorsque le document comporte des zones où l'utilisateur peut écrire, ou des vidéos ou des clips audio qui peuvent être lus, etc.

Tous les documents générés par \ConTeXt\ sont électroniques (puisque \ConTeXt\ génère un PDF qui est par définition un document électronique), mais ils ne sont pas toujours interactifs. Pour leur conférer une interactivité, il est nécessaire de l'indiquer expressément, comme le montre la section suivante.

Il faut cependant savoir que, même si \ConTeXt\ génère un PDF interactif, pour apprécier cette interactivité, nous avons besoin d'un lecteur de PDF capable de le faire, car tous les lecteurs de PDF ne nous permettent pas d'utiliser des hyperliens, des boutons et autres éléments similaires propres aux documents interactifs.


% ***  Section Enabling interactivity in documents

\startsection
[title=Permettre l'interactivité dans les documents]
\PlaceMacro{setupinteraction}

\ConTeXt\ n'utilise pas de fonctions interactives par défaut, sauf indication expresse, ce qui est normalement fait dans le préambule du document. La commande qui active cet utilitaire est~:

\placefigure [force,here,none] [] {}
{\startDemoI
\setupinteraction[state=start]
\stopDemoI}

Normalement, cette commande ne devrait être utilisée qu'une seule fois et dans le préambule du document lorsque nous voulons générer un document interactif. Mais en fait, nous pouvons l'utiliser aussi souvent que nous le souhaitons en modifiant l'état d'interactivité du document. La commande \MyKey{state=start} active l'interactivité, tandis que \MyKey{state=stop} la désactive, de sorte que nous pouvons désactiver l'interactivité dans certains chapitres ou {\em parties} de notre document comme nous le souhaitons.


\startSmallPrint

Je ne vois aucune raison pour laquelle nous voudrions avoir des parties non interactives dans des documents qui sont interactifs. Mais ce qui est important dans la philosophie de \ConTeXt\, c'est que quelque chose soit techniquement possible, même s'il est peu probable que nous l'utilisions, et elle offre donc une procédure pour le faire. C'est cette philosophie qui donne à \ConTeXt\ tant de possibilités, et empêche une simple introduction comme celle-ci d'être {\em brève}.

\stopSmallPrint

Une fois l'interaction établie~:


\startitemize

\item Certaines commandes \ConTeXt\ incluent naturellement des hyperliens telles que~:


\startitemize

\item Les commandes de création de tables des matières~: en cliquant sur une entrée de la table des matières, on accède à la page où commence la section en question.

\item Les commandes de références internes que nous avons vues dans la première partie de ce chapitre~: cliquer dessus permet de sauter automatiquement à la cible de la référence.

\item Les notes de bas de page et notes de fin de texte où un clic sur l'ancre de la note dans le corps du texte principal nous amènera à la page où la note elle-même est écrite, et un clic sur la marque de la note dans le texte de la note nous amènera au point du texte principal où l'appel a été fait.

\item idem pour les index, etc.

\stopitemize

\item D'autres commandes sont activées car spécialement conçues pour les documents interactifs, comme les présentations. Celles-ci utilisent de nombreux outils associés à l'interactivité tels que des boutons, des menus, des superpositions d'images, des sons ou des vidéos intégrés, etc. L'explication de tout cela serait trop longue et de plus, les présentations sont un type de document assez particulier. C'est pourquoi, dans les lignes qui suivent, je décrirai une seule fonctionnalité associée à l'interactivité~: les hyperliens.

\stopitemize

\stopsection

% ***  Section Basic configuration for interactivity

\startsection
[title=Configuration de base pour les éléments interactifs]

\tex{setupinteraction}, en plus d'activer ou de désactiver l'interaction, nous permet de configurer certains éléments qui y sont liés~; principalement, mais pas seulement, la couleur et le style des liens. Cela se fait par le biais des options de commande suivantes~:

\startitemize

\item {\tt color}~: contrôle la couleur {\em par défaut} des liens.

\item {\tt contrastcolor}~: détermine la couleur des liens lorsque leur cible se trouve sur la même page que l'origine. Je recommande que cette option soit toujours définie sur le même contenu que la précédente.
  
\item {\tt style}~: contrôle le style du lien.

\item {\tt title, subtitle, author, date, keyword}~: Les valeurs attribuées à ces options seront converties en métadonnées du PDF généré par \ConTeXt.

\item {\tt click}~: Cette option contrôle si le lien doit être mis en évidence lorsqu'il est cliqué.

\stopitemize

\stopsection


% **   Section 5 Creating bookmarks in the final PDF]


\startsection
[title=Création de signets dans le PDF final]

Les fichiers PDF peuvent avoir une liste de signets internes qui permet au lecteur de voir le contenu du document dans une fenêtre spéciale du programme de visualisation PDF, et de se déplacer dans le document en cliquant simplement sur chacune des sections et sous-sections.

Par défaut, \ConTeXt\ ne fournit pas au PDF de sortie une liste de signets, mais pour qu'il le fasse, il suffit d'inclure la commande \PlaceMacro{placebookmarks}\tex{placebookmarks}, dont la syntaxe est~:


\placefigure [force,here,none] [] {}
{\startDemoI
\placebookmarks[ListeDeSections][SectionOuverteparDéfaut]
\stopDemoI}

où {\em ListeDeSections} est une liste des niveaux de section, séparées par des virgules, qui doivent apparaître dans la table des matières.  {\em SectionOuverteparDéfaut}, optionnel, indique le niveau de section que le visualisateur de PDF mettre en évidence par défaut (les autres niveaux étant {\em repliés}). Gardez à l'esprit les observations suivantes concernant cette commande~:

\startitemize

% Garulfo : faux
% \item D'après mes tests, \tex{placebookmarks} ne fonctionne pas s'il se trouve dans le préambule du document. Mais, dans le corps du document (entre \tex{starttext} et \tex{stoptext}, ou entre \tex{startproduct} et \tex{stopproduct}), l'endroit où vous le placez n'a pas d'importance~: la liste des signets inclura également les sections ou sous-sections précédant la commande. Cependant, je pense que la chose la plus raisonnable pour qu'un fichier source soit compris correctement, est de placer la commande au début.

% TODO Garulfo  ?????
\item Les types de section définis par l'utilisateur (avec \tex{definehead}) ne sont pas toujours situés au bon endroit dans la liste des signets. Il est préférable de les exclure.

\item Si le titre d'une section comprend une note de fin ou de bas de page, le texte de la note de bas de page est considéré comme faisant partie du signet.

\item En guise d'argument, au lieu d'une liste de sections, on peut simplement indiquer le mot symbolique \MyKey{all} qui, comme son nom l'indique, inclura toutes les sections~; cependant, d'après mes tests, ce mot, en plus de ce qui est certainement des sections, inclut des textes placés là avec certaines commandes de non-sectionnement, de sorte que la liste résultante est quelque peu imprévisible.

\stopitemize

Tous les programmes de lecture de PDF ne nous permettent pas d'afficher les signets~; et beaucoup de ceux qui le font n'ont pas cette fonction activée par défaut. Par conséquent, pour vérifier le résultat de cette fonction, nous devons nous assurer que notre programme de lecture de PDF prend en charge cette fonction et qu'elle est activée. Je crois me souvenir qu'Acrobat, par exemple, n'affiche pas les signets par défaut, bien qu'il existe un bouton dans sa barre d'outils pour les afficher.


\stopsection


\stopchapter

\stopcomponent


%%% Local Variables:
%%% mode: ConTeXt
%%% mode: auto-fill
%%% coding: utf-8-unix
%%% TeX-master: "../introCTX_fra.tex"
%%% End:
%%% vim:set filetype=context tw=72 : %%%
