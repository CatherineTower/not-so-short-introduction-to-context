\startcomponent 02-04-01-04_Typeset_Framed

\environment introCTX_env_00

%==============================================================================

\startsection
  [title=Encadrements,
  reference=sec:typset:framed]

\TocChap


% ADDED by Garulfo BEGIN   <<<<<<<<<<<<<<<<<<<<<<<<<<<<<<<<<<<<<<<<<<<<<<
\startsubsubsection[title=Premiers exemples]

Nous allons voir quelques exemples, simples pour commencer.

\placefigure [force,here,none] [] {}
{\startDemoHW%
\defineframed
  [MonCadre]
  [background=color,
   backgroundcolor=darkred,
   foregroundcolor=white,
   foregroundstyle=\bfc]%
A
\MonCadre{coucou 1}\hfill
\MonCadre[offset=2mm]{offset}\hfill
\MonCadre{coucou 2}\hfill
\MonCadre[frameoffset=2mm]{frameoffset}\hfill   % conserve la position du mot
\MonCadre[backgroundoffset=2mm]{backgroundoffset}
\stopDemoHW}

Il y a ensuite tout une série de personnalisation sur le trait du cadre, sur la distance entre le texte et le cadre (en haut, en bas, à gauche, à droite), à découvrir au fur et à mesure de vos besoins. 

\placefigure [force,here,none] [] {}
{\startDemoVN%
\defineframed[MonTitre]
  [frame=off, topframe=on, 
   leftframe=on,
   rulethickness=2pt,
   foregroundstyle=\bfa, 
   framecolor=darkred]%
\MonTitre{Superbe titre 1}
\blank
\MonTitre
  [loffset=1em,toffset=2pt]
  {Superbe titre 2}
\blank
\MonTitre
  [corner=16,frame=on,offset=2pt]
  {Superbe titre 3}
\stopDemoVN}

\stopsubsubsection

%=======================================

\startsubsubsection[title=Retour à la ligne et alignement]

Attention, par défaut, le cadre ne fait aucun retour à la ligne. Il faut indiquer une valeur à {\tt align} (voir \in{section}[sec:setupalign] pour les valeurs à considérer, normal fait référence à un texte justifié, et \in{section}[sec:horizontaltolerance] pour les options de tolérance). Dans l'exemple on utilise {\tt align=\{normal,verytolerant\}}.

\placefigure [force,here,none] [] {}
{\startDemoVN%
\startbuffer
Si le texte est un peu trop grand, par défaut il ne sera pas coupé
\stopbuffer%
\defineframed
  [MonCadre]
  [background=color,
   backgroundcolor=darkred,
   foregroundcolor=white,
   foregroundstyle=\bfa]%
\MonCadre{\getbuffer} \blank
\MonCadre[width=5cm]{\getbuffer} \blank
\MonCadre[width=5cm,align=normal] {\getbuffer} \blank
\MonCadre[width=5cm,
          align={normal,verytolerant},
          backgroundcolor=darkgreen]
         {\getbuffer}
\stopDemoVN}

\stopsubsubsection


%=======================================

\startsubsubsection[title=Largeur]

Un point concernant le paramètre {\tt width}. La dimension de la largeur peut être indiquée directement (en cm, en pt, en fonction d'autre longueur par exemple \MyKey{0.5\backslash textwidth}). 

Des dimensions automatiques sont proposées par \ConTeXt.
{\tt broad} fait référence à la largeur totale disponible pour la zone de texte,
{\tt local} fait référence à la largeur totale disponible pour la zone en cours,
et {\tt fit} colle à la largeur du texte contenu par le cadre, 

\startbuffer[a7-testframedwidth]
\setuppapersize[A7,landscape]
\setupbodyfont[8pt]
\showframe
\setupframed[framecolor=blue,align=middle,style=tt]
\definenarrower[MyNarrow][left=1cm,default=left]

\starttext

\framed[width=4cm]   {width=6cm}
\framed[width=0.5\textwidth]   {width=0.5\backslash textwidth}
\framed[width=broad] {width=broad}
\framed[width=local] {width=local}
\framed[width=fit]   {width=fit}

\startitemize[packed]
\item \framed[width=broad] {width=broad}
\item \framed[width=local] {width=local}
\item \framed[width=fit]   {width=fit}
\stopitemize

\startMyNarrow
\startitemize[packed]
\item \framed[width=broad] {width=broad}
\item \framed[width=local] {width=local}
\item \framed[width=fit]   {width=fit}
\stopitemize
\stopMyNarrow

\stoptext

\stopbuffer

\savebuffer[list=a7-testframedwidth,file=ex_framedwidth.tex,prefix=no]
\placefigure [force,here,none] [] {}{\typesetbuffer[a7-testframedwidth][frame=on,page=1,background=color,backgroundcolor=white]
\attachment
  [file={ex_framedwidth.tex},
   title={exemple ex_framedwidth}]}

\stopsubsubsection

%=======================================

\startsubsubsection[title=Positionnement par rapport à la ligne de base]

\placefigure [force,here,none] [] {}
{\startDemoHW%
\defineframed[MonCadre][width=2.25cm,align=middle]
\define[1]\DemoLoc{\ruledhbox{%
    {\getbuffer \MonCadre[location=#1]
     {location\\ \color[darkmagenta]{\bf #1}\\location}}}}
\setupbodyfont[14pt]
%\showboxes
\startbuffer
\blackrule[height=max,depth=0pt,width=3mm]%
\blackrule[height=0pt,depth=max,width=3mm]
\stopbuffer

\strut
\DemoLoc{empty}  \dontleavehmode \DemoLoc{keep}    \dontleavehmode 
\DemoLoc{depth}  \dontleavehmode \DemoLoc{bottom}  \dontleavehmode
\DemoLoc{low} 
\blank[big]\strut
\DemoLoc{middle} \dontleavehmode \DemoLoc{lohi}    \dontleavehmode 
\DemoLoc{line}
\blank[big]\strut
\DemoLoc{top}    \dontleavehmode \DemoLoc{height}  \dontleavehmode 
\DemoLoc{high}   \dontleavehmode \DemoLoc{formula} \dontleavehmode 
\DemoLoc{hanging}
\stopDemoHW}

Plus complexe, ci-dessus les différents cas possible pour le paramètre {\tt location}. Vous voyez que de nombreux cas sont possibles, afin d'aligner le haut ou le bas de la boîte sur la ligne de base du texte, ou bien encore, souvent plus utile, d'aligner la première ou la dernière ligne de base de la boîte avec la ligne de base du texte.

\stopsubsubsection


%=======================================

%TODO Garulfo strut

\startsubsubsection[title=Attention au paramètre {\tt strut}]

L'option {\tt strut} force la première ligne de l'encadré à prendre tout la hauteur et toute la profondeur que la police en cours lui permet de prendre. Cela permet de \quotation{donner une hauteur de ligne standard}. Si vous avez des difficultés parfois à positionner correctement vos {\tt framed} par rapport à la ligne de base, ou bien si vous avez des difficultés d'alignement / positionnement vertical, pensez à regarder l'effet de {\tt strut}. Par exemple~:

Voyez l'effet comment avec {\tt strut=no} la première ligne à une hauteur réduire par rapport à la version  {\tt strut=yes} ce qui perturbe le positionnement par rapport à l'effet désiré. 

\placefigure [force,here,none] [] {}
{\startDemoHW%
\defineframed[MonCadre][width=2.25cm,align=middle]
\define[1]\DemoLoc{\ruledhbox{%
    {\getbuffer \MonCadre[location=#1]
     {location\\ \color[darkmagenta]{\bf #1}\\location}}}}
\setupbodyfont[14pt]
%\showboxes
\startbuffer
\blackrule[height=max,depth=0pt,width=3mm]%
\blackrule[height=0pt,depth=max,width=3mm]
\stopbuffer

\setupframed[MonCadre][strut=yes]
\strut
{\tt strut=yes}
\DemoLoc{bottom}    \dontleavehmode \DemoLoc{top}

{\tt strut=no}
\setupframed[MonCadre][strut=no]
\DemoLoc{bottom}    \dontleavehmode \DemoLoc{top}
\stopDemoHW}

\stopsubsubsection

% ADDED by Garulfo END     <<<<<<<<<<<<<<<<<<<<<<<<<<<<<<<<<<<<<<<<<<<<<<

\stopsubsection



\stopsection

%==============================================================================

\stopcomponent

%%% TeX-master: "../introCTX_fra.tex"
