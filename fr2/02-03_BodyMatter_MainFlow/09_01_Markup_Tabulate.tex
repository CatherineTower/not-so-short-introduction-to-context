\startcomponent 02-02-01-09_Markup_Tabulate

\environment introCTX_env_00

%==============================================================================

\startsection
  [title=Textes tabulés,
  reference=sec:mkp:tabulate]


\TocChap

\tex{starttabulate}
\tex{startlegend}
\tex{startfact}

\startsubsection
  [title={Idées générales sur les tableaux et leur emplacement dans le document}]

Les tableaux sont des objets structurés qui contiennent du texte, des formules ou même des images disposés dans une série de {\em cellules} qui permettent de visualiser graphiquement la relation entre le contenu de chaque cellule. Pour ce faire, les informations sont organisées en lignes et en colonnes~: toutes les données (ou entrées) d'une même ligne ont une certaine relation entre elles, de même que toutes les données (ou entrées) d'une même colonne. Une cellule est l'intersection d'une ligne et d'une colonne, comme le montre la \in{figure}[fig:table].


\placefigure
  [right]
  [fig:table]
  {Exemple de table simple}
{\bTABLE
\setupTABLE[offset=2pt]
\setupTABLE[r][1][style=bold]
\setupTABLE[r][3][background=color,backgroundcolor=lightcyan]
\setupTABLE[c][3][background=color,backgroundcolor=lightcyan]
\setupTABLE[3][3][background=color,backgroundcolor=middlecyan]
\setupTABLE[c][1][style=bold]
\bTR \bTD         \eTD\bTD Colonne 1\eTD\bTD Colonne 2\eTD\bTD Colonne 3\eTD\eTR
\bTR \bTD Ligne 1 \eTD \bTD  \eTD \bTD \eTD \bTD \eTD \eTR
\bTR \bTD Ligne 2 \eTD \bTD  \eTD \bTD \eTD \bTD \eTD \eTR
\bTR \bTD Ligne 3 \eTD \bTD  \eTD \bTD \eTD \bTD \eTD \eTR
\bTR \bTD Ligne 4 \eTD \bTD  \eTD \bTD \eTD \bTD \eTD \eTR
\eTABLE}
  
Les tableaux sont idéaux pour afficher des textes ou des données qui sont liés les uns aux autres, car comme chacun est enfermé dans sa propre cellule, même si son contenu ou celui des autres cellules change, la position relative de l'un par rapport aux autres ne changera pas.

De toutes les tâches liées à la composition d'un texte, la création de tableaux est la seule qui, à mon avis, est plus facile à réaliser dans un programme graphique (type traitement de texte) que dans \ConTeXt. Parce qu'il est plus facile de  {\em dessiner} le tableau (ce que l'on fait dans un programme de traitement de texte) que de le  {\em décrire} (ce que l'on fait quand on travaille avec \ConTeXt). Chaque changement de ligne et de colonne nécessite la présence d'une commande, ce qui signifie que l'implémentation du tableau prend beaucoup plus de temps, au lieu de simplement dire combien de lignes et de colonnes nous voulons.


\ConTeXt\ a trois mécanismes différents pour produire des tableaux~; l'environnement {\tt tabulate} qui est recommandé pour les tableaux simples et qui est le plus directement inspiré des tableaux \TeX\~; les tableaux dits {\em naturels (natural tables)}, inspirés des tableaux HTML, adaptés aux tableaux ayant des besoins de conception particuliers où, par exemple, toutes les lignes n'ont pas le même nombre de colonnes~; et les tableaux dits {\em extrèmes (extreme tables)}, clairement basés sur XML et recommandés pour les tableaux moyens ou longs qui prennent plus d'une page. Des trois, je n'expliquerai que les deux premiers. Les tableaux naturels sont également raisonnablement bien expliqués dans la \quotation{\ConTeXt\ Mark IV an excursion}, et pour les {\em extreme tables} il y a une \goto{documentation officielle}[url(https://www.pragma-ade.com/general/manuals/xtables-mkiv.pdf)] à leur sujet dans la documentation de la distribution \suite-.

Il se passe quelque chose de similaire à ce qui se passe avec les images pour les tableaux~: il suffit d'écrire les commandes nécessaires à un moment donné du document pour générer un tableau et celui-ci sera inséré à cet endroit précis, ou bien nous pouvons utiliser la commande \PlaceMacro{placetable} \tex{placetable} pour insérer un tableau. Cette méthode présente quelques avantages~:


\startitemize

\item \ConTeXt\ numérote le tableau et l'ajoute à la liste des tableaux permettant des références internes au tableau (par sa numérotation), l'incluant dans un éventuel index des tableaux.

\item Nous gagnerons en flexibilité dans le placement des tableaux dans le document, facilitant ainsi la tâche de la pagination.

\stopitemize


Le format de \tex{placetable} est similaire à ce que nous avons vu pour
\tex{placefigure} (voir \in{section}[sec:placefigure])~:

\placefigure [force,here,none] [] {}{
\startDemoI
\placetable[Options] [Etiquette] {Titre} {CommandeInsererImage}
\stopDemoI}

Je renvoie aux sections \in{}[sec:placingobjects] et \in{}[sec:confcaptions] concernant les options relatives au placement des tables et à la configuration du titre. Parmi les options, il y en a une, cependant, qui semble être conçue exclusivement pour les tableaux. Il s'agit de l'option \MyKey{split} qui, lorsqu'elle est définie, autorise \ConTeXt\ à étendre le tableau sur deux pages ou plus, auquel cas le tableau ne peut plus être un objet flottant simple.

De manière générale, nous pouvons définir la configuration des tableaux avec la commande \PlaceMacro{setuptables}\tex{setuptables}. De plus, comme pour les images, il est possible de générer un index des tables avec la commande
\PlaceMacro{placelistoftables} \tex{placelistoftables} ou \PlaceMacro{completelistoftables} \tex{completelistoftables}. Voir à ce sujet \in{section} [sec:variouslists].


\stopsubsection

\startsubsection
  [title={starttabulate},]
  
\PlaceMacro{starttabulate}

Les tableaux les plus simples sont ceux réalisés avec l'environnement {\em
  tabulate} dont le format est~:


\placefigure [force,here,none] [] {}{
\startDemoI
\starttabulate[Configuration de la disposition des colonnes du tableau]
... % Table contents
...
...
\stoptabulate
\stopDemoI}


Où l'argument pris entre crochets décrit (en code) le nombre de colonnes que le tableau aura, et indique (parfois indirectement) leur largeur. Je dis que l'argument décrit le dessin {\em en code}, parce qu'à première vue il semble très cryptique~: il consiste en une séquence de caractères, chacun ayant une signification particulière. Je vais l'expliquer petit à petit et par étapes dans la \in{section}[sec:cmptabulate], car je pense que de cette façon il est plus facile à comprendre.

\startSmallPrint

C'est le cas typique dans lequel le nombre énorme d'aspects que nous pouvons configurer signifie que nous avons besoin de beaucoup de texte pour le décrire. Cela semble être diablement difficile. En fait, pour la plupart des tableaux construits dans la pratique, les points 1 et 2 sont suffisants. Les autres sont des possibilités supplémentaires dont il est utile de connaître l'existence, mais qu'il n'est pas indispensable de connaître pour composer un tableau.

\stopSmallPrint

\tex{startlegend} \PlaceMacro{startlegend}
Dans un contexte mathématique, \TeX\ applique des règles différentes de sorte qu'aucun texte normal ne peut être écrit. Cependant, il arrive qu'une formule soit accompagnée d'une description des éléments qui la composent. Pour cela, il existe l'environnement \tex{startlegend} qui permet de placer un texte normal dans un contexte mathématique.

\stopsubsection

\stopsection


%==============================================================================

\stopcomponent

%%% TeX-master: "../introCTX_fra.tex"
