\startcomponent 02-04-02-08_Typeset_Lists

\environment introCTX_env_00

%==============================================================================

\startsection
  [title=Lists,
  reference=sec:typset:lists]


\TocChap

% ** Subsection liste combinées

\startsubsection
  [title=Listes combinées]

Une liste combinée est, comme son nom l'indique, une liste qui combine des éléments provenant de différentes listes précédemment définies. Par défaut, \ConTeXt\ définit une liste combinée pour les tables de contenu dont le nom est \MyKey{content}, mais nous pouvons créer d'autres listes combinées avec \PlaceMacro{definecombinedlist} \tex{definecombinedlist} dont la syntaxe est~:

\placefigure [force,here,none] [] {}{
\startDemoI
\definecombinedlist[Nom][Listes][Options]
\stopDemoI}

où~:

\startitemize[packed]

\item {\em Nom}~: est le nom que portera la nouvelle liste combinée.

\item {\em Listes}~: désigne les noms des listes à combiner, séparés par des virgules.

\item {\em Options}~: Options de configuration de la liste. Elles peuvent être indiquées au moment de la définition de la liste, ou, de préférence, lorsque la liste est invoquée. Les principales options (qui ont déjà été expliquées) sont les suivantes~: {\tt criterium} (\in{sous-section}[sec:criteriumlist] de \in{section} [sec:placecontent]) et {\tt alternative} (dans \in{sous-section}[sec:liste des alternatives] de la même section).

\stopitemize

Un effet collatéral de la création d'une liste combinée avec \tex{definecombinedlist} est qu'elle crée également une commande appelée \tex{placeListNom} qui sert à invoquer la liste, c'est-à-dire à l'inclure dans le fichier de sortie. Ainsi, par exemple,

\placefigure [force,here,none] [] {}{
\startDemoI
\definecombinedlist[TdM]
\definecombinedlist[content]
\stopDemoI}

créera les commandes \tex{placeTdM} et \tex{placecontent}.

Mais attendez, \tex{placecontent} ! N'est-ce pas la commande qui est utilisée pour créer une table des matières {\em normale}~? En effet, cela signifie que la table des matières standard est en fait créée par \ConTeXt\ au moyen de la commande suivante~:

\placefigure [force,here,none] [] {}{
\startDemoI
\definecombinedlist
  [content]
  [part, chapter, section, subsection,
    subsubsection, subsubsubsection,
    subsubsubsubsection]
\stopDemoI}

Une fois notre liste combinée définie, nous pouvons la configurer (ou la reconfigurer) avec \tex{setupcombinedlist} qui permet les options déjà expliquées {\tt criterium} (voir \in{sous-section}[sec:criteriumlist] dans \in{section}[sec:placecontent]) et {\tt alternative} (voir \in{subsection}[sec:alternativelist] dans la même section), ainsi que l'option {\tt list} pour {\em modifier} les listes incluses dans la liste combinée.

\startSmallPrint

La liste officielle des commandes \ConTeXt\ (voir \in{section}[sec:qrc-setup-fr]) ne mentionne pas l'option {\tt list} parmi les options autorisées pour \tex{setupcombinedlist}, mais elle est utilisée dans plusieurs exemples d'utilisation de cette commande dans le wiki (qui, par ailleurs, ne la mentionne pas non plus dans la page consacrée à cette commande sauf en exemple). J'ai également vérifié que l'option fonctionne.

\stopSmallPrint

\stopsubsection


\stopsection

%==============================================================================

\stopcomponent

%%% TeX-master: "../introCTX_fra.tex"
