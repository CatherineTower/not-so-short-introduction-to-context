  %==============================================================================
  % STRUCTURES DEFINED BY ME
  
\definedescription
    [description]
    [
      alternative=serried,
      width=fit,
      distance=1ex,
      headstyle=\bf
    ]

 \definedescription
    [DescOptions][description]
    [
      headstyle=\bf\tt,
      alternative=left,
      width=0.3\textwidth,
    ] 
  \definedescription
    [semitable]
    [
      alternative=left,
      width=fit,
      distance=1cm,
      headcolor=darkmagenta,
      headstyle=\bf\tt,
    ]

  %==============================================================================
  % ENVIRONMENTS DEFINED BY ME

  % DoubleExample: For two column examples. It gave some errors when
  % defining it with \definestartstop (due to my lack of knowledge) and
  % in the end what I did was to define a TeX style command for opening
  % it, and another command for closing it.


  \def\startDoubleExample{%
    \startframedtext
      [frame=off]
      \switchtobodyfont[small]
      \setuptyping[style=tt]
      \startcolumns
        [
          n=2,
          tolerance=verytolerant,
          distance=0.5cm,
          separator=rule,
          rulecolor=darkmagenta
        ]
  }

  \def\stopDoubleExample{\stopcolumns\stopframedtext}

  %------------------------------------------------------------------------------

  % Small print
  \definestartstop[SmallPrint]
                  [
                    before={\startnarrower\switchtobodyfont[small,ss]},
                    after={\stopnarrower},
                  ]

\definehighlight [people]       [style=\cap] 

  %------------------------------------------------------------------------------
  % Commands and abbreviations

  % Assumption: Print the assumption icon in the margin
  \define\Conjecture{%
    \inouter[line=1]{\externalfigure[conjecture.pdf][width=1cm]}}

  % Doubt: Print the doubt icon in the margin
  \define\Doubt{%
    \inouter[line=1]{\externalfigure[doubt.pdf][width=1cm]}}

  % example: Show in red and in small print the result of compiling an
  % earlier code snippet.
  \define[1]\example{\color[red]{\tfx#1}}

  % ConTeXt Standalone
  \def\suite-{\quotation{\ConTeXt\ Standalone}}

  % cmd: to be used in place of tex when, in the source file, there is a
  % line break in an argument.
  \define[1]\cmd{
    {\en\tt\color[darkmagenta]{\backslash #1}}}

  % PalClave:  tt and guillemets (angle quotes)
  % \define[1]\PalClave{«{\tt #1}»}

  % MyKey: tt and double inverted commas
  \define[1]\MyKey{\quotation{\tt #1}}

  

  % Partial chapter table of contents:
%  \def\Separate#1{#1;}
\def\TocChap{%
\setupcombinedlist[localcontent][list={section}]
\setupinterlinespace[2.4ex] 
{\bf Table des matières}
\placecombinedlist[localcontent][alternative=c,criterium=local]
\blank[2*big]}

  % Index
  \defineregister[macro]
  \define[1]\PlaceMacro{\macro[#1]{\backslash #1}}

  \definecombinedlist                     % Let us define our list (called a Toc)
  [Toc]
  [part, chapter, section]
  [level=subsection, alternative=c,criterium=text]
 
  \definecombinedlist                     % Let us define our list (called a Toc)
  [content]
  [part, chapter]
  
  
\setupheadtext [fr] [Toc=Table des matières]

%------------------------------------

\defineitemgroup[commanddesc][itemize]

\setupitemgroup [commanddesc]
  [1]
  [packed,joinedup]
  [
  option=headintext,
  headstyle=bold,
  inbetween={\blank[nowhite]},
  color=ctr1color,
  distance=2em,
  textdistance=2em,
  headstyle=\tt\bf,
  headcolor=darkmagenta,
  symbol=symitemize1,
]
